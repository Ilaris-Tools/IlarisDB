\newcommand{\kreaturdetailaaswolf}{\kreatur{Aaswolf}{pervertierter Wolf}{gfx/kreaturen/tier}{\kreaturkampfwerte{4}{2}{8}{2}\trennlinie \kreaturinfo{Angepasst II (Dunkelheit)}{Angepasst I/II (Umgebung) Durch deine Spezies oder langjährige Erfahrung hast du dich an eine bestimmte Umgebung oder Umweltbedingung gewöhnt. Abzüge durch diese Umgebung (Beispiele auf S. 38), insbesondere im Kampf, sinken für dich um eine/zwei Stufen. Die Kosten für Angepasst legt der Spielleiter fest, wobei er sich an der Häufigkeit der Umgebung orientieren sollte. Zu allgemein gefasste Umgebungen wie „unsicherer Untergrund“ sollte er nicht zulassen. Beispiele für Angepasst sind: • Dunkelheit: verringert Abzüge durch schlechte Lichtverhältnisse (40 EP pro Stufe) • Schnee: verringert Abzüge durch schneebedeckten oder eisigen Untergrund (20 EP pro Stufe) • Wasser: verringert Abzüge durch knie{-} oder hüfttiefes Wasser und unter Wasser (20 EP pro Stufe) • Wald: verringert Abzüge durch Wurzeln, Gestrüpp und dichtes Unterholz (40 EP pro Stufe) Voraussetzungen: keine/Angepasst I Nachkauf: häufig/selten\newline%
}\kreaturinfo{Immunität (Einfluss)}{Immunität (Fertigkeit): Die Kreatur ist immun gegen alle Zauber, die (auch) der entsprechenden Fertigkeit zugeordnet sind. Davon ausgenommen sind Effekte, die das Wesen nur indirekt betreffen – zum Beispiel erfasst ein Odem auch einen Dämon mit Immunität (Hellsicht) und eine Immunität (Umwelt) schützt nicht vor dem Regen, der mittels Wettermeisterschaft gerufen wurde.\newline%
}\kreaturinfo{Lichtscheu}{Lichtscheue Kreaturen sind an ein Leben in der Dunkelheit gewöhnt. Allerdings erleiden sie im Licht einer Fackel/eines Lagerfeuers/der Sonne einen Malus von –2/–4/–8 auf alle Proben.\newline%
}\kreaturvorteile{}\trennlinie \kreaturwaffe{Biss}{0}{6}{9}{2W6+0}{Zerbrechlich, Aasgift (Stufe 20, Verzögerung 4 Aktionen, Wirkungsdauer sofort; das Ziel erleidet eine Wunde; Wölfe und Hunde verwandeln sich binnen 1W3 Tagen auch in Aaswölfe)}\kreaturkampfvorteile{Niederwerfen, Standfest, Sturmangriff}\trennlinie \kreaturattribute{GE 14, KK 12, KO 12, MU 6}\kreaturfertigkeiten{Laufen 24, Pirschen 12, Wachsamkeit 18, Zähigkeit 4}\trennlinie \kreaturinfo{Info}{Varianten:  Alphaaaswolf (WS 5/6; TP, MR und Kampfwerte +4 IN:I +2, wird der Alphaaaswolf besiegt, nehmen auch die restlichen Wölfe die Pfoten in die Hand)}\trennlinie \kreaturinfo{Quelle}{\href{https://www.orkenspalter.de/filebase/index.php?file/2829-hilberts-bestiarium/}{Hilberts Bestarium}}}}
\newcommand{\kreaturdetailagribaal}{\kreatur{Agribaal}{Der Fluch des Eisens; eingehörnter Diener Agrimoths oder Iribaars}{gfx/kreaturen/daemon}{\kreaturkampfwerte{1}{10}{4}{8}\trennlinie \kreaturinfo{Unsichtbarkeit}{Unsichtbarkeit erlaubt es der Kreatur, sich beliebig oft und lange unsichtbar zu machen.\newline%
}\kreaturvorteile{Artefakt beseelen}\trennlinie \kreaturwaffe{Ausweichen}{0}{4}{}{}{}\trennlinie \kreaturinfo{AsP}{32}\kreaturinfo{Dämonisch}{12 (Arcanovi, je einige passende Zauber aus den Bereichen Eigenschaften, Einfluss, Umwelt und den Elementen Feuer, Humus, Erz und Luft))}\trennlinie \kreaturinfo{Beschwörung}{Invocatio}\kreaturinfo{Dienste}{Einfaches Artefakt mit 1 Ladung erschaffen+4, Einfaches Artefakt mit wenigen Ladungen erschaffen+0}\trennlinie \kreaturinfo{Info}{Der Dämon kann einen Gegenstand beseelen und aus diesem ein einfaches, ladungsbasiertes Artefakt erschaffen. Er verliert: WS, GS, Ausweichen Meist kommt es bei solchen Artefakten jedoch zu dämonischen Nebeneffekten nach SL{-}Entscheid.)\newline%
}\trennlinie \kreaturinfo{Quelle}{\href{https://dsaforum.de/viewtopic.php?p=1887303\#p1887303}{Pandämonium}}}}
\newcommand{\kreaturdetailaljallahir}{\kreatur{Al‘Jallahir}{leidenschaftlicher Anführer und Dschinn des Feuers}{gfx/kreaturen/elementar}{\kreaturkampfwerte{8}{12}{10}{14}\trennlinie \kreaturinfo{Resistenz I}{Stichwaffen}\kreaturvorteile{Schreckgestalt I}\trennlinie \kreaturwaffe{Flammensäbel}{2}{12}{16}{4W6+0}{Nachbrennen}\kreaturkampfvorteile{Kommando: Formiert Euch!, Kommando: Haltet Stand!, Kommando: Keine Gefangenen!, Kommando: Kennt Keinen Schmerz!!}\trennlinie \kreaturfertigkeiten{Anführen 16, Betören 12, Einschüchtern 16, Tulamidenlande (Gebräuche) 12, Rhetorik 12, Sinnenschärfe 12, Wachsamkeit 14}\trennlinie \kreaturinfo{AsP}{64}\kreaturinfo{Feuer}{16 (Caldofrigo, Custosigil, Ignifaxius, Ignifugo, Ignisphaero, Leib des Feuers, Manifesto, Pfeil des Feuers, Wand aus Feuer, Warmes Blut)}\trennlinie \kreaturinfo{Beschwörung}{Herbeirufung des Feuers}\kreaturinfo{Dienste}{Kommandos im Kampf (1 Stunde, +4);+0, Soldaten in die Schlacht führen+0, Kampfeslust entfachen ({-}1 Furchtstufe für alle Verbündeten in Rufweite, 1 Stunde, {-}4)+0}\trennlinie \kreaturinfo{Quelle}{\href{https://ilarisblog.wordpress.com/downloads/}{Ilaris Regeln}}}}
\newcommand{\kreaturdetailaltergolem}{\kreatur{Alter Golem}{hinterhältiger Jäger aus Stein}{gfx/kreaturen/mythen}{\kreaturkampfwerte{10}{12}{2}{2}\trennlinie \kreaturinfo{Immunität}{Eigenschaften, Hellsicht, Verwandlung}\kreaturinfo{Angepasst II (Dunkelheit)}{Angepasst I/II (Umgebung) Durch deine Spezies oder langjährige Erfahrung hast du dich an eine bestimmte Umgebung oder Umweltbedingung gewöhnt. Abzüge durch diese Umgebung (Beispiele auf S. 38), insbesondere im Kampf, sinken für dich um eine/zwei Stufen. Die Kosten für Angepasst legt der Spielleiter fest, wobei er sich an der Häufigkeit der Umgebung orientieren sollte. Zu allgemein gefasste Umgebungen wie „unsicherer Untergrund“ sollte er nicht zulassen. Beispiele für Angepasst sind: • Dunkelheit: verringert Abzüge durch schlechte Lichtverhältnisse (40 EP pro Stufe) • Schnee: verringert Abzüge durch schneebedeckten oder eisigen Untergrund (20 EP pro Stufe) • Wasser: verringert Abzüge durch knie{-} oder hüfttiefes Wasser und unter Wasser (20 EP pro Stufe) • Wald: verringert Abzüge durch Wurzeln, Gestrüpp und dichtes Unterholz (40 EP pro Stufe) Voraussetzungen: keine/Angepasst I Nachkauf: häufig/selten\newline%
}\kreaturinfo{Tarnung}{in der Stadt}\kreaturvorteile{Geisterpanzer}\trennlinie \kreaturwaffe{Klauen}{1}{2}{8}{2W6+6}{}\kreaturkampfvorteile{Doppelangriff, Niederwerfen, Standfest, Sturmangriff}\trennlinie \kreaturattribute{GE 6, KK 24, KO 28, MU 12}\kreaturfertigkeiten{Untertauchen 18, Wachsamkeit 10, Zähigkeit 16}\trennlinie \kreaturinfo{Info}{Anfällig gegen Einflusszauber (gegen Zauber der Fertigkeit Einfluss gilt eine MR von 0)}\trennlinie \kreaturinfo{Quelle}{\href{https://dsaforum.de/viewtopic.php?p=1887303\#p1887303}{Allgemeine Gegner}}}}
\newcommand{\kreaturdetailamrifas}{\kreatur{Amrifas}{Der Erderschütterer; neungehörnter Diener Agrimoths, sehr großer Gegner}{gfx/kreaturen/daemon}{\kreaturkampfwerte{18}{20}{2}{3}\trennlinie \kreaturinfo{Magieabweisend}{Zauber wirken auf dich deutlich schwächer. Du ignorierst bei allen Zaubern eine Stufe der spontanen Modifikation Mächtige Magie. Zauber ohne Mächtige Magie haben auf dich keine Wirkung. Voraussetzung: 40 EP Nachkauf: extrem selten\newline%
}\kreaturinfo{Unsichtbarkeit}{Unsichtbarkeit erlaubt es der Kreatur, sich beliebig oft und lange unsichtbar zu machen.\newline%
}\kreaturvorteile{}\trennlinie \kreaturwaffe{Steinschlag}{8}{16}{20}{3W20+0}{Flächenangriff (1 Schritt um das Hauptziel), Niederwerfen ({-}4)}\trennlinie \kreaturfertigkeiten{Pirschen 20, Wachsamkeit 30}\trennlinie \kreaturinfo{AsP}{128}\kreaturinfo{Dämonisch}{24 (alle Zauber des Elements Erz und ähnliche Effekte)}\trennlinie \kreaturinfo{Beschwörung}{Invocatio}\kreaturinfo{Dienste}{Gebäude zum Einsturz bringen+4, Vulkanausbruch verursachen+0, Erdbeben verursachen{-}4}\trennlinie \kreaturinfo{Quelle}{\href{https://dsaforum.de/viewtopic.php?p=1887303\#p1887303}{Pandämonium}}}}
\newcommand{\kreaturdetailarchorchobai}{\kreatur{Achorchobai}{Der Weiße Wurm, der Große Alchimist; viergehörnter Diener Agrimoths, sehr großer Gegner}{gfx/kreaturen/daemon}{\kreaturkampfwerte{15}{14}{4}{4}\trennlinie \kreaturvorteile{Regeneration I, Schreckgestalt II, Verwundbarkeit I (Erz)}\trennlinie \kreaturwaffe{Biss}{0}{2}{16}{4W6+4}{Nachbrennen (Säure), Verschlingen (Bei einem Triumph wird der Gegner verschlungen, 4W20 TP)}\kreaturwaffe{Säure spucken}{8}{}{}{2W6+4}{Nachbrennen (Säure)}\trennlinie \kreaturfertigkeiten{Pirschen 14, Wachsamkeit 16}\kreaturfertigkeiten{Gesteinskunde 3}\trennlinie \kreaturinfo{Beschwörung}{Invocatio}\kreaturinfo{Dienste}{Nach Edelsteinen und {-} metallen suchen (1 Tag, +4)+4 (1 Tag), Tunnel treiben+0 (1 Woche), Metall verhütten{-}4 (1 Tag)}\trennlinie \kreaturinfo{Quelle}{\href{https://dsaforum.de/viewtopic.php?p=1887303\#p1887303}{Pandämonium}}}}
\newcommand{\kreaturdetailarjunioor}{\kreatur{Arjunioor}{Meister der Orkane; achtgehörnter Diener Agrimoths, großer Gegner}{gfx/kreaturen/daemon}{\kreaturkampfwerte{13}{24}{12}{22}\trennlinie \kreaturvorteile{Flugfähig, Zusätzliche Attacke I}\trennlinie \kreaturwaffe{Orkanböe}{8}{18}{22}{1W20+0}{Flächenangriff (180° vor dem Dämon), Zurückstoßen}\kreaturkampfvorteile{Sturmangriff}\trennlinie \kreaturfertigkeiten{Fliegen 20, Sinnenschärfe 28, Wachsamkeit 24}\trennlinie \kreaturinfo{AsP}{128}\kreaturinfo{Dämonisch}{24 (Aeolitus, Aerofugo, Aerogelo, Atemnot, Krähenruf, Kulminatio, Leib des Windes, Orcanofaxius, Orcanosphaero, Orkanwand, Silentium, Tlalucs Odem, Wettermeisterschaft, Windhose, Windstille)}\trennlinie \kreaturinfo{Beschwörung}{Invocatio}\kreaturinfo{Dienste}{Kontrolle über einen Vogelschwarm{-}4 (1 Tag), Sturm heraufbeschwören+0, Landstrich mit Blitz und Hagel verwüsten{-}4}\trennlinie \kreaturinfo{Quelle}{\href{https://dsaforum.de/viewtopic.php?p=1887303\#p1887303}{Pandämonium}}}}
\newcommand{\kreaturdetailatuum}{\kreatur{Atuum}{Der gefräßige Schlund, der Lebensverschlinger; niederer Diener Asfaloths}{gfx/kreaturen/daemon}{\kreaturkampfwerte{4}{12}{3}{2}\trennlinie \kreaturinfo{Aura}{Benebelnder Geruch Zähigkeit (16), jede INI: phase, {-}2 auf Proben für 1 Stunde) \newline%
}\kreaturinfo{Explosion}{4W6}\kreaturvorteile{Regeneration I, Zusätzliche Attacke II}\trennlinie \kreaturwaffe{Tentakel}{2}{4}{8}{2W6+0}{Umklammern ({-}2, 12)}\kreaturwaffe{Maul}{0}{2}{6}{3W6+0}{}\kreaturkampfvorteile{Niederwerfen}\trennlinie \kreaturfertigkeiten{Pirschen 12, Wachsamkeit 10}\trennlinie \kreaturinfo{Beschwörung}{Invocatio}\kreaturinfo{Dienste}{Leiche eines Mordopfers restlos verschlingen+4, Ort, an dem Tod und Fäulnis herrschen, bewachen+0 (1 Jahr), Anderen Ort bewachen{-}4 (1 Wochen)}\trennlinie \kreaturinfo{Quelle}{\href{https://dsaforum.de/viewtopic.php?p=1887303\#p1887303}{Pandämonium}}}}
\newcommand{\kreaturdetailazuzar}{\kreatur{Azuzar}{mächtiger freier Geistergrolm in den Katakomben Vinsalts}{gfx/kreaturen/geist}{\kreaturkampfwerte{4}{16}{5}{4}\trennlinie \kreaturinfo{Schreckgestalt IV}{wenn er möchte}\kreaturvorteile{}\trennlinie \kreaturattribute{CH 8, FF 6, GE 8, IN 20, KK 6, KL 24, KO 6, MU 12}\kreaturfertigkeiten{Einschüchtern 6, Menschenkenntnis 18, Sinnesschärfe 16, Wachsamkeit 16, Überreden 22, Willenskraft 10, Magiekunde 18}\trennlinie \kreaturinfo{AsP}{80}\kreaturinfo{Einfluss}{22 (Alle)}\kreaturinfo{Hellsicht}{22 (Alle)}\kreaturinfo{Magische Vorteile}{Effizientes Zaubern, Flinke Magie, Kontrolliertes Zaubern, Mühelose Magie, Vorbereitendes Zaubern}\trennlinie \kreaturinfo{Quelle}{\href{https://www.orkenspalter.de/filebase/index.php?file/2829-hilberts-bestiarium/}{Hilberts Bestarium}}}}
\newcommand{\kreaturdetailbaer}{\kreatur{Bär}{Raubtier der Wälder; großer Gegner}{gfx/kreaturen/tier}{\kreaturkampfwerte{11}{6}{6}{2}\trennlinie \kreaturwaffe{Prankenhieb}{1}{6}{13}{2W6+6}{Niederwerfen ({-}4), Doppelangriff}\kreaturwaffe{Biss}{0}{3}{14}{4W6+6}{Zerbrechlich}\kreaturkampfvorteile{Standfest, Sturmangriff}\trennlinie \kreaturattribute{GE 16, KK 40, KO 42, MU 16}\kreaturfertigkeiten{Wachsamkeit 16, Pirschen 8, Laufen 16, Zähigkeit 14}\trennlinie \kreaturinfo{Quelle}{\href{https://ilarisblog.wordpress.com/downloads/}{Ilaris Regeln}}}}
\newcommand{\kreaturdetailbalirhiadh}{\kreatur{BAL'IRHIADH}{Die rechte Hand des Schwarzen Mannes; ein mächtiger Diener Blakharaz‘, sehr großer Gegner}{gfx/kreaturen/daemon}{\kreaturkampfwerte{14}{20}{12}{12}\trennlinie \kreaturvorteile{Besessenheit von Tieren, Limbusreisender, Regeneration I, Schreckgestalt III, Zusätzliche Attacke IV}\trennlinie \kreaturwaffe{Klauen}{1}{12}{20}{5W6+2}{Niederwerfen ({-}8)}\kreaturwaffe{Riemen}{8}{16}{20}{3W6+2}{Betäubung (wie Ertränken) oder Umklammern ({-}4, 20) (bei jedem Angriff frei wählbar)}\trennlinie \kreaturfertigkeiten{Pirschen 20, Untertauchen 20, Sinnenschärfe 24, Wachsamkeit 24}\trennlinie \kreaturinfo{AsP}{128}\kreaturinfo{Dämonisch}{24 (Alpgestalt, Blick in die Gedanken, Dunkelheit, Halluzination, Skelettarius, Totes handle, Traumgestalt)}\trennlinie \kreaturinfo{Beschwörung}{Invocatio}\kreaturinfo{Dienste}{Opfer mit Alpträumen und Halluzinationen in den Wahnsinn treiben+4 (bis von meisterlichem Seelenheiler geheilt), Opfer mit kontrolliertem Tier suchen und töten+0 (1 Woche), Untote erheben und Opfer töten{-}4 (1 Woche)}\trennlinie \kreaturinfo{Info}{Der Dämon fährt in ein Tier ein. Er verliert: Klauen, Riemen, Koloss, GS, Besessenheit von Tieren, Limbusreisender, Regeneration I, Schreckensgestalt III, Zusätzliche Attacke IV  und erhält die Eigenschaften und Attacken des Tiers. Der Dämon kontrolliert das Tier vollständig, das Tier erhält die Vorteile Regeneration I, Schreckgestalt II, Tarnung und Zusätzliche Attacke II. Das Tier erleidet während der Besessenheit kein Erschöpfung durch körperliche Anstrengung.\newline%
}\trennlinie \kreaturinfo{Quelle}{\href{https://dsaforum.de/viewtopic.php?p=1887303\#p1887303}{Pandämonium}}}}
\newcommand{\kreaturdetailbalkhabul}{\kreatur{BALKHA'BUL}{Wächter der unermesslichen Schätze; dreigehörnter Diener Tasfarelels, großer Gegner}{gfx/kreaturen/daemon}{\kreaturkampfwerte{12}{14}{6}{8}\trennlinie \kreaturinfo{Aura}{Gestank, Zähigkeit (20) jede INI: phase, {-}2 auf körperliche Proben (kumulativ) für 1 Tag, {-}4 auf Proben in Rededuellen für 1W6 Wochen}\kreaturinfo{Unsichtbarkeit}{Unsichtbarkeit erlaubt es der Kreatur, sich beliebig oft und lange unsichtbar zu machen.\newline%
}\kreaturvorteile{Formlosigkeit, Immunität (profan), Resistenz I (geweiht)}\trennlinie \kreaturwaffe{Biss}{1}{4}{15}{4W6+2}{}\kreaturwaffe{Klauen}{1}{6}{12}{3W6+1}{Niederwerfen ({-}4)}\kreaturwaffe{Schwanz}{2}{4}{10}{2W6+1}{Niederwerfen, Flächenangriff (180° hinter dem Dämon)}\kreaturkampfvorteile{Zusätzliche Attacke I}\trennlinie \kreaturfertigkeiten{Untertauchen 20, Sinnenschärfe 24, Wachsamkeit 24}\trennlinie \kreaturinfo{AsP}{64}\kreaturinfo{Dämonisch}{18 (Auris Nasus, Leib des Erzes, Motoricus, Invocatio (Khidma’kha‘bulim))}\trennlinie \kreaturinfo{Beschwörung}{Invocatio}\kreaturinfo{Dienste}{Schätze bewachen+4 (1 Jahr), Andere Gegenstände bewachen+0 (1 Woche), Beschwörer bewachen{-}4 (1 Tag)}\trennlinie \kreaturinfo{Quelle}{\href{https://dsaforum.de/viewtopic.php?p=1887303\#p1887303}{Pandämonium}}}}
\newcommand{\kreaturdetailbaumdrache}{\kreatur{Baumdrache}{tierhafter Drache ohne magische Fähigkeiten}{gfx/kreaturen/tier}{\kreaturkampfwerte{4}{8}{1}{5}\trennlinie \kreaturinfo{Resistenz I}{Feuer}\kreaturvorteile{Flugfähig}\trennlinie \kreaturwaffe{Klauen}{0}{6}{10}{2W6+1}{}\kreaturwaffe{Feueratem}{}{}{}{}{Wer zum Ziel eines Angriffs wird, muss eine Zähikeits{-}Probe (16) ablegen, um nicht 1 Punkt Erschöpfung zu erleiden.}\kreaturkampfvorteile{Niederwerfen, Sturmangriff}\trennlinie \kreaturattribute{GE 20, KK 6, KO 6, MU 8}\kreaturfertigkeiten{Wachsamkeit 10, Zähigkeit 6}\trennlinie \kreaturinfo{Quelle}{\href{https://ilarisblog.wordpress.com/downloads/}{Ilaris Regeln}}}}
\newcommand{\kreaturdetailbergloewe}{\kreatur{Berglöwe}{Stolzer König der Khômwüste}{gfx/kreaturen/tier}{\kreaturkampfwerte{6}{2}{6}{4}\trennlinie \kreaturwaffe{Biss}{0}{1}{13}{3W6+2}{Zerbrechlich}\kreaturwaffe{Prankenhieb}{1}{6}{14}{2W6+2}{Wendig}\kreaturkampfvorteile{Doppelangriff, Niederwerfen, Standfest, Sturmangriff}\trennlinie \kreaturattribute{GE 16, KK 20, KO 16, MU 12}\kreaturfertigkeiten{Laufen 20, Pirschen 18, Wachsamkeit 14, Zähigkeit 12, Klettern 20}\trennlinie \kreaturinfo{Quelle}{\href{https://dsaforum.de/viewtopic.php?p=1887303\#p1887303}{Allgemeine Gegner}}}}
\newcommand{\kreaturdetailbhalevek}{\kreatur{Bha'levek}{Geißel des Geistes; eingehörnter Diener Amazeroths}{gfx/kreaturen/daemon}{\kreaturkampfwerte{1}{12}{6}{8}\trennlinie \kreaturinfo{Unsichtbarkeit}{Unsichtbarkeit erlaubt es der Kreatur, sich beliebig oft und lange unsichtbar zu machen.\newline%
}\kreaturvorteile{}\trennlinie \kreaturwaffe{Ausweichen}{0}{4}{}{}{}\trennlinie \kreaturinfo{AsP}{32}\kreaturinfo{Dämonisch}{12 (Traumgestalt)}\trennlinie \kreaturinfo{Beschwörung}{Invocatio}\kreaturinfo{Dienste}{Kopfschmerzen beim Opfer erzeugen+4 (1 Tag), Alpträume erzeugen+0 (1 Woche), Zauberer verwirren{-}4 (1 Stunde)}\trennlinie \kreaturinfo{Quelle}{\href{https://dsaforum.de/viewtopic.php?p=1887303\#p1887303}{Pandämonium}}}}
\newcommand{\kreaturdetailbhurkhesch}{\kreatur{Bhurkhesch}{Die Augenkröte; niederer Diener Belkelels, sehr kleiner Gegner}{gfx/kreaturen/daemon}{\kreaturkampfwerte{1}{8}{1}{{-}1}\trennlinie \kreaturvorteile{Ausweichen in den Limbus, Präsenz}\trennlinie \kreaturwaffe{Zunge}{1}{8}{8}{2W6+2}{Giftig (Bhurkhesch{-}Sekret, Stufe 20, Verzögerung 4 Aktionen, Wirkungsdauer 8 Stunden, Halluzinationen, Euphorie und aphrodisierende Wirkung, Proben {-}2 kumulativ))\newline%
}\trennlinie \kreaturfertigkeiten{Untertauchen 20, Wachsamkeit 10}\trennlinie \kreaturinfo{Beschwörung}{Invocatio}\kreaturinfo{Dienste}{5 Portionen Sekret zur Verfügung stellen+4, Orgie auslösen+0, Opfer in den Wahnsinn treiben{-}4}\trennlinie \kreaturinfo{Quelle}{\href{https://dsaforum.de/viewtopic.php?p=1887303\#p1887303}{Pandämonium}}}}
\newcommand{\kreaturdetailblutfischschwarm}{\kreatur{Blutfischschwarm}{gierige Fressmaschinen der Dschungelflüsse; großer Gegner}{gfx/kreaturen/tier}{\kreaturkampfwerte{3}{2}{0}{1}\trennlinie \kreaturinfo{Resistenz II}{Stichwaffen}\kreaturinfo{Verwundbarkeit IV}{Flächenschaden}\kreaturvorteile{Wasserwesen, Schmerzimmun II}\trennlinie \kreaturwaffe{Biss}{0}{4}{14}{2W6+3}{}\kreaturkampfvorteile{Zusätzliche Attacke II}\trennlinie \kreaturfertigkeiten{Pirschen 16, Schwimmen 8, Wachsamkeit 14}\trennlinie \kreaturinfo{Info}{MR pro einzelnes Tier, Fortbewegung nur schwimmend}\trennlinie \kreaturinfo{Quelle}{\href{https://ilarisblog.wordpress.com/downloads/}{Ilaris Regeln}}}}
\newcommand{\kreaturdetailbraggu}{\kreatur{Braggu}{Grün-violette Dämonenfratze; ein niederer Diener Thargunitoths}{gfx/kreaturen/daemon}{\kreaturkampfwerte{4}{6}{}{8}\trennlinie \kreaturvorteile{Flieger, Schreckgestalt II, Zusätzliche Attacke I}\trennlinie \kreaturwaffe{Stinkender Nebel}{2}{12}{6}{2W6+2}{Ertränken (Giftgas)}\trennlinie \kreaturinfo{AsP}{24}\kreaturinfo{Einfluss}{10 (Böser Blick, Horriphobus)}\trennlinie \kreaturinfo{Beschwörung}{Invocatio}\kreaturinfo{Dienste}{Gegner erschrecken+4 (1 Stunde,), Alpträume erzeugen+0 (1 Tag), Ort bewachen{-}4 (1 Woche)}\trennlinie \kreaturinfo{Quelle}{\href{https://dsaforum.de/viewtopic.php?p=1887303\#p1887303}{Pandämonium}}}}
\newcommand{\kreaturdetailbrukhaklah}{\kreatur{BRUKHA'KLAH}{Der Verderber des Eisens, die gepanzerte Made; niederer Diener Belzhorashs}{gfx/kreaturen/daemon}{\kreaturkampfwerte{10}{14}{2}{4}\trennlinie \kreaturinfo{Madensäure}{Waffen aus Metall gelten gegen den Dämon als zerbrechlich}\kreaturvorteile{Regeneration I, Zusätzliche Attacke I}\trennlinie \kreaturwaffe{Tentakel}{1}{10}{10}{2W6+2}{Nachbrennen (Säure)}\trennlinie \kreaturinfo{Beschwörung}{Invocatio}\kreaturinfo{Dienste}{Metallenen Gegenstand auflösen+4, ca. 30 Stein Metall auflösen+0, Kampf{-}4 (1 Minute)}\trennlinie \kreaturinfo{Quelle}{\href{https://dsaforum.de/viewtopic.php?p=1887303\#p1887303}{Pandämonium}}}}
\newcommand{\kreaturdetailcanilaaranshakalaraan}{\kreatur{Canilaaran, Shakalaraan}{Der brünstige Hund; niederer Diener Belkelels}{gfx/kreaturen/daemon}{\kreaturkampfwerte{5}{20}{8}{8}\trennlinie \kreaturvorteile{Rudel}\trennlinie \kreaturwaffe{Biss}{0}{4}{12}{3W6+0}{}\kreaturkampfvorteile{Niederwerfen, Sturmangriff}\trennlinie \kreaturfertigkeiten{Laufen 12, Wachsamkeit 8}\trennlinie \kreaturinfo{Beschwörung}{Invocatio}\kreaturinfo{Dienste}{Hetzjagd+4 (1 Tag), Daimoniden{-}Züchtung;+0, Person suchen und vernichten{-}4 (1 Tag)}\trennlinie \kreaturinfo{Quelle}{\href{https://dsaforum.de/viewtopic.php?p=1887303\#p1887303}{Pandämonium}}}}
\newcommand{\kreaturdetailchamuyan}{\kreatur{CHA'MUYAN}{Vielgestaltige Schleicherin in den Schatten; eine gehörnte Dämonin aus dem Gefolge Aphasmayras}{gfx/kreaturen/daemon}{\kreaturkampfwerte{12}{8}{11}{10}\trennlinie \kreaturinfo{Blitzschnell}{erleidet keine Passierschläge, wenn sie sich aus dem Nahkampf zurückzieht}\kreaturinfo{Resistenz I}{magisch, geweiht}\kreaturvorteile{Tarnung, Zusätzliche Attacke II}\trennlinie \kreaturwaffe{Prankenhieb}{1}{12}{16}{2W6+4}{Doppelangriff, Niederwerfen ({-}4)}\kreaturwaffe{Biss}{0}{2}{16}{5W6+2}{}\kreaturkampfvorteile{Sturmangriff}\trennlinie \kreaturfertigkeiten{Laufen 16, Pirschen 24, Wachsamkeit 18}\trennlinie \kreaturinfo{Beschwörung}{Invocatio}\kreaturinfo{Dienste}{Beratung als Sphinx+4, Kampf als Berglöwin+0 (1 Minute), Als Panther Person suchen und töten{-}4 (1 Tag)}\trennlinie \kreaturinfo{Quelle}{\href{https://dsaforum.de/viewtopic.php?p=1887303\#p1887303}{Pandämonium}}}}
\newcommand{\kreaturdetailchuchathabomek}{\kreatur{Chuchathabomek}{Das fliegende Hebewerk, der Vetter Karakils; dreigehörnter Diener Lolgramoths, sehr großer Gegner}{gfx/kreaturen/daemon}{\kreaturkampfwerte{10}{14}{2}{4}\trennlinie \kreaturvorteile{Flugfähig, Regeneration I}\trennlinie \kreaturwaffe{Klauenhieb}{1}{12}{12}{4W6+4}{}\kreaturwaffe{Klammer}{1}{2}{8}{}{Festes Umklammern (Das Ziel wird umschlungen. Gelingt ihm oder einem Helfer bis zur nächsten INI: phase keine KK{-}Probe (32), erleidet es 4W20 SP.)}\trennlinie \kreaturfertigkeiten{Fliegen 16, Wachsamkeit 8}\trennlinie \kreaturinfo{Beschwörung}{Invocatio}\kreaturinfo{Dienste}{Beschwörer transportieren+4 (1 Tag), Lasten transportieren+0 (1 Tag), Schwere Sordulsäpfel abwerfen{-}4 (1 Stunde)}\trennlinie \kreaturinfo{Quelle}{\href{https://dsaforum.de/viewtopic.php?p=1887303\#p1887303}{Pandämonium}}}}
\newcommand{\kreaturdetailcthllanogog}{\kreatur{Cthllanogog}{Der Alles Gebärende; niederer Diener Asfaloths, sehr großer Gegner}{gfx/kreaturen/daemon}{\kreaturkampfwerte{10}{16}{0}{2}\trennlinie \kreaturinfo{Immunität}{Verwandlung}\kreaturvorteile{Lebensraub, Präsenz, Regeneration II}\trennlinie \kreaturwaffe{Asthieb}{2}{6}{12}{2W6+2}{Mutation, Niederwerfen}\trennlinie \kreaturfertigkeiten{Wachsamkeit 12}\trennlinie \kreaturinfo{AsP}{32}\kreaturinfo{Dämonisch}{12 (Wand aus Dornen, Zorn der Elemente)}\trennlinie \kreaturinfo{Beschwörung}{Invocatio}\kreaturinfo{Dienste}{Körperteil des Beschwörers wiederherstellen+4, Körperteil eines „Opfers“ wiederherstellen+0, Mehrere Körperteile wachsen lassen{-}4}\trennlinie \kreaturinfo{Info}{Jeder Dienst des Cthllanogog beginnt damit, dass man ihm Blut desjenigen opfert, dessen Körperteil wachsen soll.}\trennlinie \kreaturinfo{Quelle}{\href{https://dsaforum.de/viewtopic.php?p=1887303\#p1887303}{Pandämonium}}}}
\newcommand{\kreaturdetaildarkha}{\kreatur{Darkha}{Der Zerfetzer, Teil der blutigen Horde; ein minderer Diener Belhalhars, kleiner Gegner}{gfx/kreaturen/daemon}{\kreaturkampfwerte{3}{6}{8}{6}\trennlinie \kreaturvorteile{Rudel}\trennlinie \kreaturwaffe{Krallen}{1}{6}{6}{1W6+0}{}\kreaturwaffe{Biss}{0}{6}{6}{2W6+0}{}\kreaturwaffe{Dolch}{0}{8}{8}{1W6+1}{}\kreaturwaffe{Speer}{2}{8}{8}{2W6{-}1}{Wendig}\kreaturkampfvorteile{Sturmangriff}\trennlinie \kreaturfertigkeiten{Laufen 8, Pirschen 6, Wachsamkeit 6}\trennlinie \kreaturinfo{AsP}{24}\kreaturinfo{Dämonisch}{8 (Corpofesso)}\trennlinie \kreaturinfo{Beschwörung}{Invocatio}\kreaturinfo{Dienste}{Entflohenen Gefangenen jagen und töten+4 (1 Stunde), Kampf+0 (1 Minute), Im Rudel über eine Barrikade schwärmen{-}4}\trennlinie \kreaturinfo{Quelle}{\href{https://dsaforum.de/viewtopic.php?p=1887303\#p1887303}{Pandämonium}}}}
\newcommand{\kreaturdetaildeolgolup}{\kreatur{Deolgolup}{Der stumme Wächter der augenlosen Pforten; ein niederer Diener Agrimoths}{gfx/kreaturen/daemon}{\kreaturkampfwerte{6}{12}{8}{8}\trennlinie \kreaturinfo{Unsichtbarkeit}{Unsichtbarkeit erlaubt es der Kreatur, sich beliebig oft und lange unsichtbar zu machen.\newline%
}\kreaturvorteile{Rudel, Rüstung beseelen}\trennlinie \kreaturwaffe{Ausweichen}{0}{4}{}{}{}\kreaturwaffe{Waffe}{0}{14}{14}{}{RW und TP je nach Waffe}\trennlinie \kreaturfertigkeiten{Wachsamkeit 12}\trennlinie \kreaturinfo{AsP}{24}\kreaturinfo{Illusion}{12 (Duplicatus)}\trennlinie \kreaturinfo{Beschwörung}{Invocatio}\kreaturinfo{Dienste}{Wache+4 (1 Woche), Schutz des Beschwörers+0 (1 Stunde), Kampf{-}4 (1 Minute)}\trennlinie \kreaturinfo{Info}{Der Deolgolup beseelt eine Rüstung und eine dazugehörige Waffe. Er verliert Rüstung besselen, Unsichtbarkeit, Ausweichen. Der Dämon kontrolliert die Rüstung vollständig, seine WS* erhöht sich um den RS der jeweiligen Rüstung. Die Rüstung erhält die Eigenschaft Explosion (4W6).\newline%
}\trennlinie \kreaturinfo{Quelle}{\href{https://dsaforum.de/viewtopic.php?p=1887303\#p1887303}{Pandämonium}}}}
\newcommand{\kreaturdetaildhormarhei}{\kreatur{Dhormarhei}{Der Nachtmahr; zweigehörnter Diener Thargunitoths, kleiner Gegner}{gfx/kreaturen/daemon}{\kreaturkampfwerte{6}{12}{6}{6}\trennlinie \kreaturinfo{Unsichtbarkeit}{Unsichtbarkeit erlaubt es der Kreatur, sich beliebig oft und lange unsichtbar zu machen.\newline%
}\kreaturvorteile{}\trennlinie \kreaturwaffe{Ausweichen}{0}{6}{}{}{}\trennlinie \kreaturinfo{AsP}{32}\kreaturinfo{Dämonisch}{16 (Blitz,, Horriphobus, Traumgestalt)}\trennlinie \kreaturinfo{Beschwörung}{Invocatio}\kreaturinfo{Dienste}{Alpträume erzeugen+4 (1 Tag), Opfer suchen und in Alptraumwelt gefangenhalten+0 (1 Tag)}\trennlinie \kreaturinfo{Quelle}{\href{https://dsaforum.de/viewtopic.php?p=1887303\#p1887303}{Pandämonium}}}}
\newcommand{\kreaturdetaildrachenfliege}{\kreatur{Drachenfliege}{klein, Musca draconigena flagranta}{gfx/kreaturen/elementar}{\kreaturkampfwerte{2}{17}{9}{4}\trennlinie \kreaturwaffe{Biss}{1}{5}{13}{1W6+3}{Nachbrennen}\trennlinie \kreaturattribute{CH 2, FF 2, GE 6, IN 2, KK 2, KL 0, KO 2, MU 10}\kreaturfertigkeiten{Akrobatik 16, Willenskraft 5, Einschüchtern 10, Sinnesschärfe 6}\trennlinie \kreaturinfo{Info}{Elementübergang: Die Flügel der Drachenfliege verwandeln sich in Feuer, wenn sie bedroht wird. Dies lockt oft Mindergeister an.}\trennlinie \kreaturinfo{Quelle}{\href{https://www.orkenspalter.de/filebase/index.php?file/2829-hilberts-bestiarium/}{Hilberts Bestarium}}}}
\newcommand{\kreaturdetaildreigehoerntergurgulum}{\kreatur{Dreigehörnter Gurgulum}{Der Dunkelschlund, ein dreigehörnter Diener des Blakharaz, sehr kleiner Gegner}{gfx/kreaturen/daemon}{\kreaturkampfwerte{3}{10}{1}{4}\trennlinie \kreaturinfo{Unsichtbarkeit}{Unsichtbarkeit erlaubt es der Kreatur, sich beliebig oft und lange unsichtbar zu machen.\newline%
}\kreaturinfo{Stärkung des Wirts}{Proben des Wirts auf körperliche Attribute sind um +4, Fertigkeitsproben mit körperlichen Attributen um +2 erleichtert, WS des Wirts +1, jedoch erleidet der Wirt jeden Tag 1W6+2 SP\newline%
}\kreaturvorteile{Kritische Konsistenz, Opfer übernehmen}\trennlinie \kreaturwaffe{Ausweichen}{0}{4}{}{}{}\kreaturwaffe{Würgen}{0}{2}{12}{1W6+0}{Umklammern ({-}2, 12)}\trennlinie \kreaturinfo{AsP}{24}\kreaturinfo{Einfluss}{10 (Große Gier, Imperavi)}\trennlinie \kreaturinfo{Beschwörung}{Invocatio}\kreaturinfo{Dienste}{Kontrolle über Wirt übernehmen+0 (1 Monat), Kontrolle über Wirt übernehmen+4 (6 Monate)}\trennlinie \kreaturinfo{Info}{Der Durgulum legt sich um den Hals des Opfers und verliert alle Angriffe mit *. Er kann kurzfristig die Kontrolle über das Opfer übernehmen oder dieses zur Strafe würgen. Die Einnahme von Rauschmitteln lässt den Durgulum für einige Zeit schlafen.\newline%
}\trennlinie \kreaturinfo{Quelle}{\href{https://dsaforum.de/viewtopic.php?p=1887303\#p1887303}{Pandämonium}}}}
\newcommand{\kreaturdetaileisfee}{\kreatur{Eisfee}{nützlicher kalter Geist des hohen Nordens}{gfx/kreaturen/geist}{\kreaturkampfwerte{3}{10}{4}{4}\trennlinie \kreaturinfo{Herrin der gefesselten Seelen}{Eisfeen können gefesselte Seelen an sich binden}\kreaturinfo{Kuss der Eisfee}{Intervall 1 INI{-}Ph. 1 Wunde, freiwilliges Opfer; siehe Bannbaladin; Vergleichende Probe Willenksraft gegen Betören}\kreaturvorteile{}\trennlinie \kreaturwaffe{Waffenlos}{0}{2}{14}{3W6+0}{Rüstungsbrechend}\trennlinie \kreaturattribute{CH 20, FF 2, GE 8, IN 8, KK 6, KL 4, KO 2, MU 18}\kreaturfertigkeiten{Einschüchtern 5, Wachsamkeit 10, Betören 14, Überreden 14, Willenskraft 13}\trennlinie \kreaturinfo{AsP}{32}\kreaturinfo{Einfluss}{14 (Bannbaladin, Horriphobus)}\trennlinie \kreaturinfo{Info}{Varianten:  Starke Eisfee (6/6 WS, Zusätzliche Zauber: Erinnerung verlass dich!, Herzschlag Ruhe!)\newline%
Mächtige Eisfee (6/6 WS, Eigenschaften +2, Fertigkeiten +4, 64 AsP, Einfluss 18. Zusätzliche Zauber: Erinnerung verlass dich!, Herzschlag Ruhe!, Alpgestalt)\newline%
}\trennlinie \kreaturinfo{Quelle}{\href{https://www.orkenspalter.de/filebase/index.php?file/2829-hilberts-bestiarium/}{Hilberts Bestarium}}}}
\newcommand{\kreaturdetailelefant}{\kreatur{Elefant}{gutmütiger Pflanzenfresser; sehr großer Gegner}{gfx/kreaturen/tier}{\kreaturkampfwerte{15}{4}{9}{1}\trennlinie \kreaturwaffe{Stoßzähne}{1}{6}{8}{5W6+4}{Niederwerfen}\kreaturwaffe{Trampeln}{1}{4}{6}{3W20+0}{Flächenangriff (1 Schritt Umkreis), Niederwerfen ({-}8)}\kreaturkampfvorteile{Standfest, Überrennen, Unaufhaltsam, Zerstörerisch I, Zerstörerisch II}\trennlinie \kreaturattribute{KK 120, KO 140, MU 10}\kreaturfertigkeiten{Laufen 14, Wachsamkeit 12, Zähigkeit 14}\trennlinie \kreaturinfo{Quelle}{\href{https://ilarisblog.wordpress.com/downloads/}{Ilaris Regeln}}}}
\newcommand{\kreaturdetailelefantruestung}{\kreatur{Elefant mit Rüstung}{gutmütiger Pflanzenfresser; sehr großer Gegner}{gfx/kreaturen/tier}{\kreaturkampfwerte{15}{4}{{-}1}{1}\trennlinie \kreaturwaffe{Stoßzähne}{1}{6}{8}{5W6+4}{Niederwerfen}\kreaturwaffe{Trampeln}{1}{4}{6}{3W20+0}{Flächenangriff (1 Schritt Umkreis), Niederwerfen ({-}8)}\kreaturkampfvorteile{Standfest, Überrennen, Unaufhaltsam, Zerstörerisch I, Zerstörerisch II}\trennlinie \kreaturattribute{KK 120, KO 140, MU 10}\kreaturfertigkeiten{Laufen 14, Wachsamkeit 12, Zähigkeit 14}\trennlinie \kreaturinfo{Quelle}{\href{https://ilarisblog.wordpress.com/downloads/}{Ilaris Regeln}}}}
\newcommand{\kreaturdetailelymelusinias}{\kreatur{Elymelusinias}{Der Bote der Tiefe; dreigehörnter Diener Charyptoroths}{gfx/kreaturen/daemon}{\kreaturkampfwerte{8}{12}{6}{8}\trennlinie \kreaturinfo{Tarnung}{im Wasser}\kreaturvorteile{Amphibisches Wesen, Regeneration I}\trennlinie \kreaturwaffe{Stab}{2}{14}{14}{2W6+2}{}\kreaturwaffe{Wasserstrahl}{8}{}{}{2W6+4}{Ertränken}\trennlinie \kreaturfertigkeiten{Geographie 16, Pirschen 16, Menschenkenntnis 14, Schwimmen 20, Überreden 16, Überleben (Meer) 20, Wachsamkeit 12}\trennlinie \kreaturinfo{AsP}{128}\kreaturinfo{Dämonisch}{18 (alle Wasserzauber und ähnliche Effekte, Brenne!, Kulminatio, Skelettarius, Totes Handle, Weiches Erstarre!)}\trennlinie \kreaturinfo{Beschwörung}{Invocatio}\kreaturinfo{Dienste}{Preisgeben von Informationen (lohnende Beute für Piraten, gute Fischgründe oder Meeresströmungen)+4, Gegenstand auf dem Meeresgrund oder einem Schiff suchen und stehlen+0 (1 Tag), Wasserleichen erheben{-}4}\trennlinie \kreaturinfo{Quelle}{\href{https://dsaforum.de/viewtopic.php?p=1887303\#p1887303}{Pandämonium}}}}
\newcommand{\kreaturdetailenzomaincoon}{\kreatur{Enzo (Main Coon)}{spionierender Stubentieger}{gfx/kreaturen/tier}{\kreaturkampfwerte{3}{6}{6}{6}\trennlinie \kreaturinfo{Angepasst II (Dunkelheit)}{Angepasst I/II (Umgebung) Durch deine Spezies oder langjährige Erfahrung hast du dich an eine bestimmte Umgebung oder Umweltbedingung gewöhnt. Abzüge durch diese Umgebung (Beispiele auf S. 38), insbesondere im Kampf, sinken für dich um eine/zwei Stufen. Die Kosten für Angepasst legt der Spielleiter fest, wobei er sich an der Häufigkeit der Umgebung orientieren sollte. Zu allgemein gefasste Umgebungen wie „unsicherer Untergrund“ sollte er nicht zulassen. Beispiele für Angepasst sind: • Dunkelheit: verringert Abzüge durch schlechte Lichtverhältnisse (40 EP pro Stufe) • Schnee: verringert Abzüge durch schneebedeckten oder eisigen Untergrund (20 EP pro Stufe) • Wasser: verringert Abzüge durch knie{-} oder hüfttiefes Wasser und unter Wasser (20 EP pro Stufe) • Wald: verringert Abzüge durch Wurzeln, Gestrüpp und dichtes Unterholz (40 EP pro Stufe) Voraussetzungen: keine/Angepasst I Nachkauf: häufig/selten\newline%
}\kreaturvorteile{}\trennlinie \kreaturwaffe{Krallen}{0}{14}{14}{1W6+2}{None}\trennlinie \kreaturattribute{KK {-}16, KL {-}6, MU 0}\kreaturfertigkeiten{Klettern 8, Pirschen 20, Wachsamkeit 12, Zähigkeit {-}2}\trennlinie \kreaturinfo{Quelle}{\href{https://www.orkenspalter.de/filebase/index.php?file/2829-hilberts-bestiarium/}{Hilberts Bestarium}}}}
\newcommand{\kreaturdetailetemasoroph}{\kreatur{Etema{-}Soroph}{Jenes-das-nimmermehr-freigibt; ein minderer Diener Tasfarelels, sehr kleiner Gegner}{gfx/kreaturen/daemon}{\kreaturkampfwerte{1}{6}{0}{0}\trennlinie \kreaturwaffe{Biss}{0}{}{14}{1W6+4}{}\trennlinie \kreaturfertigkeiten{Untertauchen 14, Wachsamkeit 12}\trennlinie \kreaturinfo{AsP}{24}\kreaturinfo{Einfluss}{8 (Große Gier, Harmlose Gestalt)}\trennlinie \kreaturinfo{Beschwörung}{Invocatio}\kreaturinfo{Dienste}{Geld des Beschwörers verwahren und Dieben die Finger abbeißen+4 (1 Monat), Geld eines mit dem Geldbeutel beschenkten Konkurrenten verschlingen+0 (1 Woche), Gier in einem Opfer auslösen+4 (1 Tag)}\trennlinie \kreaturinfo{Quelle}{\href{https://dsaforum.de/viewtopic.php?p=1887303\#p1887303}{Pandämonium}}}}
\newcommand{\kreaturdetaileugalp}{\kreatur{Eugalp}{Der Plagenbringer; eine viergehörnte Wesenheit, vermutlich einstmals aus dem Gefolge Mishkharas}{gfx/kreaturen/daemon}{\kreaturkampfwerte{14}{20}{11}{10}\trennlinie \kreaturvorteile{Regeneration I, Schreckgestalt II}\trennlinie \kreaturwaffe{Krallen}{1}{10}{12}{2W6+4}{Infektion (Das Ziel wird mit der Dämonenfäule angesteckt.)}\trennlinie \kreaturinfo{AsP}{64}\kreaturinfo{Dämonisch}{16 (Fluch der Pestilenz)}\trennlinie \kreaturinfo{Beschwörung}{Invocatio}\kreaturinfo{Dienste}{Nahes Ziel mit Dämonenfäule infizieren+4, Lebensmittel verseuchen+0, Besitzer eines vorhandenen Körperteils suchen und mit Dämonenfäule infizieren{-}4}\trennlinie \kreaturinfo{Quelle}{\href{https://dsaforum.de/viewtopic.php?p=1887303\#p1887303}{Pandämonium}}}}
\newcommand{\kreaturdetailfargyraff}{\kreatur{Fargy'raff}{Der freudlose Quell der Habgier; ein sechsgehörnter Diener Tasfarelels}{gfx/kreaturen/daemon}{\kreaturkampfwerte{10}{18}{6}{12}\trennlinie \kreaturwaffe{Faust}{0}{14}{14}{2W6+2}{}\trennlinie \kreaturfertigkeiten{Gebräuche 20, Menschenkenntnis 24, Überreden 24}\kreaturfertigkeiten{Händler 3}\trennlinie \kreaturinfo{AsP}{64}\kreaturinfo{Dämonisch}{20 (Bannbaladin, Große Gier, Seidenzunge)}\trennlinie \kreaturinfo{Beschwörung}{Invocatio}\kreaturinfo{Dienste}{Dem Beschwörer einen seltenen Gegenstand verkaufen+4, Schatzkammer des Beschwörers füllen+0, Jemanden zu einem Pakt verführen{-}4}\trennlinie \kreaturinfo{Quelle}{\href{https://dsaforum.de/viewtopic.php?p=1887303\#p1887303}{Pandämonium}}}}
\newcommand{\kreaturdetailfernhaendlerpaktierer}{\kreatur{Fernhändler und Paktierer}{ein gewitzter Händler und Paktierer der Charyptoroth}{gfx/kreaturen/humanoid}{\kreaturkampfwerte{5}{9}{4}{5}\trennlinie \kreaturvorteile{Gefahreninstinkt, Kreis der Verdammnis I, Willensstark I}\trennlinie \kreaturwaffe{Dolch}{0}{11}{11}{1W6+1}{}\kreaturkampfvorteile{Durchatmen}\trennlinie \kreaturattribute{CH 14, FF 88, GE 6, IN 12, KK 4, KL 8, KO 8, MU 12}\kreaturfertigkeiten{Andergast \& Nostria (Gebräuche) 11, Autorität 8, Betören 14, Überreden 14, Untertauchen 6, Verschlagenheit 7, Willenskraft 11, Wahrnehmung 11, Zähigkeit 8}\kreaturfertigkeiten{Händler 3, Seefahrer 2, Verwalter 2, Weinkenner 1}\kreaturinfo{Profane Vorteile}{Eindrucksvoll I, Eindrucksvoll II, Vorausschauend I, Vorausschauend II}\trennlinie \kreaturinfo{GuP}{25}\kreaturinfo{Dämonische Hilfe Charyptoroth}{10 (Amrychots Tanz, Dämonische Stärkung (Derekunde; KO; MU), Meister der Maritimen, Mholurenhaut, Wasserleiche erheben)}\kreaturinfo{Paktvorteile}{Erzdämonische Tradition I}\trennlinie \kreaturinfo{Info}{Als angesehener Bürger Nostrias ist der Fernhändler eine Quelle für Gerüchte und Nachrichten aus fernen Landen und führt das eine oder andere exotische Stück in seinem Sortiment. Um Konkurrenten zu schaden oder verbotene Waren zu schmuggeln, greift er gerne auch auf phexische Spezialisten zurück. Varianten: In einer ausweglosen Situation wird der Fernhändler einen Kreis der Verdammnis aufsteigen. Er füllt damit seine GuP auf, steigert den PW in Dämonische Hilfe auf 14 und erhält die Anrufung Ertränken, mit der er sich gegen Angreifer zur Wehr setzt.}\trennlinie \kreaturinfo{Quelle}{\href{https://ilarisblog.wordpress.com/downloads/}{Ilaris Regeln}}}}
\newcommand{\kreaturdetailfeuerdschinn}{\kreatur{Feuerdschinn}{Gebieter elementarer Kraft}{gfx/kreaturen/elementar}{\kreaturkampfwerte{10}{8}{8}{6}\trennlinie \kreaturwaffe{Flammenhand}{1}{7}{15}{3W6+6}{Nachbrennen}\kreaturkampfvorteile{Niederwerfen, Sturmangriff, Zusätzliche Attacke I}\trennlinie \kreaturattribute{CH 10, FF 6, GE 14, IN 12, KK 26, KL 8, KO 26, MU 14}\kreaturfertigkeiten{Akrobatik 14, Willenskraft 7, Zähigkeit 14, Pirschen 14, Einschüchtern 11, Sinnesschärfe 14}\trennlinie \kreaturinfo{AsP}{64}\kreaturinfo{Feuer}{12 (alle)}\trennlinie \kreaturinfo{Quelle}{\href{https://www.orkenspalter.de/filebase/index.php?file/2829-hilberts-bestiarium/}{Hilberts Bestarium}}}}
\newcommand{\kreaturdetailfeuergeist}{\kreatur{Feuergeist}{klein, neugierig und reizbar}{gfx/kreaturen/elementar}{\kreaturkampfwerte{2}{4}{5}{1}\trennlinie \kreaturvorteile{Regeneration I}\trennlinie \kreaturwaffe{Flammenhand}{1}{5}{13}{2W6+2}{Nachbrennen}\trennlinie \kreaturattribute{CH 2, FF 2, GE 6, IN 2, KK 2, KL 0, KO 2, MU 10}\kreaturfertigkeiten{Akrobatik 4, Willenskraft 5, Zähigkeit 5, Einschüchtern 5, Sinnesschärfe 5}\trennlinie \kreaturinfo{Quelle}{\href{https://www.orkenspalter.de/filebase/index.php?file/2829-hilberts-bestiarium/}{Hilberts Bestarium}}}}
\newcommand{\kreaturdetailfeuermaehre}{\kreatur{Feuermähre}{großes Feuerross}{gfx/kreaturen/elementar}{\kreaturkampfwerte{5}{17}{10}{4}\trennlinie \kreaturinfo{Jähzorn}{Willkürlich greift die Feuermähre an. IN(20 ,I) sonst {-}8 auf die erste Parade. Ebenso willkürlich endet der Angriff}\kreaturvorteile{}\trennlinie \kreaturwaffe{Tritt}{2}{7}{16}{3W6+4}{Nachbrennen, Niederwerfen}\trennlinie \kreaturattribute{CH 4, FF 2, GE 8, IN 2, KK 16, KL 2, KO 16, MU 6}\kreaturfertigkeiten{Willenskraft 4, Einschüchtern 6, Wachsamkeit 4}\trennlinie \kreaturinfo{Behütend}{Der Reiter ist immun gegen Feuerschaden, nicht aber gegen den Jähzorn der Feuermähre.}\trennlinie \kreaturinfo{Quelle}{\href{https://www.orkenspalter.de/filebase/index.php?file/2829-hilberts-bestiarium/}{Hilberts Bestarium}}}}
\newcommand{\kreaturdetailfleckenhai}{\kreatur{Fleckenhai}{Räuber der Meere}{gfx/kreaturen/tier}{\kreaturkampfwerte{7}{4}{8}{4}\trennlinie \kreaturvorteile{Wasserwesen}\trennlinie \kreaturwaffe{Biss}{0}{4}{10}{1W6+3}{Zerbrechlich, Gift (Stufe 24, Verzögerung 0, Intervall 0, 1 Wunde, WD 1W6 Tage, Lähmung: körperliche Proben {-}2, kumulativ)}\kreaturkampfvorteile{Sturmangriff}\trennlinie \kreaturattribute{GE 14, KK 20, KO 20, MU 12}\kreaturfertigkeiten{Pirschen 8, Schwimmen 16, Wachsamkeit 12, Zähigkeit 14}\trennlinie \kreaturinfo{Quelle}{\href{https://dsaforum.de/viewtopic.php?p=1887303\#p1887303}{Allgemeine Gegner}}}}
\newcommand{\kreaturdetailflugechse}{\kreatur{Flugechse}{Großer Jäger der Lüfte und Reittier von Achaz}{gfx/kreaturen/tier}{\kreaturkampfwerte{6}{4}{2}{5}\trennlinie \kreaturvorteile{Flugfähig, Kältestarre}\trennlinie \kreaturwaffe{Klauen}{0}{4}{8}{2W6+2}{}\kreaturwaffe{Biss}{0}{6}{8}{1W6+4}{Zerbrechlich}\kreaturkampfvorteile{Niederwerfen, Sturmangriff}\trennlinie \kreaturattribute{GE 6, KK 16, KO 12}\kreaturfertigkeiten{Wachsamkeit 10, Fliegen 10}\trennlinie \kreaturinfo{Quelle}{\href{https://dsaforum.de/viewtopic.php?f=180&p=1738549\#p1738549}{Bestarium+}}}}
\newcommand{\kreaturdetailfrazzaroth}{\kreatur{Frazzaroth}{Der kindliche Herr der Dunklen Pforten; niederer Diener Lolgramoths, kleiner Gegner}{gfx/kreaturen/daemon}{\kreaturkampfwerte{3}{10}{10}{10}\trennlinie \kreaturvorteile{Ausweichen in den Limbus, Limbusreisender}\trennlinie \kreaturwaffe{Hieb}{1}{12}{6}{1W6+2}{}\trennlinie \kreaturfertigkeiten{Laufen 12, Pirschen 10, Wachsamkeit 12}\trennlinie \kreaturinfo{AsP}{32}\kreaturinfo{Dämonisch}{10 (Axxeleratus, Transversalis)}\trennlinie \kreaturinfo{Beschwörung}{Invocatio}\kreaturinfo{Dienste}{Transport des Beschwörers durch den Limbus an einen bereits besuchten Ort+4, Transport des Beschwörers durch den Limbus an einen bekannten, aber noch nicht besuchten Ort+0}\trennlinie \kreaturinfo{Info}{Sollte der Frazzaroth jemanden durch den Limbus transportieren, so muss der Mitreisende eine Willenskraftprobe (16) bestehen, um keinen Furcht{-} Effekt der Stufe I erleiden. Zusätzlich kann der Frazzaroth in solch einem Fall seinen Mitreisenden an einen anderen als den gewünschten Ort bringen oder sogar im Limbus stranden lassen.\newline%
}\trennlinie \kreaturinfo{Quelle}{\href{https://dsaforum.de/viewtopic.php?p=1887303\#p1887303}{Pandämonium}}}}
\newcommand{\kreaturdetailgargyl}{\kreatur{Gargyl}{hinterhältiger Jäger aus Stein}{gfx/kreaturen/mythen}{\kreaturkampfwerte{8}{12}{6}{4}\trennlinie \kreaturinfo{Immunität}{Eigenschaften, Hellsicht, Verwandlung}\kreaturinfo{Angepasst II (Dunkelheit)}{Angepasst I/II (Umgebung) Durch deine Spezies oder langjährige Erfahrung hast du dich an eine bestimmte Umgebung oder Umweltbedingung gewöhnt. Abzüge durch diese Umgebung (Beispiele auf S. 38), insbesondere im Kampf, sinken für dich um eine/zwei Stufen. Die Kosten für Angepasst legt der Spielleiter fest, wobei er sich an der Häufigkeit der Umgebung orientieren sollte. Zu allgemein gefasste Umgebungen wie „unsicherer Untergrund“ sollte er nicht zulassen. Beispiele für Angepasst sind: • Dunkelheit: verringert Abzüge durch schlechte Lichtverhältnisse (40 EP pro Stufe) • Schnee: verringert Abzüge durch schneebedeckten oder eisigen Untergrund (20 EP pro Stufe) • Wasser: verringert Abzüge durch knie{-} oder hüfttiefes Wasser und unter Wasser (20 EP pro Stufe) • Wald: verringert Abzüge durch Wurzeln, Gestrüpp und dichtes Unterholz (40 EP pro Stufe) Voraussetzungen: keine/Angepasst I Nachkauf: häufig/selten\newline%
}\kreaturinfo{Tarnung}{in der Stadt}\kreaturvorteile{Flugfähig, Geisterpanzer}\trennlinie \kreaturwaffe{Klauen}{1}{6}{12}{2W6+2}{}\kreaturkampfvorteile{Doppelangriff, Niederwerfen, Standfest, Sturmangriff}\trennlinie \kreaturattribute{GE 6, KK 24, KO 28, MU 12}\kreaturfertigkeiten{Untertauchen 18, Wachsamkeit 10, Zähigkeit 16}\trennlinie \kreaturinfo{Info}{Anfällig gegen Einflusszauber (gegen Zauber der Fertigkeit Einfluss gilt eine MR von 0)}\trennlinie \kreaturinfo{Quelle}{\href{https://dsaforum.de/viewtopic.php?p=1887303\#p1887303}{Allgemeine Gegner}}}}
\newcommand{\kreaturdetailgefesselteseele}{\kreatur{Gefesselte Seele}{der schwache Geist eines Menschen}{gfx/kreaturen/geist}{\kreaturkampfwerte{2}{10}{8}{6}\trennlinie \kreaturvorteile{Vorteile bis 80 EP}\trennlinie \kreaturwaffe{Waffenlos}{0}{6}{6}{1W6+0}{Rüstungsbrechend}\kreaturkampfvorteile{Nach Verstorbenem}\trennlinie \kreaturinfo{AsP}{30}\kreaturinfo{Einfluss}{12 (Horriphobus)}\trennlinie \kreaturinfo{Info}{Attribute:  2x4, 4x6, 2x8,  Talente:  4x6, 1x10\newline%
Varianten:  \newline%
Nützliche Seele (Vorteile bis 160 EP, ein Attribut auf 12 statt auf 4, eine zusätzliche Fertigkeit auf 10 und 14) \newline%
Starker Nekromentor (nach Wissen strebender Lehrgeist, der durchaus erfahren in seinem Wissensgebiet ist. Von 17:00 bis 5:00 morgens ist er bereit sein Wissen weiterzugeben: 2x14 Wissensfertigkeiten. KL 20 IN: 18)\newline%
Mächtige Seele (auch andere Wesen als Mensch möglich, Werte abweichend)\newline%
Mächtige Entfesselte Seele (Eigenschaften, Fertigkeiten wie Erhobener, Zauber wie Erhobener, persönliche Waffe/Pferd/Gegenstände des Erhobenenen erscheinen geisterhaft, grobe Erinnerungen ans Leben)\newline%
}\trennlinie \kreaturinfo{Quelle}{\href{https://www.orkenspalter.de/filebase/index.php?file/2829-hilberts-bestiarium/}{Hilberts Bestarium}}}}
\newcommand{\kreaturdetailgeisterdrache}{\kreatur{Geisterdrache}{uralter mächtiger sehr großer untoter Ratgeber}{gfx/kreaturen/geist}{\kreaturkampfwerte{9}{16}{8}{4}\trennlinie \kreaturinfo{Aura}{Wahnsinn, Willenskraft (28) jede Initiativephase, 1 Erschöpfung}\kreaturinfo{Aura}{Disruptivo, Alle Zauber anderer Wesen sind in einer Reichweite von 32 Schritt um 8 erschwert}\kreaturinfo{Immunität}{magisch}\kreaturvorteile{Schreckgestalt III}\trennlinie \kreaturwaffe{Biss}{0}{8}{14}{3W6+6}{Rüstungsbrechend}\kreaturwaffe{Pranke}{1}{8}{14}{4W6+2}{Rüstungsbrechend, Niederwerfen ({-}4) (Willenskraft als Gegenprobe statt KK)}\kreaturwaffe{Schwanzhieb}{2}{8}{14}{3W6+0}{Rüstungsbrechend, Flächenangriff (90° hinter dem Drachen), Niederwerfen ({-}8) (Willenskraft als Gegenprobe statt KK)}\kreaturkampfvorteile{Hammerschlag, Kraftvoller Kampf III, Standfest, Sturmangriff, Unaufhaltsam, Zusätzliche Attacke I}\trennlinie \kreaturattribute{CH 6, FF 1, GE 12, IN 8, KK 24, KL 8, KO 16, MU 18}\kreaturfertigkeiten{Einschüchtern 14, Menschenkenntnis 8, Sinnesschärfe 6, Wachsamkeit 6, Überreden 4, Willenskraft 10, Magiekunde 18, Derekunde 18, Mythenkunde 18}\trennlinie \kreaturinfo{AsP}{150}\kreaturinfo{Kraft}{18 (Alle)}\kreaturinfo{Illusion}{16 (Alle)}\kreaturinfo{Magische Vorteile}{Drachenmagie(Reichweite und Wirkungsdauer jedes Zaubers ist vervielfacht)}\trennlinie \kreaturinfo{Quelle}{\href{https://www.orkenspalter.de/filebase/index.php?file/2829-hilberts-bestiarium/}{Hilberts Bestarium}}}}
\newcommand{\kreaturdetailgeisterschwarm}{\kreatur{Geisterschwarm}{der schwache Geist}{gfx/kreaturen/geist}{\kreaturkampfwerte{2}{6}{5}{10}\trennlinie \kreaturinfo{Astralsauger}{Richtet der Geisterschwarm Wunden bei einem Wesen an, das über AsP verfügt, so verliert dieses Wesen WS Anzahl der Wunden AsP.}\kreaturinfo{Resistenz}{Stichwaffen}\kreaturinfo{Verwundbarkeit IV}{Flächenschaden}\kreaturvorteile{Schmerzimmun II}\trennlinie \kreaturwaffe{Waffenlos}{0}{6}{10}{1W6+0}{Rüstungsbrechend}\kreaturkampfvorteile{Zusätzliche Attacke I}\trennlinie \kreaturattribute{CH 4, FF 0, GE 14, IN 20, KK 2, KL 2, KO 4, MU 10}\kreaturfertigkeiten{Einschüchtern 5, Pirschen 7, Sinnesschärfe 8, Willenskraft 1}\trennlinie \kreaturinfo{Info}{Varianten:  Starker Geisterschwarm (Koloss II, TP +2, Fertigkeiten +2, Zusätzliche Attacke II)}\trennlinie \kreaturinfo{Quelle}{\href{https://www.orkenspalter.de/filebase/index.php?file/2829-hilberts-bestiarium/}{Hilberts Bestarium}}}}
\newcommand{\kreaturdetailghul}{\kreatur{Ghul}{ansteckender und gieriger Leichenfresser}{gfx/kreaturen/tier}{\kreaturkampfwerte{4}{9}{6}{6}\trennlinie \kreaturinfo{Empfindlichkeit II}{Sonnenlicht}\kreaturinfo{Immunität}{Einfluss}\kreaturvorteile{Lichtscheu, Schreckgestalt I}\trennlinie \kreaturwaffe{Unbewaffnet}{1}{6}{6}{1W6+0}{Stumpf, Zerbrechlich, Hochansteckend (Jede erlittene Wunde erhöht das Chance einer Ansteckung mit Wundbrand oder einer anderen Krankheit um 50\%. Eine Probe auf Gifte und Krankheiten (24) eliminiert das Risiko.)}\kreaturwaffe{Zähne}{0}{6}{6}{2W6+0}{Zerbrechlich, Ghulgift (Stufe 16, Verzögerung 0 Aktionen, Wirkungs{-} dauer 8 Stunden; Proben auf körperliche   attribute und Fertigkeiten sind kumulativ um {-}4 erschwert. Fällt bei der KO{-}Probe eine 1{-}4, verwandelt sich der Vergiftete binnen 1W6 Tagen selbst in einen Ghul, wenn keine Probe auf Gifte und Krankheiten (28) gelingt.)}\trennlinie \kreaturattribute{GE 8, KK 8, KL 2, KO 8, MU 8}\kreaturfertigkeiten{Pirschen 8, Wachsamkeit 6}\trennlinie \kreaturinfo{Quelle}{\href{https://ilarisblog.wordpress.com/downloads/}{Ilaris Regeln}}}}
\newcommand{\kreaturdetailghulzwei}{\kreatur{Ghul}{ansteckender und gieriger Leichenfresser}{gfx/kreaturen/tier}{\kreaturkampfwerte{4}{9}{6}{6}\trennlinie \kreaturinfo{Empfindlichkeit II}{Sonnenlicht}\kreaturinfo{Immunität}{Einfluss}\kreaturinfo{Angepasst II (Dunkelheit)}{Angepasst I/II (Umgebung) Durch deine Spezies oder langjährige Erfahrung hast du dich an eine bestimmte Umgebung oder Umweltbedingung gewöhnt. Abzüge durch diese Umgebung (Beispiele auf S. 38), insbesondere im Kampf, sinken für dich um eine/zwei Stufen. Die Kosten für Angepasst legt der Spielleiter fest, wobei er sich an der Häufigkeit der Umgebung orientieren sollte. Zu allgemein gefasste Umgebungen wie „unsicherer Untergrund“ sollte er nicht zulassen. Beispiele für Angepasst sind: • Dunkelheit: verringert Abzüge durch schlechte Lichtverhältnisse (40 EP pro Stufe) • Schnee: verringert Abzüge durch schneebedeckten oder eisigen Untergrund (20 EP pro Stufe) • Wasser: verringert Abzüge durch knie{-} oder hüfttiefes Wasser und unter Wasser (20 EP pro Stufe) • Wald: verringert Abzüge durch Wurzeln, Gestrüpp und dichtes Unterholz (40 EP pro Stufe) Voraussetzungen: keine/Angepasst I Nachkauf: häufig/selten\newline%
}\kreaturvorteile{Lichtscheu, Schreckgestalt I}\trennlinie \kreaturwaffe{Unbewaffnet}{1}{6}{6}{1W6+0}{Stumpf, Zerbrechlich, Hochansteckend}\kreaturwaffe{Zähne}{0}{6}{6}{2W6+0}{Zerbrechlich, Ghulgift}\trennlinie \kreaturattribute{GE 8, KK 8, KL 2, KO 8, MU 8}\kreaturfertigkeiten{Pirschen 8, Wachsamkeit 6}\trennlinie \kreaturinfo{Info}{Varianten:  \newline%
Fresser (WS 4/6)\newline%
Sammler (WS 3/5 KO:  10, KL: 4, AT 4, VT 4 (Leichenteilträger (Säcke), Menschenähnlicher, buckelig))\newline%
Späher (WS 2/4, Klettern 7, Wachsamkeit 10, Sinnesschärfe 10, AT 9, VT 9, BHKI, BHKII (spinnenartig, schnell))\newline%
Wühler (WS 4/6, AT 4, VT 4, Klauen 1W6+4, Graben II (blind, milchige Augen, sehr leicht zu übersehen: Tarnung(Unter der Erde))\newline%
}\trennlinie \kreaturinfo{Quelle}{\href{https://www.orkenspalter.de/filebase/index.php?file/2829-hilberts-bestiarium/}{Hilberts Bestarium}}}}
\newcommand{\kreaturdetailghulkoenig}{\kreatur{Ghulkönig}{verschmolzener großer Haufen Leichenfresser}{gfx/kreaturen/tier}{\kreaturkampfwerte{8}{12}{6}{8}\trennlinie \kreaturinfo{Empfindlichkeit II}{Sonnenlicht}\kreaturinfo{Immunität}{Einfluss}\kreaturinfo{Regemutation}{Verschmelzen mit Ghulen heilt 1 Wunde, bei großen Ghulen sogar 2 Wunden}\kreaturinfo{Angepasst II (Dunkelheit)}{Angepasst I/II (Umgebung) Durch deine Spezies oder langjährige Erfahrung hast du dich an eine bestimmte Umgebung oder Umweltbedingung gewöhnt. Abzüge durch diese Umgebung (Beispiele auf S. 38), insbesondere im Kampf, sinken für dich um eine/zwei Stufen. Die Kosten für Angepasst legt der Spielleiter fest, wobei er sich an der Häufigkeit der Umgebung orientieren sollte. Zu allgemein gefasste Umgebungen wie „unsicherer Untergrund“ sollte er nicht zulassen. Beispiele für Angepasst sind: • Dunkelheit: verringert Abzüge durch schlechte Lichtverhältnisse (40 EP pro Stufe) • Schnee: verringert Abzüge durch schneebedeckten oder eisigen Untergrund (20 EP pro Stufe) • Wasser: verringert Abzüge durch knie{-} oder hüfttiefes Wasser und unter Wasser (20 EP pro Stufe) • Wald: verringert Abzüge durch Wurzeln, Gestrüpp und dichtes Unterholz (40 EP pro Stufe) Voraussetzungen: keine/Angepasst I Nachkauf: häufig/selten\newline%
}\kreaturvorteile{Lichtscheu, Schreckgestalt II, Zusätzliche Attacke III, Zusätzliche Verteidigung I}\trennlinie \kreaturwaffe{Faust}{1}{6}{10}{1W6+6}{Stumpf, Hochansteckend}\kreaturwaffe{Zähne}{0}{8}{14}{2W6+4}{Ghulgift}\kreaturwaffe{Würgen}{2}{}{}{2W6+2}{Säure (KO(I,20) sonst 1 Wunde), Flächenangriff (180 Grad)}\kreaturkampfvorteile{Kraftvoller Kampf III, Niederwerfen, Sturmangriff}\trennlinie \kreaturattribute{GE 6, KK 36, KL 4, KO 32, MU 8}\kreaturfertigkeiten{Laufen 10, Wachsamkeit 6, Pirschen 8, Zähigkeit 16}\trennlinie \kreaturinfo{Quelle}{\href{https://www.orkenspalter.de/filebase/index.php?file/2829-hilberts-bestiarium/}{Hilberts Bestarium}}}}
\newcommand{\kreaturdetailglaathoyub}{\kreatur{Glaa{-}Tho{-}Yub}{Der Schlangenwanst; ein vielgehörnter unabhängiger Dämon, großer Gegner}{gfx/kreaturen/daemon}{\kreaturkampfwerte{4}{16}{2}{1}\trennlinie \kreaturvorteile{Präsenz,, Regeneration II, Schlangenwanst}\trennlinie \kreaturwaffe{Prankenhieb}{1}{12}{16}{2W6+4}{Doppelangriff, Niederwerfen ({-}4)}\kreaturwaffe{Biss}{0}{2}{16}{5W6+2}{Nachbrennen (Säure)}\trennlinie \kreaturfertigkeiten{Wachsamkeit 16, Pirschen 6}\trennlinie \kreaturinfo{AsP}{64}\kreaturinfo{Dämonisch}{16 (Exposami, Fluch der Wandlung, Krähenruf (Schlangen), Krabbelnder Schrecken, Reptilea)}\trennlinie \kreaturinfo{Beschwörung}{Invocatio}\kreaturinfo{Dienste}{Dorf mit Schlangen tyrannisieren+4 (1 Woche), Viehherde vernichten+0, Wache+4 (1 Monat)}\trennlinie \kreaturinfo{Info}{Der Dämon trägt 10 Schlangenschwärme in seinem Bauch, die er jeden Tag aussendet um Lebewesen zu fressen. Pro Schwarm, der vollgefressen zu ihm zurückkehrt, erhöht sich seine WS um +1, bis zu einem Maximum von 14. Werte für einen Schwarm: WS 5, koloss 1, INI 4, GS 2, PA 6, RW 0, AT 12, TP 2W6 (Nachbrennen (Säure), Schmerzimmun II, Resistenz II (Stichwaffen), Verwundbarkeit IV (Flächenschaden)), Schreckgestalt II\newline%
}\trennlinie \kreaturinfo{Quelle}{\href{https://dsaforum.de/viewtopic.php?p=1887303\#p1887303}{Pandämonium}}}}
\newcommand{\kreaturdetailglazmadraa}{\kreatur{GLAZ'MADRAA}{Der Herzensfrost; siebengehörnter Diener Belshirashs, sehr großer Gegner}{gfx/kreaturen/daemon}{\kreaturkampfwerte{16}{20}{8}{6}\trennlinie \kreaturinfo{Aura}{Kälte, Zähigkeit (24), alle 4 INI: phasen, 1 Wunde}\kreaturinfo{Verwundbarkeit I}{Feuer}\kreaturinfo{Zusätzliche Attacke I}{nur Trampeln}\kreaturvorteile{Regeneration I, Schreckgestalt II}\trennlinie \kreaturwaffe{Faust}{2}{12}{16}{2W20+5}{Niederwerfen ({-}12)}\kreaturwaffe{Keule}{5}{12}{18}{3W20+10}{Flächenangriff (180° vor dem Dämon), Niederwerfen ({-}16), Zurückstoßen}\kreaturwaffe{Trampeln}{1}{6}{16}{4W20+0}{Flächenangriff (1 Schritt Umkreis), Niederwerfen ({-}16), Überrennen (wird nicht durch erfolgreiche VT: gestoppt)}\kreaturkampfvorteile{Sturmangriff, Unaufhaltsam}\trennlinie \kreaturfertigkeiten{Laufen 16, Pirschen 2, Sinnenschärfe 24, Wachsamkeit 24}\trennlinie \kreaturinfo{AsP}{128}\kreaturinfo{Dämonisch}{24 (Caldofrigo, Frigisphaero, Gletscherwand, Warmes Gefriere!)}\trennlinie \kreaturinfo{Beschwörung}{Invocatio}\kreaturinfo{Dienste}{Person suchen und töten+4 (1 Tag), Burg einfrieren+0 (1 Woche), Dorf zerstören{-}4}\trennlinie \kreaturinfo{Quelle}{\href{https://dsaforum.de/viewtopic.php?p=1887303\#p1887303}{Pandämonium}}}}
\newcommand{\kreaturdetailgnarishajtumar}{\kreatur{Gna{-}Rishaj{-}Tumar}{Der Architekt des Ewigen Bauplatzes der Berstenden Zitadelle; ein viergehörnter Diener Agrimoths, sehr großer Gegner}{gfx/kreaturen/daemon}{\kreaturkampfwerte{12}{18}{2}{4}\trennlinie \kreaturvorteile{Regeneration I}\trennlinie \kreaturwaffe{Prankenhieb}{1}{6}{16}{3W6+2}{Umklammern ({-}4, 16)}\trennlinie \kreaturfertigkeiten{Wachsamkeit 20, Pirschen 14, Untertauchen 14}\kreaturfertigkeiten{Baumeister 3}\trennlinie \kreaturinfo{AsP}{64}\kreaturinfo{Dämonisch}{16 (Fesselranken, Hartes Schmelze!, Leib des Erzes))}\trennlinie \kreaturinfo{Beschwörung}{Invocatio}\kreaturinfo{Dienste}{Kontrolle von Baudämonen+4 (1 Woche), Gestein formen+0 (1 Tag), Transport schwerer Lasten{-}4 (1 Tag)}\trennlinie \kreaturinfo{Quelle}{\href{https://dsaforum.de/viewtopic.php?p=1887303\#p1887303}{Pandämonium}}}}
\newcommand{\kreaturdetailgoblin}{\kreatur{Goblin}{rotpelzige Landplage des Nordens}{gfx/kreaturen/humanoid}{\kreaturkampfwerte{4}{4}{5}{3}\trennlinie \kreaturinfo{Angepasst I (Dunkelheit)}{Angepasst I/II (Umgebung) Durch deine Spezies oder langjährige Erfahrung hast du dich an eine bestimmte Umgebung oder Umweltbedingung gewöhnt. Abzüge durch diese Umgebung (Beispiele auf S. 38), insbesondere im Kampf, sinken für dich um eine/zwei Stufen. Die Kosten für Angepasst legt der Spielleiter fest, wobei er sich an der Häufigkeit der Umgebung orientieren sollte. Zu allgemein gefasste Umgebungen wie „unsicherer Untergrund“ sollte er nicht zulassen. Beispiele für Angepasst sind: • Dunkelheit: verringert Abzüge durch schlechte Lichtverhältnisse (40 EP pro Stufe) • Schnee: verringert Abzüge durch schneebedeckten oder eisigen Untergrund (20 EP pro Stufe) • Wasser: verringert Abzüge durch knie{-} oder hüfttiefes Wasser und unter Wasser (20 EP pro Stufe) • Wald: verringert Abzüge durch Wurzeln, Gestrüpp und dichtes Unterholz (40 EP pro Stufe) Voraussetzungen: keine/Angepasst I Nachkauf: häufig/selten\newline%
}\kreaturvorteile{Resistenz gegen Kälte}\trennlinie \kreaturwaffe{Holzspeer}{2}{8}{8}{2W6{-}1}{Wendig}\kreaturwaffe{Keule}{1}{8}{8}{2W6+0}{Kopflastig, Stumpf}\kreaturwaffe{Kurzbogen}{16}{}{}{2W6+1}{}\trennlinie \kreaturattribute{CH 4, FF 10, GE 8, IN 6, KK 4, KL 4, KO 4, MU 4}\kreaturfertigkeiten{Laufen 6, Pirschen 8, Wachsamkeit 6, Zähigkeit 4}\trennlinie \kreaturinfo{Quelle}{\href{https://ilarisblog.wordpress.com/downloads/}{Ilaris Regeln}}}}
\newcommand{\kreaturdetailgrakvaloth}{\kreatur{Grakvaloth}{Bote der Niederhöllen, das ungesehene Grauen; ein viergehörnter Diener des Namenlosen, großer Gegner}{gfx/kreaturen/daemon}{\kreaturkampfwerte{10}{18}{10}{10}\trennlinie \kreaturinfo{Resistenz I}{geweiht}\kreaturinfo{Unsichtbarkeit}{kann sich nach Wunsch nur bestimmten Personen, meist dem Opfer, zeigen}\kreaturvorteile{Flugfähig, Schreckgestalt II, Zusätzliche Attacke I}\trennlinie \kreaturwaffe{Prankenhieb}{1}{12}{16}{2W6+4}{Doppelangriff, Niederwerfen ({-}4)}\kreaturwaffe{Biss}{0}{2}{16}{5W6+2}{}\kreaturkampfvorteile{Sturmangriff}\trennlinie \kreaturfertigkeiten{Wachsamkeit 20, Pirschen 24, Laufen 16}\trennlinie \kreaturinfo{Beschwörung}{Invocatio}\kreaturinfo{Dienste}{Wache+4 (1 Woche), Kampf+0 (1 Minute), Person suchen und töten{-}4 (1 Tag)}\trennlinie \kreaturinfo{Quelle}{\href{https://dsaforum.de/viewtopic.php?p=1887303\#p1887303}{Pandämonium}}}}
\newcommand{\kreaturdetailgregorroth}{\kreatur{Gregorroth}{Meister der Kakophonie; niederer Diener Amazeroths oder Belkelels}{gfx/kreaturen/daemon}{\kreaturkampfwerte{1}{12}{6}{8}\trennlinie \kreaturinfo{Unsichtbarkeit}{kann sich nach Wunsch nur bestimmten Personen, meist dem Opfer, zeigen}\kreaturvorteile{Besessenheit}\trennlinie \kreaturwaffe{Ausweichen}{0}{2}{}{}{}\trennlinie \kreaturinfo{AsP}{32}\kreaturinfo{Einfluss}{12 (Melodie des Einlullens, Melodie der Verwirrung, Schriller Klang)}\trennlinie \kreaturinfo{Beschwörung}{Invocatio}\kreaturinfo{Dienste}{Besessenheit eines Musikinstruments+4 (1 Woche), Besessenheit eines Musikers+0 (1 Stunde), Verwirrung einer Festgesellschaft{-}4 (1 Stunde)}\trennlinie \kreaturinfo{Info}{Der Dämon kann von einem Musikinstrument oder ein Wesen Besitz ergreifen. Er verliert Besessenheit, WS, GS, Ausweichen und erhält die Eigenschaften/Attacken des Ziels (bei einem Wesen). Musiker oder Instrumente können keine wohlklingende Musik mehr erzeugen, eventuelle Proben auf entsprechende Freie Fertigkeiten sind um +12 erschwert.), Sphärenklang (alle Personen in einer Zone von 32 Schritt Durchmesser um den Dämon müssen eine Probe auf MR (20) bestehen, um nicht Opfer eines zufälligen Zaubers der Fertigkeit Einfluss zu werden.\newline%
}\trennlinie \kreaturinfo{Quelle}{\href{https://dsaforum.de/viewtopic.php?p=1887303\#p1887303}{Pandämonium}}}}
\newcommand{\kreaturdetailgreif}{\kreatur{Greif}{Mythenwesen und mächtiger Diener Praios‘; großer Gegner}{gfx/kreaturen/mythen}{\kreaturkampfwerte{14}{20}{6}{8}\trennlinie \kreaturvorteile{Flugfähig, Willensstark II, Geisterpanzer (WS 20), Unbeugsamkeit}\trennlinie \kreaturwaffe{Prankenhieb}{1}{11}{17}{2W6+4}{}\kreaturwaffe{Schnabelhieb}{0}{2}{17}{3W6+4}{}\kreaturkampfvorteile{Zusätzlicher Attacke I, Kommando: Haltet Stand!, Hammerschlag, Niederwerfen, Standfest, Sturmangriff}\trennlinie \kreaturattribute{GE 16, KK 24, KO 26, MU 40}\kreaturfertigkeiten{Anführen 10, Menschenkenntnis 20, Pirschen 0, Wachsamkeit 16, Zähigkeit 16}\trennlinie \kreaturinfo{KaP}{64}\kreaturinfo{Ordnung und Magiebann}{16 (alle Praiosliturgien)}\kreaturinfo{Karmale Vorteile}{Aura der Heiligkeit, Gesegnete Waffe, Liebling Praios, Liturgische Disziplin, Liturgische Routine, Stärke des Glaubens, Streiter der Schöpfung, Unterstützung der Gläubigen, Verseuchung erspüren, Licht}\trennlinie \kreaturinfo{Info}{Gesegnete Waffe, Stärke des Glaubens, gegen unheilige Kreaturen: TP:  +1W6, WS +2, AT/VT:  +2, Liturgien +4}\trennlinie \kreaturinfo{Quelle}{\href{https://ilarisblog.wordpress.com/downloads/}{Ilaris Regeln}}}}
\newcommand{\kreaturdetailgrosserhund}{\kreatur{Großer Hund}{verbreitetes Rudeltier}{gfx/kreaturen/tier}{\kreaturkampfwerte{4}{2}{8}{2}\trennlinie \kreaturinfo{Angepasst II (Dunkelheit)}{Angepasst I/II (Umgebung) Durch deine Spezies oder langjährige Erfahrung hast du dich an eine bestimmte Umgebung oder Umweltbedingung gewöhnt. Abzüge durch diese Umgebung (Beispiele auf S. 38), insbesondere im Kampf, sinken für dich um eine/zwei Stufen. Die Kosten für Angepasst legt der Spielleiter fest, wobei er sich an der Häufigkeit der Umgebung orientieren sollte. Zu allgemein gefasste Umgebungen wie „unsicherer Untergrund“ sollte er nicht zulassen. Beispiele für Angepasst sind: • Dunkelheit: verringert Abzüge durch schlechte Lichtverhältnisse (40 EP pro Stufe) • Schnee: verringert Abzüge durch schneebedeckten oder eisigen Untergrund (20 EP pro Stufe) • Wasser: verringert Abzüge durch knie{-} oder hüfttiefes Wasser und unter Wasser (20 EP pro Stufe) • Wald: verringert Abzüge durch Wurzeln, Gestrüpp und dichtes Unterholz (40 EP pro Stufe) Voraussetzungen: keine/Angepasst I Nachkauf: häufig/selten\newline%
}\kreaturvorteile{}\trennlinie \kreaturwaffe{Biss}{0}{6}{6}{2W6+0}{Zerbrechlich}\kreaturkampfvorteile{Niederwerfen, Standfest, Sturmangriff}\trennlinie \kreaturattribute{GE 14, KK 6, KO 6, MU 6}\kreaturfertigkeiten{Laufen 24, Pirschen 12, Wachsamkeit 18, Zähigkeit 4}\trennlinie \kreaturinfo{Quelle}{\href{https://ilarisblog.wordpress.com/downloads/}{Ilaris Regeln}}}}
\newcommand{\kreaturdetailgrosserorciosil}{\kreatur{Großer Orciosil}{Die brüllende Seele des Sturms; ein minderer Diener Agrimoths, mittlerer Gegner}{gfx/kreaturen/daemon}{\kreaturkampfwerte{3}{8}{1}{{-}1}\trennlinie \kreaturvorteile{Rudel}\trennlinie \kreaturwaffe{Zunge}{1}{4}{12}{1W6+2}{}\trennlinie \kreaturfertigkeiten{Pirschen 18, Wachsamkeit 12}\trennlinie \kreaturinfo{AsP}{16}\kreaturinfo{Dämonisch}{10 (Aeolitus)}\trennlinie \kreaturinfo{Beschwörung}{Invocatio}\kreaturinfo{Dienste}{Segelboot antreiben+4 (1 Tag), Gegner auf Abstand halten+0 (1 Minute), Wachstum zu einem großen Orciosil{-}4 (1 Tag)}\trennlinie \kreaturinfo{Quelle}{\href{https://dsaforum.de/viewtopic.php?p=1887303\#p1887303}{Pandämonium}}}}
\newcommand{\kreaturdetailgruftassel}{\kreatur{Gruftassel}{flinker Aasfresser}{gfx/kreaturen/tier}{\kreaturkampfwerte{4}{11}{5}{3}\trennlinie \kreaturinfo{Angepasst II (Dunkelheit)}{Angepasst I/II (Umgebung) Durch deine Spezies oder langjährige Erfahrung hast du dich an eine bestimmte Umgebung oder Umweltbedingung gewöhnt. Abzüge durch diese Umgebung (Beispiele auf S. 38), insbesondere im Kampf, sinken für dich um eine/zwei Stufen. Die Kosten für Angepasst legt der Spielleiter fest, wobei er sich an der Häufigkeit der Umgebung orientieren sollte. Zu allgemein gefasste Umgebungen wie „unsicherer Untergrund“ sollte er nicht zulassen. Beispiele für Angepasst sind: • Dunkelheit: verringert Abzüge durch schlechte Lichtverhältnisse (40 EP pro Stufe) • Schnee: verringert Abzüge durch schneebedeckten oder eisigen Untergrund (20 EP pro Stufe) • Wasser: verringert Abzüge durch knie{-} oder hüfttiefes Wasser und unter Wasser (20 EP pro Stufe) • Wald: verringert Abzüge durch Wurzeln, Gestrüpp und dichtes Unterholz (40 EP pro Stufe) Voraussetzungen: keine/Angepasst I Nachkauf: häufig/selten\newline%
}\kreaturvorteile{}\trennlinie \kreaturwaffe{Zange}{0}{5}{9}{2W6+0}{Rüstungsbrechend}\kreaturkampfvorteile{Doppelangriff, Standfest}\trennlinie \kreaturattribute{GE 4, KK 6, KO 12, MU 4}\kreaturfertigkeiten{Pirschen 12, Wachsamkeit 10, Zähigkeit 10}\trennlinie \kreaturinfo{Quelle}{\href{https://dsaforum.de/viewtopic.php?p=1887303\#p1887303}{Allgemeine Gegner}}}}
\newcommand{\kreaturdetailgurgulum}{\kreatur{Gurgulum}{Der Dunkelschlund, ein niederer Diener des Blakharaz, sehr kleiner Gegner}{gfx/kreaturen/daemon}{\kreaturkampfwerte{3}{10}{1}{4}\trennlinie \kreaturinfo{Unsichtbarkeit}{kann sich nach Wunsch nur bestimmten Personen, meist dem Opfer, zeigen}\kreaturvorteile{Kritische Konsistenz, Opfer übernehmen}\trennlinie \kreaturwaffe{Ausweichen}{0}{4}{}{}{}\kreaturwaffe{Würgen}{0}{2}{12}{2W6+0}{Umklammern ({-}2, 12)}\trennlinie \kreaturinfo{AsP}{24}\kreaturinfo{Einfluss}{10 (Große Gier, Imperavi)}\trennlinie \kreaturinfo{Beschwörung}{Invocatio}\kreaturinfo{Dienste}{Kontrolle über Wirt übernehmen+0 (1 Monat), Kontrolle über Wirt übernehmen+4 (6 Monate)}\trennlinie \kreaturinfo{Info}{Der Durgulum legt sich um den Hals des Opfers und verliert alle Angriffe mit *. Er kann kurzfristig die Kontrolle über das Opfer übernehmen oder dieses zur Strafe würgen. Die Einnahme von Rauschmitteln lässt den Durgulum für einige Zeit schlafen.\newline%
}\trennlinie \kreaturinfo{Quelle}{\href{https://dsaforum.de/viewtopic.php?p=1887303\#p1887303}{Pandämonium}}}}
\newcommand{\kreaturdetailhammerhai}{\kreatur{Hammerhai}{Räuber der Meere}{gfx/kreaturen/tier}{\kreaturkampfwerte{10}{4}{8}{4}\trennlinie \kreaturvorteile{Wasserwesen}\trennlinie \kreaturwaffe{Biss}{0}{2}{8}{2W6+0}{Zerbrechlich}\kreaturkampfvorteile{Sturmangriff, Raserei 2 (bei Verletzung: AT+2, VT{-}2, TP+2)}\trennlinie \kreaturattribute{GE 14, KK 20, KO 20, MU 12}\kreaturfertigkeiten{Pirschen 8, Schwimmen 16, Wachsamkeit 12, Zähigkeit 14}\trennlinie \kreaturinfo{Quelle}{\href{https://dsaforum.de/viewtopic.php?p=1887303\#p1887303}{Allgemeine Gegner}}}}
\newcommand{\kreaturdetailhanaestil}{\kreatur{Hanaestil}{Die Allschöne, der verderbte Traum, die dunkle Verführerin; fünfgehörnte (vermutlich einzigartige) Dienerin der Shaz-man-yat}{gfx/kreaturen/daemon}{\kreaturkampfwerte{14}{14}{14}{10}\trennlinie \kreaturinfo{Unsichtbarkeit}{kann sich nach Wunsch nur bestimmten Personen, meist dem Opfer, zeigen}\kreaturvorteile{}\trennlinie \kreaturwaffe{Ausweichen}{0}{20}{}{}{}\kreaturwaffe{Krallen}{1}{20}{16}{3W6+4}{}\kreaturkampfvorteile{Defensiver Kampfstil}\trennlinie \kreaturfertigkeiten{Betören 24, Wachsamkeit 20}\trennlinie \kreaturinfo{AsP}{64}\kreaturinfo{Einfluss}{16 (Bannbaladin, Imperavi, Levthans Feuer, Satuarias Herrlichkeit)}\trennlinie \kreaturinfo{Beschwörung}{Invocatio}\kreaturinfo{Dienste}{Liebesspiel mit dem Beschwörer+4 (1 Stunde), Person in der Nähe verführen+0 (1 Tag), Person verführen und töten{-}4 (1 Tag)}\trennlinie \kreaturinfo{Quelle}{\href{https://dsaforum.de/viewtopic.php?p=1887303\#p1887303}{Pandämonium}}}}
\newcommand{\kreaturdetailhaqoum}{\kreatur{Haqoum}{Der Verfluchte Feilscher, der dämonische Steuereintreiber; ein eingehörnter Diener Tasfarelels oder Amazeroths, kleiner Gegner
}{gfx/kreaturen/daemon}{\kreaturkampfwerte{8}{10}{4}{8}\trennlinie \kreaturinfo{Tarnung}{in Gebäuden}\kreaturvorteile{Regeneration I}\trennlinie \kreaturwaffe{Kurzschwert}{0}{12}{12}{2W6+0}{Wendig}\kreaturwaffe{Leichte Armbrust}{32}{}{}{3W6+1}{Zweihändig}\kreaturkampfvorteile{Reflexschuss, Ruhige Hand, Schnellziehen}\trennlinie \kreaturfertigkeiten{Pirschen 16, Untertauchen 20, Wachsamkeit 12}\trennlinie \kreaturinfo{AsP}{32}\kreaturinfo{Dämonisch}{12 (Duplicatus, Foramen, Impersona, Motoricus, Somnigravis, Visibili)}\trennlinie \kreaturinfo{Beschwörung}{Invocatio}\kreaturinfo{Dienste}{Geld von Schuldiger eintreiben+4 (1 Woche), Diebstahl+0 (1 Tag), Person aufspüren und meucheln{-}4 (1 Tag)}\trennlinie \kreaturinfo{Quelle}{\href{https://dsaforum.de/viewtopic.php?p=1887303\#p1887303}{Pandämonium}}}}
\newcommand{\kreaturdetailharpyie}{\kreatur{Harpyie}{wahnsinnige Verschmelzung von Vogel und Frau}{gfx/kreaturen/tier}{\kreaturkampfwerte{4}{4}{2}{8}\trennlinie \kreaturvorteile{Flieger}\trennlinie \kreaturwaffe{Klauen}{0}{8}{12}{2W6+0}{}\kreaturkampfvorteile{Niederwerfen, Sturmangriff}\trennlinie \kreaturattribute{GE 16, KK 6, KO 4, MU 4}\kreaturfertigkeiten{Fliegen 12, Wachsamkeit 10}\trennlinie \kreaturinfo{Quelle}{\href{https://ilarisblog.wordpress.com/downloads/}{Ilaris Regeln}}}}
\newcommand{\kreaturdetailhektabelus}{\kreatur{Hektabelus}{Fliegender Bote der Pestilenz; gehörnter Diener Belzhorashs, sehr kleiner Gegner}{gfx/kreaturen/daemon}{\kreaturkampfwerte{1}{12}{}{8}\trennlinie \kreaturinfo{Unsichtbarkeit}{kann sich nach Wunsch nur bestimmten Personen, meist dem Opfer, zeigen}\kreaturvorteile{Flieger, Rudel}\trennlinie \kreaturwaffe{Ausweichen}{0}{2}{}{}{}\trennlinie \kreaturinfo{AsP}{32}\kreaturinfo{Dämonisch}{12 (Fluch der Pestilenz)}\trennlinie \kreaturinfo{Beschwörung}{Invocatio}\kreaturinfo{Dienste}{Nahes Ziel mit einer Krankheit bis Stufe 24 infizieren+4, Besessenheit eines Erkrankten und weitere Verbreitung der Krankheit+0 (1 Tag), Besitzer eines vorhandenen Körperteils suchen und mit einer Krankheit bis Stufe 24 infizieren{-}4}\trennlinie \kreaturinfo{Quelle}{\href{https://dsaforum.de/viewtopic.php?p=1887303\#p1887303}{Pandämonium}}}}
\newcommand{\kreaturdetailhetzer}{\kreatur{Hetzer}{Seele eines Mörders oder Attentäters}{gfx/kreaturen/geist}{\kreaturkampfwerte{2}{5}{8}{5}\trennlinie \kreaturvorteile{Tarnung, Rudel}\trennlinie \kreaturwaffe{Krallen}{0}{4}{12}{2W6+4}{Rüstungsbrechend}\kreaturkampfvorteile{Sturmangriff}\trennlinie \kreaturattribute{CH 4, FF 10, GE 6, IN 10, KK 4, KL 2, KO 2, MU 10}\kreaturfertigkeiten{Einschüchtern 6, Laufen 6, Menschenkenntnis 6, Pirschen 8}\trennlinie \kreaturinfo{AsP}{16}\kreaturinfo{Hellsicht}{8 (Exposami Lebenskraft, Sensattaco Meisterstreich)}\kreaturinfo{Einfluss}{8 (Harmlose Gestalt, Blitz dich Find)}\trennlinie \kreaturinfo{Quelle}{\href{https://www.orkenspalter.de/filebase/index.php?file/2829-hilberts-bestiarium/}{Hilberts Bestarium}}}}
\newcommand{\kreaturdetailhirrnirat}{\kreatur{HIRR'NIRAT}{Der Rattenfürst; niederer Diener Belzhorashs, sehr kleiner Gegner}{gfx/kreaturen/daemon}{\kreaturkampfwerte{2}{8}{6}{10}\trennlinie \kreaturvorteile{Rudel}\trennlinie \kreaturwaffe{Biss}{0}{6}{14}{2W6+1}{}\trennlinie \kreaturfertigkeiten{Untertauchen 16, Wachsamkeit 8}\trennlinie \kreaturinfo{AsP}{24}\kreaturinfo{Einfluss}{10 (Krähenruf)}\trennlinie \kreaturinfo{Beschwörung}{Invocatio}\kreaturinfo{Dienste}{Kontrolle über Rattenschwarm+4 (1 Woche), Lebensmittel vernichten+0 (1 Stunde), Pestrattenschwarm beschwören und Krankheiten verbreiten{-}4}\trennlinie \kreaturinfo{Info}{Krähenruf mit Ratten statt Krähen, Werte wie Pestrattenschwarm}\trennlinie \kreaturinfo{Quelle}{\href{https://dsaforum.de/viewtopic.php?p=1887303\#p1887303}{Pandämonium}}}}
\newcommand{\kreaturdetailhoehlenspinne}{\kreatur{Höhlenspinne}{einzelgängerische Spinne mit Lähmungsgift}{gfx/kreaturen/tier}{\kreaturkampfwerte{3}{9}{3}{3}\trennlinie \kreaturinfo{Angepasst II (Dunkelheit)}{Angepasst I/II (Umgebung) Durch deine Spezies oder langjährige Erfahrung hast du dich an eine bestimmte Umgebung oder Umweltbedingung gewöhnt. Abzüge durch diese Umgebung (Beispiele auf S. 38), insbesondere im Kampf, sinken für dich um eine/zwei Stufen. Die Kosten für Angepasst legt der Spielleiter fest, wobei er sich an der Häufigkeit der Umgebung orientieren sollte. Zu allgemein gefasste Umgebungen wie „unsicherer Untergrund“ sollte er nicht zulassen. Beispiele für Angepasst sind: • Dunkelheit: verringert Abzüge durch schlechte Lichtverhältnisse (40 EP pro Stufe) • Schnee: verringert Abzüge durch schneebedeckten oder eisigen Untergrund (20 EP pro Stufe) • Wasser: verringert Abzüge durch knie{-} oder hüfttiefes Wasser und unter Wasser (20 EP pro Stufe) • Wald: verringert Abzüge durch Wurzeln, Gestrüpp und dichtes Unterholz (40 EP pro Stufe) Voraussetzungen: keine/Angepasst I Nachkauf: häufig/selten\newline%
}\kreaturvorteile{Immunität (Niederwerfen, Umreißen und ähnliche Effekte)}\trennlinie \kreaturwaffe{Biss}{0}{6}{9}{1W6+2}{Giftig (Spinnengift, Stufe 20, Verzögerung 0, Wirkungsdauer 8 Stunden; körperliche   attribute und Fertigkeiten sind kumulativ um {-}1 erschwert)}\trennlinie \kreaturattribute{KK 4, KO 4, MU 4}\kreaturfertigkeiten{Wachsamkeit 14, Pirschen 14, Zähigkeit 8}\trennlinie \kreaturinfo{Info}{Das Netz der Spinne ist im Fackelschein mit einer Wachsamkeits{-}Probe (24) zu erkennen. Misslingt die Probe, verfängt sich ein Opfer im Netz, das wie ein Umklammern ({-}4, 20) wirkt.}\trennlinie \kreaturinfo{Quelle}{\href{None}{None}}}}
\newcommand{\kreaturdetailhornechse}{\kreatur{Hornechse}{friedfertiger Pflanzenfresser und Reittier der Achaz; großer Gegner}{gfx/kreaturen/tier}{\kreaturinfo{Angepasst I (Sumpf)}{Angepasst I/II (Umgebung) Durch deine Spezies oder langjährige Erfahrung hast du dich an eine bestimmte Umgebung oder Umweltbedingung gewöhnt. Abzüge durch diese Umgebung (Beispiele auf S. 38), insbesondere im Kampf, sinken für dich um eine/zwei Stufen. Die Kosten für Angepasst legt der Spielleiter fest, wobei er sich an der Häufigkeit der Umgebung orientieren sollte. Zu allgemein gefasste Umgebungen wie „unsicherer Untergrund“ sollte er nicht zulassen. Beispiele für Angepasst sind: • Dunkelheit: verringert Abzüge durch schlechte Lichtverhältnisse (40 EP pro Stufe) • Schnee: verringert Abzüge durch schneebedeckten oder eisigen Untergrund (20 EP pro Stufe) • Wasser: verringert Abzüge durch knie{-} oder hüfttiefes Wasser und unter Wasser (20 EP pro Stufe) • Wald: verringert Abzüge durch Wurzeln, Gestrüpp und dichtes Unterholz (40 EP pro Stufe) Voraussetzungen: keine/Angepasst I Nachkauf: häufig/selten\newline%
}\kreaturvorteile{Verwundbarkeit I (Eis)}\trennlinie \kreaturwaffe{Horn}{0}{4}{8}{4W6+6}{Niederwerfen ({-}8)}\kreaturkampfvorteile{Sturmangriff, Überrennen}\trennlinie \kreaturattribute{GE 6, KK 48, KO 54, MU 12}\kreaturfertigkeiten{Wachsamkeit 8, Zähigkeit 16}\trennlinie \kreaturinfo{Quelle}{\href{None}{None}}}}
\newcommand{\kreaturdetailhummerier}{\kreatur{Hummerier}{Schreckliche Ausgeburt der blutigen See}{gfx/kreaturen/daimonid}{\kreaturkampfwerte{8}{4}{1}{2}\trennlinie \kreaturinfo{Angepasst II (Wasser)}{Angepasst I/II (Umgebung) Durch deine Spezies oder langjährige Erfahrung hast du dich an eine bestimmte Umgebung oder Umweltbedingung gewöhnt. Abzüge durch diese Umgebung (Beispiele auf S. 38), insbesondere im Kampf, sinken für dich um eine/zwei Stufen. Die Kosten für Angepasst legt der Spielleiter fest, wobei er sich an der Häufigkeit der Umgebung orientieren sollte. Zu allgemein gefasste Umgebungen wie „unsicherer Untergrund“ sollte er nicht zulassen. Beispiele für Angepasst sind: • Dunkelheit: verringert Abzüge durch schlechte Lichtverhältnisse (40 EP pro Stufe) • Schnee: verringert Abzüge durch schneebedeckten oder eisigen Untergrund (20 EP pro Stufe) • Wasser: verringert Abzüge durch knie{-} oder hüfttiefes Wasser und unter Wasser (20 EP pro Stufe) • Wald: verringert Abzüge durch Wurzeln, Gestrüpp und dichtes Unterholz (40 EP pro Stufe) Voraussetzungen: keine/Angepasst I Nachkauf: häufig/selten\newline%
}\kreaturvorteile{Natürliche Rüstung, Amphibisch}\trennlinie \kreaturwaffe{Scheren}{0}{10}{10}{3W6+0}{Doppelangriff}\kreaturwaffe{Partisane}{2}{11}{11}{3W6+7}{}\kreaturkampfvorteile{Kraftvoller Kampf II, Ausfall, Offensiver Kampfstil, Niederwerfen, Hammerschlag, Eisern, Zäher Hund}\trennlinie \kreaturattribute{CH 4, FF 2, GE 4, IN 4, KK 16, KL 6, KO 16, MU 10}\kreaturfertigkeiten{Einschüchtern 12, Wachsamkeit 4, Zähigkeit 12}\kreaturinfo{Profane Vorteile}{Abgehärtet I, Abgehärtet II, Schnelle Heilung, Muskelprotz, Zerstörerisch II}\trennlinie \kreaturinfo{Quelle}{\href{https://dsaforum.de/viewtopic.php?f=180&p=1738549\#p1738549}{Bestarium+}}}}
\newcommand{\kreaturdetailhummerierpriester}{\kreatur{Hummerier{-}Priester}{Anführer und Veteran der Truppen der tiefen Tochter}{gfx/kreaturen/daimonid}{\kreaturkampfwerte{12}{4}{1}{2}\trennlinie \kreaturinfo{Angepasst II (Wasser)}{Angepasst I/II (Umgebung) Durch deine Spezies oder langjährige Erfahrung hast du dich an eine bestimmte Umgebung oder Umweltbedingung gewöhnt. Abzüge durch diese Umgebung (Beispiele auf S. 38), insbesondere im Kampf, sinken für dich um eine/zwei Stufen. Die Kosten für Angepasst legt der Spielleiter fest, wobei er sich an der Häufigkeit der Umgebung orientieren sollte. Zu allgemein gefasste Umgebungen wie „unsicherer Untergrund“ sollte er nicht zulassen. Beispiele für Angepasst sind: • Dunkelheit: verringert Abzüge durch schlechte Lichtverhältnisse (40 EP pro Stufe) • Schnee: verringert Abzüge durch schneebedeckten oder eisigen Untergrund (20 EP pro Stufe) • Wasser: verringert Abzüge durch knie{-} oder hüfttiefes Wasser und unter Wasser (20 EP pro Stufe) • Wald: verringert Abzüge durch Wurzeln, Gestrüpp und dichtes Unterholz (40 EP pro Stufe) Voraussetzungen: keine/Angepasst I Nachkauf: häufig/selten\newline%
}\kreaturvorteile{Natürliche Rüstung, Amphibisch}\trennlinie \kreaturwaffe{Scheren}{0}{12}{12}{3W6+2}{Doppelangriff}\kreaturwaffe{Partisane}{2}{13}{13}{3W6+8}{}\kreaturkampfvorteile{Kraftvoller Kampf III, Ausfall, Offensiver Kampfstil, Niederwerfen, Hammerschlag, Eisern, Zäher Hund}\trennlinie \kreaturattribute{CH 8, FF 2, GE 4, IN 4, KK 18, KL 6, KO 18, MU 12}\kreaturfertigkeiten{Einschüchtern 16, Wachsamkeit 4, Zähigkeit 14}\kreaturinfo{Profane Vorteile}{Abgehärtet I, Abgehärtet II, Schnelle Heilung, Muskelprotz, Zerstörerisch II, Kommando: Haltet Stand!}\trennlinie \kreaturinfo{GuP}{16}\kreaturinfo{Dämonische Stärkung (Derekunde, KO, MU)}{}\kreaturinfo{Meister der Maritimen}{}\kreaturinfo{Gebieter der Gezeiten}{}\kreaturinfo{Ertränken}{}\kreaturinfo{Paktvorteile}{Erzdämonische Tradition II, Paktierer II}\trennlinie \kreaturinfo{Quelle}{\href{https://dsaforum.de/viewtopic.php?f=180&p=1738549\#p1738549}{Bestarium+}}}}
\newcommand{\kreaturdetailhundeblume}{\kreatur{Hundeblume}{Pflanze und Tier}{gfx/kreaturen/daimonid}{\kreaturkampfwerte{2}{15}{0}{4}\trennlinie \kreaturwaffe{Biss}{0}{2}{11}{1W6+2}{}\trennlinie \kreaturattribute{CH 10, FF 2, GE 2, IN 10, KK 10, KL {-}4, KO 10, MU 14}\kreaturfertigkeiten{Wachsamkeit 12, Pirschen 10, Willenskraft 7}\trennlinie \kreaturinfo{Quelle}{\href{https://www.orkenspalter.de/filebase/index.php?file/2829-hilberts-bestiarium/}{Hilberts Bestarium}}}}
\newcommand{\kreaturdetailhyralkor}{\kreatur{Hyralkor, ein Kaiserdrache}{Herrscher seines Landstrichs; sehr großer Endgegner}{gfx/kreaturen/mythen}{\kreaturkampfwerte{{-}1}{17}{4}{9}\trennlinie \kreaturinfo{Aura}{Hitze Zähigkeit (28) jede INI:phase 1 Wunde}\kreaturinfo{Immunität}{Feuer}\kreaturinfo{Verwundbarkeit I}{Eis}\kreaturvorteile{Flugfähig, Schreckgestalt II}\trennlinie \kreaturwaffe{Biss}{2}{10}{19}{2W20+10}{Zerbrechlich}\kreaturwaffe{Flammenstrahl}{32}{}{}{5W6+8}{Nachbrennen}\kreaturwaffe{Flammeninferno}{16}{}{}{4W6+4}{Flächenangriff (90° vor dem Drachen), Nachbrennen}\kreaturwaffe{Trampeln (Beine)}{0}{6}{11}{3W20+0}{Flächenangriff (1 Schritt Umkreis), Niederwerfen ({-}8)}\kreaturwaffe{Pranken (Beine)}{1}{10}{16}{2W20+5}{Niederwerfen ({-}8)}\kreaturwaffe{Schwanzhieb (S.)}{8}{8}{16}{4W6+2}{Flächenangriff (90° hinter dem Drachen), Niederwerfen ({-}4)}\kreaturwaffe{Windstoß (Flügel)}{4}{6}{20}{}{Flächenangriff (360 ° um den Drachen), Zurückstoßen, Raumgreifend (nur einsetzbar, wenn Hyralkor seine Flügel voll spannen kann)}\kreaturkampfvorteile{Hammerschlag, Offensiver Kampfstil, Sturmangriff, Zusätzliche Attacke I}\trennlinie \kreaturattribute{GE 12, KK 200, KL 20, KO 220, MU 30}\kreaturfertigkeiten{Anführen 10, Einschüchtern 20, Fliegen 14, Menschenkenntnis 16, Wachsamkeit 12, Zähigkeit 24}\trennlinie \kreaturinfo{AsP}{140}\kreaturinfo{Einfluss}{20 (Band und Fessel, Bannbaladin, Blitz dich find, Große Gier, Halluzination, Herr über das Tierreich, Horriphobus, Imperavi, Respondami, Zauberzwang)}\kreaturinfo{Hellsicht}{16 (Analys, Blick aufs Wesen, Blick durch fremde Augen, Blick in die Gedaken, Exposami, Oculus, Odem, Sensibar)}\kreaturinfo{Umwelt}{16 (Dunkelheit, Flim Flam, Motoricus, Nihilogravo, Wettermeisterschaft)}\kreaturinfo{Magische Vorteile}{Drachenmagie(Reichweite und Wirkungsdauer jedes Zaubers ist vervielfacht)}\trennlinie \kreaturinfo{vorgehen}{Hyralkor bleibt auf Distanz und nutzt seine Flammenangriffe. Nur wenn seine Flügel schwer verletzt wurden oder er in beengten Verhältnissen kämpft, geht Hyralkor in den Nahkampf über {-} bevorzugt mit einem Sturmangriff aus der Luft. Am Boden verwendet Hyralkor seine Feuerangriffe gegen entfernte Gegner und seinen Prankenhieb gegen Angreifer in Schlagreichweite. Kämpfer unter sich versucht er zu zertrampeln, während er Angreifer hinter sich mit Schwanzhieben zur Seite fegt. Rücken ihm zu viele Angreifer zu nahe, versucht er sie mit einem Windstoß umzuwerfen um sich dann mit einem gewaltigen Satz Luft zu verschaffen. Sein Gebiss setzt er nur gegen schwer angeschlagene oder sehr widerstandsfähige Gegner ein, da er schmerzhafte Gegenangriffe in den Rachen befürchtet. Schwachpunkte: Um den Drachen zu töten, muss sein Rumpf oder Kopf vernichtet werden. Ersterer ist extrem zäh, während zweiterer für Nahkampfwaffen unerreichbar ist. Deswegen sollten zuerst anderer Zonen angegriffen werden: Ohne die Flügel wird Hyralkor auf den Boden gezwungen und wenn seine Beine einknicken, kommt der Kopf in Reichweite. Dabei hilft eine alte Verletzung Hyralkors: Einst rammten ihm die Brüder Turgosch und Targosch einen Drachentöter durch den rechten Hinterlauf. Die Spitze der Waffe steckt noch immer in der Wunde. Kennt ein Charakter die zwergische Geschichte, kann er mit einer GE{-} oder Akrobatik{-}Probe (24) in Position gelangen und die Waffe mit einer AT:  {-}12 weiter ins Bein treiben. Dadurch wird der Drache für eine INI:phase handlungsunfähig und heißes Blut schießt aus der Wunde (Akrobatik (24) oder 4W6 Schadenspunkte). Kampfunfähigkeit: Hyralkor flieht, wenn er 6 Wunden erlitten hat. In seinem Hort kämpft er jedoch bis zum Tod.}\trennlinie \kreaturinfo{Quelle}{\href{https://ilarisblog.wordpress.com/downloads/}{Ilaris Regeln}}}}
\newcommand{\kreaturdetailifirnshai}{\kreatur{Ifirnshai}{Räuber der Meere}{gfx/kreaturen/tier}{\kreaturkampfwerte{9}{4}{8}{4}\trennlinie \kreaturvorteile{Wasserwesen}\trennlinie \kreaturwaffe{Biss}{0}{4}{10}{2W6+2}{Zerbrechlich}\kreaturkampfvorteile{Sturmangriff}\trennlinie \kreaturattribute{GE 14, KK 20, KO 20, MU 12}\kreaturfertigkeiten{Pirschen 8, Schwimmen 16, Wachsamkeit 12, Zähigkeit 14}\trennlinie \kreaturinfo{Quelle}{\href{https://dsaforum.de/viewtopic.php?p=1887303\#p1887303}{Allgemeine Gegner}}}}
\newcommand{\kreaturdetailirhiadhzal}{\kreatur{Irhiadhzal}{Der unbarmherzige Verfolger; ein gehörnter Diener Blakharaz}{gfx/kreaturen/daemon}{\kreaturkampfwerte{12}{16}{10}{8}\trennlinie \kreaturvorteile{Besessenheit von Tieren}\trennlinie \kreaturwaffe{Ausweichen}{0}{14}{}{}{}\trennlinie \kreaturinfo{AsP}{64}\kreaturinfo{Dämonisch}{16 (Alpgestalt, Halluzination, Skelettarius, Totes handle, Traumgestalt)}\trennlinie \kreaturinfo{Beschwörung}{Invocatio}\kreaturinfo{Dienste}{Opfer mit Alpträumen und Halluzinationen in den Wahnsinn treiben+4 (bis von meisterlichem Seelenheiler geheilt), Opfer mit kontrolliertem Tier suchen und töten+0 (1 Tag), Untote erheben und Opfer töten{-}4 (1 Tag)}\trennlinie \kreaturinfo{Info}{Der Dämon fährt in ein Tier ein. Er verliert WS, GS, Ausweichen und erhält die Eigenschaften und Attacken des Tiers. Der Dämon kontrolliert das Tier vollständig, das Tier erhält die Vorteile Regeneration I, Schreckgestalt I, Tarnung und Zusätzliche Attacke I. Das Tier erleidet während der Besessenheit kein Erschöpfung durch körperliche Anstrengung.\newline%
}\trennlinie \kreaturinfo{Quelle}{\href{https://dsaforum.de/viewtopic.php?p=1887303\#p1887303}{Pandämonium}}}}
\newcommand{\kreaturdetailirrlicht}{\kreatur{Irrlicht}{ein Licht im Nebel}{gfx/kreaturen/geist}{\kreaturkampfwerte{1}{10}{8}{8}\trennlinie \kreaturinfo{Magieabweisend}{Zauber wirken auf dich deutlich schwächer. Du ignorierst bei allen Zaubern eine Stufe der spontanen Modifikation Mächtige Magie. Zauber ohne Mächtige Magie haben auf dich keine Wirkung. Voraussetzung: 40 EP Nachkauf: extrem selten\newline%
}\kreaturvorteile{keine Schreckgestalt}\trennlinie \kreaturattribute{GE 12, IN 8}\kreaturfertigkeiten{Laufen 16, Überreden 10}\trennlinie \kreaturinfo{AsP}{30}\kreaturinfo{Einfluss}{12 (Bannbaladin)}\trennlinie \kreaturinfo{Info}{Varianten:  Nützliches Hasserfülltes Irrlicht (Einfluss 14 (Bannbaladin, Horriphobus), Aura (Folge mir!, Willenskraft (20) jede Initiativephase, 1 Erschöpfung))}\trennlinie \kreaturinfo{Quelle}{\href{https://www.orkenspalter.de/filebase/index.php?file/2829-hilberts-bestiarium/}{Hilberts Bestarium}}}}
\newcommand{\kreaturdetailisphanil}{\kreatur{Isphanil}{Die unsichtbare Bringerin dunkler Gaben; Nutznießerin finsterer Gelüste; dreigehörnte Dienerin der Shaz-Man-Yat}{gfx/kreaturen/daemon}{\kreaturkampfwerte{6}{10}{18}{12}\trennlinie \kreaturinfo{Unsichtbarkeit}{kann sich nach Wunsch nur bestimmten Personen, meist dem Opfer, zeigen}\kreaturvorteile{Ausweichen in den Limbus, Limbusreisender}\trennlinie \kreaturwaffe{Ausweichen}{0}{10}{}{}{}\trennlinie \kreaturfertigkeiten{Untertauchen 20, Wachsamkeit 24}\trennlinie \kreaturinfo{AsP}{32}\kreaturinfo{Eigenschaften}{18 (Große Verwirrung)}\trennlinie \kreaturinfo{Beschwörung}{Invocatio}\kreaturinfo{Dienste}{Opfer ausspionieren+4 (1 Woche), Körperteil für ein Ritual vom Opfer beschaffen+0 (1 Tag)}\trennlinie \kreaturinfo{Quelle}{\href{https://dsaforum.de/viewtopic.php?p=1887303\#p1887303}{Pandämonium}}}}
\newcommand{\kreaturdetailivash}{\kreatur{Ivash}{Der Feuerteufel; niederer Diener des Namenlosen Gottes}{gfx/kreaturen/daemon}{\kreaturkampfwerte{5}{13}{10}{10}\trennlinie \kreaturinfo{Flammenkörper}{Alle Waffen, die Holz beinhalten, gelten gegen den Ivash als zerbrechlich}\kreaturvorteile{Formlosigkeit,, Immunität (Feuer),, Resistenz I (geweiht),, Verwundbarkeit I (Wasser)}\trennlinie \kreaturwaffe{Flammenzunge}{2}{12}{12}{2W6+3}{Nachbrennen}\trennlinie \kreaturfertigkeiten{Pirschen 4, Wachsamkeit 10}\trennlinie \kreaturinfo{AsP}{32}\kreaturinfo{Dämonisch}{12 (Brenne!, Ignifaxius)}\trennlinie \kreaturinfo{Beschwörung}{Invocatio}\kreaturinfo{Dienste}{Kampf+4 (1 Minute), Ort bewachen+0 (1 Woche), Schutz des Beschwörers{-}4 (1 Stunde)}\trennlinie \kreaturinfo{Quelle}{\href{https://dsaforum.de/viewtopic.php?p=1887303\#p1887303}{Pandämonium}}}}
\newcommand{\kreaturdetailjegvilbaresidde}{\kreatur{Jegvilbar Esidde}{die ewige Flamme}{gfx/kreaturen/elementar}{\kreaturkampfwerte{8}{16}{1}{8}\trennlinie \kreaturvorteile{Regeneration IV}\trennlinie \kreaturattribute{CH 4, FF 4, GE 4, IN 14, KK 20, KL 10, KO 20, MU 20}\kreaturfertigkeiten{Willenskraft 20, Einschüchtern 2, Wahrnehmung 8}\trennlinie \kreaturinfo{Unbeweglich}{In der dritten Sphäre kann sich die ewige Flamme fast nicht bewegen. Sie verbleibt am liebsten an einem Ort. Dienste die Bewegung enthalten, erschweren die Beherrschungsprobe um 8.}\kreaturinfo{Flammenwand}{Alle 4 INI{-}Ph. kann Jegvilbar Esidde mittels einer Aktion Konzentration eine neue Flammenwand (and aus Flammen, 2xMM) erschaffen. Jegvilbar Esidde kann in einer Aktion Konzentration eine gewirkte Flammenwand in eine beliebige Richtung abfeuern. Die Wand bewegt 32 Schritt in die angesagte Richtung und löst sich dann auf. Ziele in ihrem Pfad müssen eine Gegenprobe GE (30,I) ablegen, die um ihre GS erleichtert ist. Getroffene Ziele erhalten 8W6 SP und erleiden Nachbrennen.}\trennlinie \kreaturinfo{Quelle}{\href{https://www.orkenspalter.de/filebase/index.php?file/2829-hilberts-bestiarium/}{Hilberts Bestarium}}}}
\newcommand{\kreaturdetailkahthurakarfai}{\kreatur{Kah{-}Thurak{-}Arfai}{Der Nachtdämon; dreigehörnter und dennoch überaus mächtiger Diener Agrimoths, großer Gegner}{gfx/kreaturen/daemon}{\kreaturkampfwerte{13}{24}{5}{16}\trennlinie \kreaturinfo{Nachtdämon}{Untertags ist der Dämon wesentlich schwächer und versteckt sich üblicherweise in einer Blume: WS 13 koloss: 1, MR 12, INI 8 und er verliert Regeneration I, Schreckgestalt II, Tarnung)\newline%
}\kreaturvorteile{Flugfähig, Regeneration I, Schreckgestalt II, Tarnung, Zusätzliche Attacke I}\trennlinie \kreaturwaffe{Prankenhieb}{1}{10}{18}{3W6+3}{Wendig}\kreaturwaffe{Biss}{0}{4}{16}{4W6+2}{}\kreaturwaffe{Schwanz}{1}{4}{16}{3W6+0}{Niederwerfen}\kreaturkampfvorteile{Niederwerfen, Sturmangriff}\trennlinie \kreaturfertigkeiten{Pirschen 20, Wachsamkeit 24}\trennlinie \kreaturinfo{AsP}{32}\kreaturinfo{Verwandlung}{14 (Visibili)}\trennlinie \kreaturinfo{Beschwörung}{Invocatio}\kreaturinfo{Dienste}{Kampf+4 (1 Minute), Wache+0 (1 Woche), Nächtliche Spionage+0 (1 Nacht), Suchen und Töten einer Person{-}4 (1 Nacht)}\trennlinie \kreaturinfo{Quelle}{\href{https://dsaforum.de/viewtopic.php?p=1887303\#p1887303}{Pandämonium}}}}
\newcommand{\kreaturdetailkampfhund}{\kreatur{Kampfhund}{verbreitetes Rudeltier}{gfx/kreaturen/tier}{\kreaturkampfwerte{5}{3}{7}{3}\trennlinie \kreaturinfo{Angepasst II (Dunkelheit)}{Angepasst I/II (Umgebung) Durch deine Spezies oder langjährige Erfahrung hast du dich an eine bestimmte Umgebung oder Umweltbedingung gewöhnt. Abzüge durch diese Umgebung (Beispiele auf S. 38), insbesondere im Kampf, sinken für dich um eine/zwei Stufen. Die Kosten für Angepasst legt der Spielleiter fest, wobei er sich an der Häufigkeit der Umgebung orientieren sollte. Zu allgemein gefasste Umgebungen wie „unsicherer Untergrund“ sollte er nicht zulassen. Beispiele für Angepasst sind: • Dunkelheit: verringert Abzüge durch schlechte Lichtverhältnisse (40 EP pro Stufe) • Schnee: verringert Abzüge durch schneebedeckten oder eisigen Untergrund (20 EP pro Stufe) • Wasser: verringert Abzüge durch knie{-} oder hüfttiefes Wasser und unter Wasser (20 EP pro Stufe) • Wald: verringert Abzüge durch Wurzeln, Gestrüpp und dichtes Unterholz (40 EP pro Stufe) Voraussetzungen: keine/Angepasst I Nachkauf: häufig/selten\newline%
}\kreaturvorteile{}\trennlinie \kreaturwaffe{Biss}{0}{6}{6}{2W6+0}{Zerbrechlich}\kreaturkampfvorteile{Niederwerfen, Standfest, Sturmangriff}\trennlinie \kreaturattribute{GE 14, KK 8, KO 8, MU 6}\kreaturfertigkeiten{Laufen 20, Pirschen 12, Wachsamkeit 18, Zähigkeit 4}\trennlinie \kreaturinfo{Quelle}{\href{https://ilarisblog.wordpress.com/downloads/}{Ilaris Regeln}}}}
\newcommand{\kreaturdetailkarmoth}{\kreatur{Karmoth}{Der Vernichter, der unaufhaltsame Bulle, das blutsaufende Rind; sechsgehörnter Diener Belhalhars, sehr großer Gegner}{gfx/kreaturen/daemon}{\kreaturkampfwerte{20}{20}{9}{6}\trennlinie \kreaturinfo{Aura}{Kampfeslust, Willenskraft (28), jede INI: phase, Opfer wird zum Kampf mit Karmoth gezwungen}\kreaturvorteile{Regeneration II, Schreckgestalt III, Unbeugsamkeit, Zusätzliche Attacke II}\trennlinie \kreaturwaffe{Axt}{5}{12}{20}{3W20+10}{Niederwerfen ({-}16), Zurückstoßen}\kreaturwaffe{Trampeln}{1}{6}{16}{4W20+0}{Flächenangriff (1 Schritt Umkreis), Niederwerfen ({-}16), Überrennen (wird nicht durch erfolgreiche VT: gestoppt}\kreaturkampfvorteile{Befreiungsschlag, Gegenhalten, Niederwerfen, Offensiver Kampfstil, Sturmangriff, Unaufhaltsam}\trennlinie \kreaturfertigkeiten{Einschüchtern 32, Wachsamkeit 32}\trennlinie \kreaturinfo{Quelle}{\href{https://dsaforum.de/viewtopic.php?p=1887303\#p1887303}{Pandämonium}}}}
\newcommand{\kreaturdetailkarunga}{\kreatur{Karunga}{Der Grünwisch; ein minderer Diener Amazeroths, kleiner Gegner}{gfx/kreaturen/daemon}{\kreaturkampfwerte{2}{6}{}{15}\trennlinie \kreaturvorteile{Flieger, Immunität (Schadenszauber), Rudel}\trennlinie \kreaturwaffe{Ausweichen}{0}{2}{}{}{}\trennlinie \kreaturinfo{AsP}{24}\kreaturinfo{Dämonisch}{10 (Auris Nasus, Blitz, Menetekel)}\trennlinie \kreaturinfo{Beschwörung}{Invocatio}\kreaturinfo{Dienste}{Gegner verwirren+4 (1 Minute), Nachricht überbringen+0, Trugbilder erzeugen{-}4 (1 Stunde)}\trennlinie \kreaturinfo{Quelle}{\href{https://dsaforum.de/viewtopic.php?p=1887303\#p1887303}{Pandämonium}}}}
\newcommand{\kreaturdetailkharzoreel}{\kreatur{Kharz'oreel}{Der Eishauer und niederhöllische Alchimist; eingehörnter Diener Belshirashs}{gfx/kreaturen/daemon}{\kreaturkampfwerte{12}{14}{6}{8}\trennlinie \kreaturinfo{Unsichtbarkeit}{kann sich nach Wunsch nur bestimmten Personen, meist dem Opfer, zeigen}\kreaturvorteile{Zusätzliche Attacke I}\trennlinie \kreaturwaffe{Kalter Hauch}{2}{6}{16}{2W6+2}{Erfrieren}\trennlinie \kreaturinfo{AsP}{32}\kreaturinfo{Dämonisch}{12 (Corporfrigo, Metamorpho))}\trennlinie \kreaturinfo{Beschwörung}{Invocatio}\kreaturinfo{Dienste}{Eis formen+4 (1 Tag), Theriak{-}Nadel formen+0 (1 Jahr), Gegner einfrieren+4}\trennlinie \kreaturinfo{Quelle}{\href{https://dsaforum.de/viewtopic.php?p=1887303\#p1887303}{Pandämonium}}}}
\newcommand{\kreaturdetailkhelevathan}{\kreatur{Khelevathan}{Der gehörnte Schänder; ein viergehörnter Diener Belkelels}{gfx/kreaturen/daemon}{\kreaturkampfwerte{12}{20}{4}{8}\trennlinie \kreaturvorteile{Ausweichen in den Limbus, Lebensraub, Tarnung, Zusätzliche Attacke I}\trennlinie \kreaturwaffe{Hornstoß}{0}{8}{10}{3W6+4}{Niederwerfen}\kreaturwaffe{Krallen}{1}{12}{16}{2W6+4}{}\kreaturwaffe{Tritt}{1}{6}{12}{2W6+4}{Niederwerfen}\trennlinie \kreaturfertigkeiten{Betören 20, Pirschen 16, Untertauchen 16, Wachsamkeit 14}\trennlinie \kreaturinfo{AsP}{64}\kreaturinfo{Dämonisch}{16 (Böser Blick, Levthans Feuer, Große Gier, Große Verwirrung, Halluzination, Höllenpein, Satuarias Herrlichkeit, Zunge lähmen)}\trennlinie \kreaturinfo{Beschwörung}{Invocatio}\kreaturinfo{Dienste}{Jemanden in einen Mannwidder verwandeln+4, Ziel verführen und Regeneration rauben+0 (1 Woche), Ziel verführen und quälen{-}4 (1 Tag)}\trennlinie \kreaturinfo{Keine}{}\trennlinie \kreaturinfo{Quelle}{\href{https://dsaforum.de/viewtopic.php?p=1887303\#p1887303}{Pandämonium}}}}
\newcommand{\kreaturdetailkhoramsbestie}{\kreatur{Khoramsbestie}{verbreitetes Rudeltier}{gfx/kreaturen/tier}{\kreaturkampfwerte{4}{4}{10}{4}\trennlinie \kreaturinfo{Angepasst II (Dunkelheit)}{Angepasst I/II (Umgebung) Durch deine Spezies oder langjährige Erfahrung hast du dich an eine bestimmte Umgebung oder Umweltbedingung gewöhnt. Abzüge durch diese Umgebung (Beispiele auf S. 38), insbesondere im Kampf, sinken für dich um eine/zwei Stufen. Die Kosten für Angepasst legt der Spielleiter fest, wobei er sich an der Häufigkeit der Umgebung orientieren sollte. Zu allgemein gefasste Umgebungen wie „unsicherer Untergrund“ sollte er nicht zulassen. Beispiele für Angepasst sind: • Dunkelheit: verringert Abzüge durch schlechte Lichtverhältnisse (40 EP pro Stufe) • Schnee: verringert Abzüge durch schneebedeckten oder eisigen Untergrund (20 EP pro Stufe) • Wasser: verringert Abzüge durch knie{-} oder hüfttiefes Wasser und unter Wasser (20 EP pro Stufe) • Wald: verringert Abzüge durch Wurzeln, Gestrüpp und dichtes Unterholz (40 EP pro Stufe) Voraussetzungen: keine/Angepasst I Nachkauf: häufig/selten\newline%
}\kreaturvorteile{}\trennlinie \kreaturwaffe{Biss}{0}{6}{6}{2W6+0}{Zerbrechlich}\kreaturkampfvorteile{Niederwerfen, Standfest, Sturmangriff}\trennlinie \kreaturattribute{GE 14, KK 8, KO 8, MU 6}\kreaturfertigkeiten{Laufen 24, Pirschen 12, Wachsamkeit 18, Zähigkeit 4}\trennlinie \kreaturinfo{Quelle}{\href{https://ilarisblog.wordpress.com/downloads/}{Ilaris Regeln}}}}
\newcommand{\kreaturdetailkhuralthu}{\kreatur{Khuralthu}{Der Blutdrescher; ein niederer Diener Belzorashs}{gfx/kreaturen/daemon}{\kreaturkampfwerte{6}{12}{6}{10}\trennlinie \kreaturvorteile{Regeneration I, Zusätzliche Attacke II}\trennlinie \kreaturwaffe{Hieb}{1}{10}{12}{2W6+1}{Wendig}\kreaturwaffe{Biss}{0}{2}{10}{3W6+1}{Erfrieren (Übelkeit), Infektion (Das Ziel wird mit einer Krankheit bis Stufe 24 infiziert.)}\kreaturkampfvorteile{Niederwerfen, Sturmangriff}\trennlinie \kreaturfertigkeiten{Laufen 10, Pirschen 4, Wachsamkeit 10}\trennlinie \kreaturinfo{AsP}{24}\kreaturinfo{Dämonisch}{10 (Tlalucs Odem)}\trennlinie \kreaturinfo{Beschwörung}{Invocatio}\kreaturinfo{Dienste}{Ritualplatz bewachen+4 (1 Tag), Beschwörer beschützen+0 (1 Tag), Kampf{-}4 (1 Minute)}\trennlinie \kreaturinfo{Quelle}{\href{https://dsaforum.de/viewtopic.php?p=1887303\#p1887303}{Pandämonium}}}}
\newcommand{\kreaturdetailkrakenmolch}{\kreatur{Krakenmolch}{vielarmiger Schrecken der Küsten und Sümpfe; großer Gegner}{gfx/kreaturen/tier}{\kreaturkampfwerte{{-}1}{16}{2}{2}\trennlinie \kreaturinfo{Angepasst II (Wasser)}{Angepasst I/II (Umgebung) Durch deine Spezies oder langjährige Erfahrung hast du dich an eine bestimmte Umgebung oder Umweltbedingung gewöhnt. Abzüge durch diese Umgebung (Beispiele auf S. 38), insbesondere im Kampf, sinken für dich um eine/zwei Stufen. Die Kosten für Angepasst legt der Spielleiter fest, wobei er sich an der Häufigkeit der Umgebung orientieren sollte. Zu allgemein gefasste Umgebungen wie „unsicherer Untergrund“ sollte er nicht zulassen. Beispiele für Angepasst sind: • Dunkelheit: verringert Abzüge durch schlechte Lichtverhältnisse (40 EP pro Stufe) • Schnee: verringert Abzüge durch schneebedeckten oder eisigen Untergrund (20 EP pro Stufe) • Wasser: verringert Abzüge durch knie{-} oder hüfttiefes Wasser und unter Wasser (20 EP pro Stufe) • Wald: verringert Abzüge durch Wurzeln, Gestrüpp und dichtes Unterholz (40 EP pro Stufe) Voraussetzungen: keine/Angepasst I Nachkauf: häufig/selten\newline%
}\kreaturvorteile{Amphibisch, Schreckgestalt I}\trennlinie \kreaturwaffe{Biss}{1}{4}{12}{4W6+4}{}\kreaturwaffe{Tentakelhieb}{8}{10}{14}{2W6+4}{Niederwerfen}\kreaturwaffe{Tentakelklammer}{8}{10}{13}{2W6+0}{Umklammern ({-}4, 22)}\kreaturkampfvorteile{Zusätzliche Attacke II}\trennlinie \kreaturattribute{KK 12, KO 100, MU 20}\kreaturfertigkeiten{Pirschen 8, Schwimmen 14, Wachsamkeit 10}\trennlinie \kreaturinfo{Quelle}{\href{https://ilarisblog.wordpress.com/downloads/}{Ilaris Regeln}}}}
\newcommand{\kreaturdetailkriegermumie}{\kreatur{Kriegermumie}{starker Wächter und letzter Anblick für zahlreiche Grabräuber}{gfx/kreaturen/untot}{\kreaturkampfwerte{10}{12}{4}{5}\trennlinie \kreaturinfo{Astralsinn}{Astralsinn erlaubt es der Kreatur, ihre Umgebung magisch wahrzunehmen. Sie erleidet keine Abzüge durch schlechte Sicht. Der Astralsinn kann durch Antimagie, zum Beispiel Hellsicht trüben in der Modifikation Magie unterdrücken, gestört werden. Die Schwierigkeit dafür liegt mindestens bei 20, bei mächtigen Wesen deutlich höher.\newline%
}\kreaturvorteile{Körperlosigkeit, Resistenz I (Stichwaffen), Verwundbarkeit II (Feuer), Schmerzimmun II, Schreckgestalt II}\trennlinie \kreaturwaffe{Khunchomer}{1}{10}{16}{5W6+4}{Niederwerfen, Unaufhaltsam}\kreaturwaffe{Hände}{0}{10}{14}{2W6+6}{Stumpf, Ansteckend (Jede erlittene Wunde erhöht die Chance einer Ansteckung mit Wundbrand um 25\%.  Eine Probe auf Gifte und Krankheiten (20) eliminiert das Risiko.)}\kreaturkampfvorteile{Kraftvoller Kampf III, Hammerschlag, Niederwerfen}\trennlinie \kreaturattribute{GE 12, KK 32}\kreaturfertigkeiten{Laufen 8, Untertauchen 6, Wachsamkeit 8}\trennlinie \kreaturinfo{Beschwörung}{Skelettarius, Totes handle!}\kreaturinfo{Dienste}{Kampf+0 (1 Minute), Wache{-}4 (1 Woche (mit Totes handle! permanent)')}\trennlinie \kreaturinfo{Quelle}{\href{https://ilarisblog.wordpress.com/downloads/}{Ilaris Regeln}}}}
\newcommand{\kreaturdetailkriegselefant}{\kreatur{Kriegselefant}{gutmütiger Pflanzenfresser; sehr großer Gegner}{gfx/kreaturen/tier}{\kreaturkampfwerte{15}{4}{9}{1}\trennlinie \kreaturwaffe{Stoßzähne}{1}{10}{12}{5W6+4}{Niederwerfen}\kreaturwaffe{Trampeln}{1}{8}{10}{3W20+0}{Flächenangriff (1 Schritt Umkreis), Niederwerfen ({-}8)}\kreaturkampfvorteile{Standfest, Überrennen, Unaufhaltsam, Zerstörerisch I, Zerstörerisch II}\trennlinie \kreaturattribute{KK 120, KO 140, MU 10}\kreaturfertigkeiten{Laufen 14, Wachsamkeit 12, Zähigkeit 14}\trennlinie \kreaturinfo{Quelle}{\href{https://ilarisblog.wordpress.com/downloads/}{Ilaris Regeln}}}}
\newcommand{\kreaturdetailkriegspferd}{\kreatur{Kriegspferd}{kampferprobtes Reittier aus den Ställen Elenvinas; großer Gegner}{gfx/kreaturen/tier}{\kreaturkampfwerte{15}{6}{10}{1}\trennlinie \kreaturwaffe{Tritt}{1}{5}{12}{2W6+2}{Niederwerfen}\kreaturwaffe{Biss}{0}{2}{12}{1W6+3}{Zerbrechlich}\kreaturkampfvorteile{Rüstungsgewöhnung, Standfest, im Reiterkampf}\trennlinie \kreaturattribute{GE 8, KK 26, KO 24, MU 12}\kreaturfertigkeiten{Wachsamkeit 10, Zähigkeit 10}\trennlinie \kreaturinfo{Info}{Im Reiterkampf sind hauptsächlich WS:, GS: und TP:  bedeutend, da AT: und VT: auf den PW Reiten abgelegt werden.}\trennlinie \kreaturinfo{Quelle}{\href{https://ilarisblog.wordpress.com/downloads/}{Ilaris Regeln}}}}
\newcommand{\kreaturdetailkriegspferdleichteruestung}{\kreatur{Kriegspferd mit leichter Rüstung}{kampferprobtes Reittier aus den Ställen Elenvinas; großer Gegner}{gfx/kreaturen/tier}{\kreaturkampfwerte{15}{6}{10}{1}\trennlinie \kreaturwaffe{Tritt}{1}{5}{12}{2W6+2}{Niederwerfen}\kreaturwaffe{Biss}{0}{2}{12}{1W6+3}{Zerbrechlich}\kreaturkampfvorteile{Rüstungsgewöhnung, Standfest, im Reiterkampf}\trennlinie \kreaturattribute{GE 8, KK 26, KO 24, MU 12}\kreaturfertigkeiten{Wachsamkeit 10, Zähigkeit 10}\trennlinie \kreaturinfo{Info}{Im Reiterkampf sind hauptsächlich WS:, GS: und TP:  bedeutend, da AT: und VT: auf den PW Reiten abgelegt werden.}\trennlinie \kreaturinfo{Quelle}{\href{https://ilarisblog.wordpress.com/downloads/}{Ilaris Regeln}}}}
\newcommand{\kreaturdetailkriegspferdmittlereruestung}{\kreatur{Kriegspferd mit mittlerer Rüstung}{kampferprobtes Reittier aus den Ställen Elenvinas; großer Gegner}{gfx/kreaturen/tier}{\kreaturkampfwerte{15}{6}{9}{1}\trennlinie \kreaturwaffe{Tritt}{1}{4}{11}{2W6+2}{Niederwerfen}\kreaturwaffe{Biss}{0}{1}{11}{1W6+3}{Zerbrechlich}\kreaturkampfvorteile{Rüstungsgewöhnung, Standfest, im Reiterkampf}\trennlinie \kreaturattribute{GE 8, KK 26, KO 24, MU 12}\kreaturfertigkeiten{Wachsamkeit 10, Zähigkeit 10}\trennlinie \kreaturinfo{Info}{Im Reiterkampf sind hauptsächlich WS:, GS: und TP:  bedeutend, da AT: und VT: auf den PW Reiten abgelegt werden.}\trennlinie \kreaturinfo{Quelle}{\href{https://ilarisblog.wordpress.com/downloads/}{Ilaris Regeln}}}}
\newcommand{\kreaturdetailkriegswildschwein}{\kreatur{Kriegswildschwein}{gefräßiger Schädling und goblinisches Reittier}{gfx/kreaturen/tier}{\kreaturkampfwerte{11}{3}{7}{3}\trennlinie \kreaturwaffe{Stoß}{0}{4}{11}{2W6+6}{Niederwerfen}\kreaturkampfvorteile{Standfest, Sturmangriff, im Reiterkampf}\trennlinie \kreaturattribute{GE 8, KK 18, KO 22}\kreaturfertigkeiten{Laufen 12, Pirschen 4, Wachsamkeit 10, Zähigkeit 4}\trennlinie \kreaturinfo{Quelle}{\href{https://ilarisblog.wordpress.com/downloads/}{Ilaris Regeln}}}}
\newcommand{\kreaturdetaillederschwinge}{\kreatur{Lederschwinge}{Mensch und Fledermaus}{gfx/kreaturen/daimonid}{\kreaturkampfwerte{6}{5}{8}{4}\trennlinie \kreaturvorteile{Flieger}\trennlinie \kreaturwaffe{Krallen}{0}{7}{7}{2W6+1}{Wendig}\kreaturkampfvorteile{Niederwerfen, Sturmangriff}\trennlinie \kreaturattribute{CH 4, FF 6, GE 8, IN 8, KK 8, KL {-}12, KO 8, MU 10}\kreaturfertigkeiten{Klettern 2, Akrobatik 4, Pirschen 4, Fliegen 7, Willenskraft 4, Zähigkeit 4, Einschüchtern 7, Wachsamkeit 6}\trennlinie \kreaturinfo{Info}{In die Luft: Gelingt der Lederschwinge eine Umklammerung. kann sie ihr Opfer jetzt und in jeder folgenden INI{-}Ph. 1W3 Schritt nach oben reißen. Löst sich das Opfer aus der Umklammerung, erhält es unter Umständen Fallschaden.}\trennlinie \kreaturinfo{Quelle}{\href{https://www.orkenspalter.de/filebase/index.php?file/2829-hilberts-bestiarium/}{Hilberts Bestarium}}}}
\newcommand{\kreaturdetailleibwaechter}{\kreatur{Leibwächter}{persönlicher Wächter eines Würdenträgers}{gfx/kreaturen/humanoid}{\kreaturkampfwerte{4}{5}{6}{12}\trennlinie \kreaturwaffe{Amazonensäbel}{1}{16}{13}{2W6+3}{Wendig}\kreaturwaffe{Holzschild}{0}{18}{15}{1W6+2}{Schild, Stumpf}\kreaturwaffe{Faust}{0}{14}{14}{1W6+4}{Kopflastig, Stumpf, Wendig, Zerbrechlich}\kreaturkampfvorteile{Schildkampf III, Aufmerksamkeit, Ausfall, Defensiver Kampfstil, Kampfreflexe, Niederwerfen, Standfest, Sturmangriff, Waffenloser Kampf}\trennlinie \kreaturattribute{CH 6, FF 10, GE 12, IN 16, KK 16, KL 6, KO 10, MU 8}\kreaturfertigkeiten{Einhandklingenwaffen 14, Gebräuche 7, Menschenkenntnis 15, Schilde 15, Wachsamkeit 15, Waffenlos 15, Zähigkeit 12}\kreaturinfo{Profane Vorteile}{Flink I, Flink II, Vorausschauend I, Vorausschauend II}\trennlinie \kreaturinfo{Quelle}{\href{https://ilarisblog.wordpress.com/downloads/}{Ilaris Regeln}}}}
\newcommand{\kreaturdetaillibaal}{\kreatur{LI'BAAL}{Die Fleischformerin der inneren Einöden; eine sechsgehörnter Dienerin Asfaloths, großer Gegner}{gfx/kreaturen/daemon}{\kreaturkampfwerte{10}{18}{4}{4}\trennlinie \kreaturvorteile{Schreckgestalt II, Regeneration I, Zusätzliche Attacke IV}\trennlinie \kreaturwaffe{Zange}{1}{8}{16}{3W6+4}{Niederwerfen ({-}4), Rüstungsbrechend}\kreaturwaffe{Biss}{0}{4}{18}{4W6+4}{Mutation, Rüstungsbrechend}\trennlinie \kreaturfertigkeiten{Pirschen 14, Wachsamkeit 20}\trennlinie \kreaturinfo{AsP}{128}\kreaturinfo{Dämonisch}{24 (Chimaeroform, Krabbelnder Schrecken, Pandämonium, Salander)}\trennlinie \kreaturinfo{Beschwörung}{Invocatio}\kreaturinfo{Dienste}{6 nützliche Chimären gebären+4, 3 starke Chimären gebären+0, 1 mächtige Chimäre gebären{-}4}\trennlinie \kreaturinfo{Quelle}{\href{https://dsaforum.de/viewtopic.php?p=1887303\#p1887303}{Pandämonium}}}}
\newcommand{\kreaturdetailloewe}{\kreatur{Löwe}{König der Dschungel und nördlichen Steppen; großer Gegner}{gfx/kreaturen/tier}{\kreaturkampfwerte{10}{6}{11}{10}\trennlinie \kreaturinfo{Angepasst II (Dunkelheit)}{Angepasst I/II (Umgebung) Durch deine Spezies oder langjährige Erfahrung hast du dich an eine bestimmte Umgebung oder Umweltbedingung gewöhnt. Abzüge durch diese Umgebung (Beispiele auf S. 38), insbesondere im Kampf, sinken für dich um eine/zwei Stufen. Die Kosten für Angepasst legt der Spielleiter fest, wobei er sich an der Häufigkeit der Umgebung orientieren sollte. Zu allgemein gefasste Umgebungen wie „unsicherer Untergrund“ sollte er nicht zulassen. Beispiele für Angepasst sind: • Dunkelheit: verringert Abzüge durch schlechte Lichtverhältnisse (40 EP pro Stufe) • Schnee: verringert Abzüge durch schneebedeckten oder eisigen Untergrund (20 EP pro Stufe) • Wasser: verringert Abzüge durch knie{-} oder hüfttiefes Wasser und unter Wasser (20 EP pro Stufe) • Wald: verringert Abzüge durch Wurzeln, Gestrüpp und dichtes Unterholz (40 EP pro Stufe) Voraussetzungen: keine/Angepasst I Nachkauf: häufig/selten\newline%
}\kreaturinfo{Blitzschnell}{erleidet keine Passierschläge wenn er sich aus dem Nahkampf zurückzieht}\kreaturvorteile{}\trennlinie \kreaturwaffe{Prankenhieb}{1}{12}{16}{2W6+0}{Doppelangriff, Niederwerfen ({-}4)}\kreaturwaffe{Biss}{0}{2}{16}{5W6{-}2}{Zerbrechlich}\kreaturkampfvorteile{Standfest, Sturmangriff}\trennlinie \kreaturattribute{GE 16, KK 16, KO 18, MU 14}\kreaturfertigkeiten{Laufen 16, Pirschen 18, Wachsamkeit 18, Zähigkeit 16}\trennlinie \kreaturinfo{Quelle}{\href{https://ilarisblog.wordpress.com/downloads/}{Ilaris Regeln}}}}
\newcommand{\kreaturdetaillynx}{\kreatur{Lynx}{Grüßer der Todgeweihten}{gfx/kreaturen/geist}{\kreaturkampfwerte{6}{10}{8}{8}\trennlinie \kreaturinfo{Unsichtbarkeit}{Ausnahme: Ziel der Jagd}\kreaturvorteile{Schreckgestalt II}\trennlinie \kreaturwaffe{Pranken}{0}{8}{16}{1W6+6}{Rüstungsbrechend}\kreaturwaffe{Biss}{0}{8}{14}{3W6+3}{Rüstungsbrechend}\kreaturkampfvorteile{Ausfall, Offensiver Kampfstil, Niederwerfen, Standfest, Sturmangriff, Todesstoß, Zusätzliche Attacke I}\trennlinie \kreaturattribute{CH 4, FF 4, GE 16, IN 6, KK 12, KL 10, KO 10, MU 10}\kreaturfertigkeiten{Einschüchtern 10, Laufen 12, Sinnesschärfe 8, Pirschen 10, Willenskraft 8}\trennlinie \kreaturinfo{Quelle}{\href{https://www.orkenspalter.de/filebase/index.php?file/2829-hilberts-bestiarium/}{Hilberts Bestarium}}}}
\newcommand{\kreaturdetailmactans}{\kreatur{Mactans}{Die gepanzerte Spinne, das weitreichende Netz; ein fünfgehörnter Dämon, großer Gegner}{gfx/kreaturen/daemon}{\kreaturkampfwerte{10}{18}{6}{16}\trennlinie \kreaturinfo{Verwundbarkeit}{Seeschlangenzähne}\kreaturvorteile{Regeneration II, Schreckgestalt II, Zusätzliche Attacke IV}\trennlinie \kreaturwaffe{Tentakel}{2}{8}{14}{3W6+2}{}\kreaturwaffe{Schnabel}{0}{2}{16}{4W6+4}{}\trennlinie \kreaturinfo{AsP}{64}\kreaturinfo{Dämonisch}{16 (Arachnea, Auge des Limbus, Granit und Marmor, Große Verwirrung, Paralysis))}\trennlinie \kreaturinfo{Beschwörung}{Invocatio}\kreaturinfo{Dienste}{Wache+4 (1 Jahr), Kampf+0 (1 Minute), Gegner versteinern oder in den Limbus schleudern{-}4}\trennlinie \kreaturinfo{Quelle}{\href{https://dsaforum.de/viewtopic.php?p=1887303\#p1887303}{Pandämonium}}}}
\newcommand{\kreaturdetailmamut}{\kreatur{Mamut}{gutmütiger Pflanzenfresser; sehr großer Gegner}{gfx/kreaturen/tier}{\kreaturkampfwerte{17}{4}{9}{1}\trennlinie \kreaturwaffe{Stoßzähne}{1}{6}{8}{5W6+9}{Niederwerfen}\kreaturwaffe{Trampeln}{1}{4}{6}{3W20+5}{Flächenangriff (1 Schritt Umkreis), Niederwerfen ({-}8)}\kreaturkampfvorteile{Standfest, Überrennen, Unaufhaltsam, Zerstörerisch I, Zerstörerisch II, Resistenz gegen Kälte}\trennlinie \kreaturattribute{KK 160, KO 200, MU 10}\kreaturfertigkeiten{Laufen 14, Wachsamkeit 12, Zähigkeit 14}\trennlinie \kreaturinfo{Quelle}{\href{https://ilarisblog.wordpress.com/downloads/}{Ilaris Regeln}}}}
\newcommand{\kreaturdetailmantikor}{\kreatur{Mantikor}{Mensch, Löwe und Skorpion}{gfx/kreaturen/daimonid}{\kreaturkampfwerte{7}{15}{10}{8}\trennlinie \kreaturinfo{Angepasst I (Dunkelheit)}{Angepasst I/II (Umgebung) Durch deine Spezies oder langjährige Erfahrung hast du dich an eine bestimmte Umgebung oder Umweltbedingung gewöhnt. Abzüge durch diese Umgebung (Beispiele auf S. 38), insbesondere im Kampf, sinken für dich um eine/zwei Stufen. Die Kosten für Angepasst legt der Spielleiter fest, wobei er sich an der Häufigkeit der Umgebung orientieren sollte. Zu allgemein gefasste Umgebungen wie „unsicherer Untergrund“ sollte er nicht zulassen. Beispiele für Angepasst sind: • Dunkelheit: verringert Abzüge durch schlechte Lichtverhältnisse (40 EP pro Stufe) • Schnee: verringert Abzüge durch schneebedeckten oder eisigen Untergrund (20 EP pro Stufe) • Wasser: verringert Abzüge durch knie{-} oder hüfttiefes Wasser und unter Wasser (20 EP pro Stufe) • Wald: verringert Abzüge durch Wurzeln, Gestrüpp und dichtes Unterholz (40 EP pro Stufe) Voraussetzungen: keine/Angepasst I Nachkauf: häufig/selten\newline%
}\kreaturinfo{Sprunghaft}{keine Passierschläge durch Bewegung}\kreaturinfo{Katzenhaft}{+4 PA gegen Gegner, die in dieser INI{-}Ph. nicht Ziel deiner Angriffe waren}\kreaturvorteile{}\trennlinie \kreaturwaffe{Biss}{0}{6}{17}{3W6+6}{}\kreaturwaffe{Prankenhieb}{1}{6}{17}{2W6+8}{Doppelangriff, Niederwerfen ({-}4), Wendig}\kreaturwaffe{Stachel}{2}{6}{17}{2W6+8}{Rüstungsbrechend, Mantikorgift (Stufe 24, Intervall 1, 1 Erschöpfung, Wirkungsdauer 9; 3{-}9x pro Tag)}\kreaturkampfvorteile{Standfest, Sturmangriff, Unaufhaltsam}\trennlinie \kreaturattribute{CH 8, FF 4, GE 10, IN 8, KK 54, KL 10, KO 58, MU 10}\kreaturfertigkeiten{Klettern 4, Laufen 9, Zähigkeit 7, Wachsamkeit 6, Pirschen 8, Willenskraft 7, Einschüchtern 10}\trennlinie \kreaturinfo{Info}{Klugheit 6 bis 14}\trennlinie \kreaturinfo{Quelle}{\href{https://www.orkenspalter.de/filebase/index.php?file/2829-hilberts-bestiarium/}{Hilberts Bestarium}}}}
\newcommand{\kreaturdetailmaraske}{\kreatur{Maraske}{Spinne und Skorpion}{gfx/kreaturen/tier}{\kreaturkampfwerte{2}{10}{11}{5}\trennlinie \kreaturinfo{Immunität}{Niederwerfen, Umreißen und ähnliche Effekte}\kreaturvorteile{}\trennlinie \kreaturwaffe{Waffe}{1}{11}{9}{1W6+6}{Rüstungsbrecher, Giftig (Maraskengift; ILARIS:35)}\kreaturkampfvorteile{Standfest}\trennlinie \kreaturattribute{CH 4, FF 4, GE 10, IN 10, KK 4, KL {-}22, KO 4, MU 2}\kreaturfertigkeiten{Laufen 6, Klettern 10, Schwimmen 2, Akrobatik 16, Willenskraft 7, Zähigkeit 12, Pirschen 10, Einschüchtern 8, Menschenkenntnis 7, Sinnesschärfe 13, Wachsamkeit 13}\trennlinie \kreaturinfo{Info}{Varianten Jungtier (AT, VT {-}4, 1W6 TP)}\trennlinie \kreaturinfo{Quelle}{\href{https://www.orkenspalter.de/filebase/index.php?file/2829-hilberts-bestiarium/}{Hilberts Bestarium}}}}
\newcommand{\kreaturdetailmaru}{\kreatur{Maru}{Große, wütende Kriegerechse}{gfx/kreaturen/humanoid}{\kreaturkampfwerte{6}{5}{3}{8}\trennlinie \kreaturinfo{Angepasst I (Wasser)}{Angepasst I/II (Umgebung) Durch deine Spezies oder langjährige Erfahrung hast du dich an eine bestimmte Umgebung oder Umweltbedingung gewöhnt. Abzüge durch diese Umgebung (Beispiele auf S. 38), insbesondere im Kampf, sinken für dich um eine/zwei Stufen. Die Kosten für Angepasst legt der Spielleiter fest, wobei er sich an der Häufigkeit der Umgebung orientieren sollte. Zu allgemein gefasste Umgebungen wie „unsicherer Untergrund“ sollte er nicht zulassen. Beispiele für Angepasst sind: • Dunkelheit: verringert Abzüge durch schlechte Lichtverhältnisse (40 EP pro Stufe) • Schnee: verringert Abzüge durch schneebedeckten oder eisigen Untergrund (20 EP pro Stufe) • Wasser: verringert Abzüge durch knie{-} oder hüfttiefes Wasser und unter Wasser (20 EP pro Stufe) • Wald: verringert Abzüge durch Wurzeln, Gestrüpp und dichtes Unterholz (40 EP pro Stufe) Voraussetzungen: keine/Angepasst I Nachkauf: häufig/selten\newline%
}\kreaturvorteile{Kältestarre, Natürliche Rüstung}\trennlinie \kreaturwaffe{Echsische Axt}{}{12}{12}{2W6+4}{Rüstungsbrechend, Wendig, Zweihändig}\kreaturwaffe{Säbel}{}{12}{12}{2W6+4}{}\kreaturwaffe{Kurzschwert}{}{12}{12}{2W6+0}{Wendig}\kreaturwaffe{Schwanz}{1}{12}{12}{1W6+2}{Niederwerfen}\kreaturkampfvorteile{Beidhändiger Kampfstil II, Kraftvoller Kampf III, Ausfall, Offensiver Kampfstil, Niederwerfen, Hammerschlag, Eisern, Zäher Hund, Kampfreflexe}\trennlinie \kreaturattribute{CH 4, FF 4, GE 12, IN 8, KK 16, KL 6, KO 16, MU 14}\kreaturfertigkeiten{Einschüchtern 12, Wachsamkeit 7, Zähigkeit 12}\kreaturinfo{Profane Vorteile}{Abgehärtet I, Abgehärtet II, Schnelle Heilung, Muskelprotz, Zerstörerisch II}\trennlinie \kreaturinfo{Quelle}{\href{https://dsaforum.de/viewtopic.php?f=180&p=1738549\#p1738549}{Bestarium+}}}}
\newcommand{\kreaturdetailmarukmethai}{\kreatur{Maruk{-}Methai}{Der Wille zur Macht, die Kraft des Namenlosen, seine abgeschlagene und immer wachsende rechte Hand; fünfgehörnter Diener des Namenlosen
}{gfx/kreaturen/daemon}{\kreaturkampfwerte{2}{6}{2}{4}\trennlinie \kreaturinfo{Unsichtbarkeit}{Ausnahme: Ziel der Jagd}\kreaturvorteile{Ausweichen, Beschwörer stärken}\trennlinie \kreaturwaffe{Ausweichen}{0}{2}{}{}{}\trennlinie \kreaturinfo{Beschwörung}{Invocatio}\kreaturinfo{Dienste}{Beschwörer stärken+0 (1 Woche)}\trennlinie \kreaturinfo{Info}{Beschwörer stärken (Der Dämon fährt in den Beschwörer ein und verleiht diesem übermenschliche Kräfte. Der Dämon verliert (WS, GS, Ausweichen, Unsichtbarkeit) und erhält die Eigenschaften/Attacken des Opfers. Der Beschwörer erhält folgende Bonusse: Proben auf körperliche Attribute sind um +8, Fertigkeitsproben mit körperlichen Attributen um +4 erleichtert, MR +8, INI +8, WS +2, Resistenz I (profan), Schmerzimmun II, Kalte Wut, Proben auf die übernatürliche Fertigkeit Einfluss sind um +4 erleichtert. Der Beschwörer erleidet 2W6 SP pro Tag, zudem versucht Maruk{-} Methai gänzlich von ihm Besitz zu ergreifen. Dies gelingt ihm wenn bei einem täglich anfallenden Wurf mit W20 18{-}20 gewürfelt wird.)\newline%
}\trennlinie \kreaturinfo{Quelle}{\href{https://dsaforum.de/viewtopic.php?p=1887303\#p1887303}{Pandämonium}}}}
\newcommand{\kreaturdetailmeisterdesfeuers}{\kreatur{Meister des Feuers}{großer, rachsüchtiger Meister}{gfx/kreaturen/elementar}{\kreaturkampfwerte{20}{24}{9}{10}\trennlinie \kreaturwaffe{Flammenhand}{2}{16}{26}{4W20+3}{Befreiungsschlag, Nachbrennen, Niederwerfen ({-}16)}\kreaturkampfvorteile{Unaufhaltsam}\trennlinie \kreaturfertigkeiten{Willenskraft 30, Einschüchtern 30, Menschenkenntnis 7, Sinnesschärfe 14, Wachsamkeit 14}\trennlinie \kreaturinfo{AsP}{128}\kreaturinfo{Feuer}{24 (alle und ähnliche Effekte)}\trennlinie \kreaturinfo{Quelle}{\href{https://www.orkenspalter.de/filebase/index.php?file/2829-hilberts-bestiarium/}{Hilberts Bestarium}}}}
\newcommand{\kreaturdetailmepharasch}{\kreatur{Mepharasch}{der Feuerwirsch}{gfx/kreaturen/elementar}{\kreaturkampfwerte{4}{13}{9}{8}\trennlinie \kreaturwaffe{Feuerbiss}{0}{10}{14}{2W6+2}{Nachbrennen}\kreaturkampfvorteile{Sturmangriff, Zusätzliche Attacke I}\trennlinie \kreaturattribute{CH 8, FF 6, GE 16, IN 16, KK 10, KL 8, KO 10, MU 12}\kreaturfertigkeiten{Laufen 10, Willenskraft 7, Einschüchtern 12, Wachsamkeit 14}\trennlinie \kreaturinfo{AsP}{64}\kreaturinfo{Feuer}{12 (alle bis 60 EP)}\trennlinie \kreaturinfo{Feuerteufel}{Die Modifikation Mehrere Ziele ist nicht erschwert. Für jedes weitere Ziel werden nur die halben Kosten bezahlt.}\kreaturinfo{Schwer zu kontrollieren}{Der Modifikator geht bis {-}16 je nach Dienst. Mepharasch hasst es, Dinge nicht in Brand setzen zu dürfen oder wenn ihm mehr als ein Dienst abverlangt wird, ohne dies vorher anzukündigen.}\trennlinie \kreaturinfo{Quelle}{\href{https://www.orkenspalter.de/filebase/index.php?file/2829-hilberts-bestiarium/}{Hilberts Bestarium}}}}
\newcommand{\kreaturdetailmorcan}{\kreatur{Morcan}{Bote der Nacht des Irrsinns, Sklavenmeister der Seelen; minderer Diener Thargunitoths}{gfx/kreaturen/daemon}{\kreaturkampfwerte{5}{8}{4}{8}\trennlinie \kreaturinfo{Unsichtbarkeit}{Ausnahme: Ziel der Jagd}\kreaturvorteile{Besessenheit}\trennlinie \kreaturwaffe{Ausweichen}{0}{2}{}{}{}\trennlinie \kreaturinfo{AsP}{24}\kreaturinfo{Einfluss}{10 (Traumgestalt)}\trennlinie \kreaturinfo{Beschwörung}{Invocatio}\kreaturinfo{Dienste}{Besessenheit+4 (1 Tag), Alpträume erzeugen+0 (1 Woche), Opfer mit einer Alptraumwelt in den Wahnsinn treiben{-}4 (bis von erfahrenem Seelenheiler geheilt)}\trennlinie \kreaturinfo{Info}{Besessenheit (Der Morcan fährt in sein Opfer ein und kontrolliert dieses in der Nacht, sofern diesem keine Probe auf Willenskraft (20) gelingt. Der Morcan verliert WS, GS, Ausweichen, Unsichtbarkeit und erhält die Eigenschaften/Attacken des Opfers.)\newline%
}\trennlinie \kreaturinfo{Quelle}{\href{https://dsaforum.de/viewtopic.php?p=1887303\#p1887303}{Pandämonium}}}}
\newcommand{\kreaturdetailmorfu}{\kreatur{Morfu}{schleimige Giftschnecke, die ihre Stacheln verschießt; großer Gegner}{gfx/kreaturen/tier}{\kreaturkampfwerte{4}{16}{1}{0}\trennlinie \kreaturvorteile{Immunität (Niederwerfen, Umreißen und ähnliche Effekte), Resistenz I (Stumpfe Waffen)}\trennlinie \kreaturwaffe{Dornensalve}{8}{}{}{2W6+2}{Morfugift (Stufe 20, Verzögerung 4 Aktionen, Wirkungsdauer sofort; das Ziel erleidet eine Wunde)}\trennlinie \kreaturattribute{KK 8, KO 8, MU 10}\kreaturfertigkeiten{Laufen {-}10, Wachsamkeit 8}\trennlinie \kreaturinfo{Quelle}{\href{https://ilarisblog.wordpress.com/downloads/}{Ilaris Regeln}}}}
\newcommand{\kreaturdetailmuwalaaran}{\kreatur{Muwalaaran}{Der schwarze Hengst; zweigehörnter Diener Belkelels, großer Gegner}{gfx/kreaturen/daemon}{\kreaturkampfwerte{15}{20}{10}{10}\trennlinie \kreaturvorteile{Flugfähig, Regeneration I}\trennlinie \kreaturwaffe{Tritt}{1}{6}{14}{3W6+4}{Niederwerfen}\kreaturwaffe{Biss}{0}{2}{12}{2W6+3}{}\kreaturkampfvorteile{Sturmangriff, Überrennen}\trennlinie \kreaturfertigkeiten{Laufen 16, Fliegen 18, Wachsamkeit 10}\trennlinie \kreaturinfo{Beschwörung}{Invocatio}\kreaturinfo{Dienste}{Beschwörer transportieren+4 (1 Stunde), Daimoniden{-}Züchtung+0, Transport großer Lasten{-}4}\trennlinie \kreaturinfo{Quelle}{\href{https://dsaforum.de/viewtopic.php?p=1887303\#p1887303}{Pandämonium}}}}
\newcommand{\kreaturdetailnachtalp}{\kreatur{Nachtalp}{auf der Suche nach Auferstehung}{gfx/kreaturen/geist}{\kreaturkampfwerte{2}{5}{5}{3}\trennlinie \kreaturinfo{Tarnung}{Dunkelheit}\kreaturinfo{Aura}{Albträume, Willenskraft (28) jede Nacht, Schlafende in 4 Schritt Reichweite erleiden eine Wunde}\kreaturvorteile{}\trennlinie \kreaturwaffe{Waffenlos}{0}{6}{10}{2W6+0}{Umklammern ({-}4), Rüstungsbrechend}\kreaturwaffe{Biss}{0}{6}{6}{3W6+0}{Zerbrechlich}\trennlinie \kreaturattribute{CH 10, FF 2, GE 6, IN 4, KK 2, KL 10, KO 4, MU 10}\kreaturfertigkeiten{Einschüchtern 6, Menschenkenntnis 6, Sinnesschärfe 8, Pirschen 10, Willenskraft 12}\trennlinie \kreaturinfo{Info}{Sonderregeln:\newline%
Wiedergeburt: Nachtalpe erhalten +1 WS pro Wunde die sie mit Bissen oder Albträumen angerichtet haben. Wann immer der Nachtalp eine Wunde erleidet oder 24h lang keine neue WS hinzugewinnt, verliert er eine WS (bis zu seinem Ursprungswert). Erreicht ein Nachtalp Koloss II WS 16, so erhält er einen sterblichen Körper. Er ist dann wieder ein Lebewesen und kein Geist mehr. Werte und Aussehen entsprechen dann denen seines ursprünglichen Körpers vor dem Tod.\newline%
Triumph: Triumphe bei einer Biss{-}Attacke richten zusätzlich doppelten Waffenschaden an (siehe Hammerschlag).\newline%
Dienst: Gegen Blut sind freie Nachtalpe in der Regel bereit Dienste zu verrichten, die keinen Kampf enthalten.\newline%
}\trennlinie \kreaturinfo{Quelle}{\href{https://www.orkenspalter.de/filebase/index.php?file/2829-hilberts-bestiarium/}{Hilberts Bestarium}}}}
\newcommand{\kreaturdetailnachtwind}{\kreatur{Nachtwind}{Sehr kleiner, magiehassender, geflügelter Jäger}{gfx/kreaturen/tier}{\kreaturkampfwerte{2}{4}{1}{2}\trennlinie \kreaturinfo{Magiegespür}{In der Nähe astraler Kräfte überfällt dich ein Frösteln, du hörst sphärische Klänge oder ein Farbschleier legt sich für dich über die Umgebung. Mit dem Talent Sinnenschärfe kannst du Intensitätsanalysen (S. 80) von magischen Gegenständen durchführen. Nach dem aktiven Einsatz der Gabe erleidest du einen Punkt Erschöpfung. Voraussetzungen: 60 EP Nachkauf: extrem selten\newline%
}\kreaturinfo{Angepasst II (Dunkelheit)}{Angepasst I/II (Umgebung) Durch deine Spezies oder langjährige Erfahrung hast du dich an eine bestimmte Umgebung oder Umweltbedingung gewöhnt. Abzüge durch diese Umgebung (Beispiele auf S. 38), insbesondere im Kampf, sinken für dich um eine/zwei Stufen. Die Kosten für Angepasst legt der Spielleiter fest, wobei er sich an der Häufigkeit der Umgebung orientieren sollte. Zu allgemein gefasste Umgebungen wie „unsicherer Untergrund“ sollte er nicht zulassen. Beispiele für Angepasst sind: • Dunkelheit: verringert Abzüge durch schlechte Lichtverhältnisse (40 EP pro Stufe) • Schnee: verringert Abzüge durch schneebedeckten oder eisigen Untergrund (20 EP pro Stufe) • Wasser: verringert Abzüge durch knie{-} oder hüfttiefes Wasser und unter Wasser (20 EP pro Stufe) • Wald: verringert Abzüge durch Wurzeln, Gestrüpp und dichtes Unterholz (40 EP pro Stufe) Voraussetzungen: keine/Angepasst I Nachkauf: häufig/selten\newline%
}\kreaturvorteile{Flieger}\trennlinie \kreaturwaffe{Krallen}{0}{6}{13}{1W6+1}{}\kreaturkampfvorteile{Sturmangriff}\trennlinie \kreaturattribute{GE 2, KK {-}6, KO {-}2, MU 4}\kreaturfertigkeiten{Pirschen 16, Wachsamkeit 18, Zähigkeit {-}4}\trennlinie \kreaturinfo{Quelle}{\href{https://dsaforum.de/viewtopic.php?p=1887303\#p1887303}{Allgemeine Gegner}}}}
\newcommand{\kreaturdetailnecrophagion}{\kreatur{Necrophagion}{skelettierter großer Tatzelwurm}{gfx/kreaturen/untot}{\kreaturkampfwerte{8}{8}{3}{2}\trennlinie \kreaturinfo{Astralsinn}{Astralsinn erlaubt es der Kreatur, ihre Umgebung magisch wahrzunehmen. Sie erleidet keine Abzüge durch schlechte Sicht. Der Astralsinn kann durch Antimagie, zum Beispiel Hellsicht trüben in der Modifikation Magie unterdrücken, gestört werden. Die Schwierigkeit dafür liegt mindestens bei 20, bei mächtigen Wesen deutlich höher.\newline%
}\kreaturinfo{Aura}{Gestank, Zähigkeit (16) jede Initiativephase, –2 auf körperliche Proben (kumulativ) für 1 Tag}\kreaturinfo{Resistenz II}{Feuer, Stich}\kreaturvorteile{}\trennlinie \kreaturwaffe{Biss}{1}{4}{15}{3W6+5}{Zerbrechlich}\kreaturwaffe{Klauen}{1}{6}{12}{2W6+4}{Niederwerfen ({-}4)}\kreaturwaffe{Schwanz}{2}{4}{10}{1W6+4}{Niederwerfen, Flächenangriff (180° hinter dem Wurm)}\kreaturkampfvorteile{Zusätzliche Attacke I, Standfest, Unaufhaltsam}\trennlinie \kreaturattribute{KK 26, KO 32}\kreaturfertigkeiten{Pirschen 0, Wachsamkeit 8}\trennlinie \kreaturinfo{Quelle}{\href{https://www.orkenspalter.de/filebase/index.php?file/2829-hilberts-bestiarium/}{Hilberts Bestarium}}}}
\newcommand{\kreaturdetailnemanar}{\kreatur{Nemanar}{Der Schattentod, der Vollstrecker im Dunkeln; ein zweigehörnter Diener Blakharaz}{gfx/kreaturen/daemon}{\kreaturkampfwerte{8}{14}{8}{12}\trennlinie \kreaturvorteile{Lichtscheu, Regeneration I, Tarnung}\trennlinie \kreaturwaffe{Dolch}{0}{12}{16}{2W6+2}{}\kreaturwaffe{Leichte Armbrust}{32}{}{}{3W6+1}{Zweihändig}\kreaturkampfvorteile{Reflexschuss, Ruhige Hand, Schnellschuss, Meisterschuss, Todesstoß}\trennlinie \kreaturfertigkeiten{Pirschen 20, Sinnenschärfe 14, Untertauchen 20, Wachsamkeit 14}\trennlinie \kreaturinfo{AsP}{64}\kreaturinfo{Dämonisch}{14 (Dunkelheit, Falkenauge, Silentium, Spurlos, Umbraporta)}\trennlinie \kreaturinfo{Beschwörung}{Invocatio}\kreaturinfo{Dienste}{Opfer suchen und meucheln+4 (1 Tag), Mehrere Opfer an einem Ort suchen und meucheln{-}4 (1 Tag)}\trennlinie \kreaturinfo{Quelle}{\href{https://dsaforum.de/viewtopic.php?p=1887303\#p1887303}{Pandämonium}}}}
\newcommand{\kreaturdetailnishkakat}{\kreatur{Nishkakat}{Der scheinheilige Diener der dreiköpfigen Echse von Nabuleth; dreigehörnter Diener Amazeroths, kleiner Gegner}{gfx/kreaturen/daemon}{\kreaturkampfwerte{2}{16}{}{10}\trennlinie \kreaturvorteile{Flieger}\trennlinie \kreaturwaffe{Biss}{0}{16}{6}{2W6+0}{}\kreaturkampfvorteile{Sturmangriff}\trennlinie \kreaturfertigkeiten{Dämonenkunde 20, Derekunde 16, Handwerk 16, Magietheorie 16, Mythenkunde 16}\kreaturfertigkeiten{Lehrer 3, etliche Sprachen und Schriften 3}\trennlinie \kreaturinfo{AsP}{32}\kreaturinfo{Verwandlung}{14 (Visibili)}\kreaturinfo{Dämonisch}{}\trennlinie \kreaturinfo{Beschwörung}{Invocatio}\kreaturinfo{Dienste}{Berater für Beschwörungen+4 (1 Tag), Opfer falsches Wissen einflüstern;+0, Lehrmeister für profane Talente, Schriften oder Sprachen{-}4 (1 Woche)}\trennlinie \kreaturinfo{Quelle}{\href{https://dsaforum.de/viewtopic.php?p=1887303\#p1887303}{Pandämonium}}}}
\newcommand{\kreaturdetailoboraddon}{\kreatur{Oboraddon}{Der Herrscher der Nimmer Ruhenden; sechsgehörnter Diener Thargunitoths, großer Gegner}{gfx/kreaturen/daemon}{\kreaturkampfwerte{8}{16}{8}{10}\trennlinie \kreaturinfo{Unsichtbarkeit}{Ausnahme: Ziel der Jagd}\kreaturvorteile{Leiche beseelen}\trennlinie \kreaturwaffe{Ausweichen}{0}{12}{}{}{}\kreaturkampfvorteile{Kommando: Formiert Euch!, Kommando: Keine Gefangenen!}\trennlinie \kreaturfertigkeiten{Anführen 24, Pirschen 20, Wachsamkeit 16}\trennlinie \kreaturinfo{AsP}{128}\kreaturinfo{Dämonisch}{20 (Skelettarius, Totes Handle!)}\trennlinie \kreaturinfo{Beschwörung}{Invocatio}\kreaturinfo{Dienste}{Mehrere schwache Untote erschaffen und kontrollieren+4 (1 Stunde), Mehrere nützliche Untote erschaffen und kontrollieren+0 (1 Stunde), König der Untoten erschaffen{-}4 (1 Woche oder bis der Untote zerfällt)}\trennlinie \kreaturinfo{Info}{Der Dämon erhebt eine Leiche als Untoten. Er verliert Unsichtbarkeit, WS, GS und erhält die Eigenschaften/Attacken des Untoten. Der Dämon kontrolliert den Untoten vollständig. Jeder Dienst des Oboraddon beginnt damit, dass er eine Leiche als Untoten erhebt, wie oben beschrieben. König der Untoten: indem er dem von ihm beseelten Untoten zusätzliche Fähigkeiten im Wert von {-}20 verleiht\newline%
}\trennlinie \kreaturinfo{Quelle}{\href{https://dsaforum.de/viewtopic.php?p=1887303\#p1887303}{Pandämonium}}}}
\newcommand{\kreaturdetailoger}{\kreatur{Oger}{stinkender Vielfraß und Menschenfresser; großer Gegner}{gfx/kreaturen/humanoid}{\kreaturkampfwerte{12}{8}{6}{4}\trennlinie \kreaturwaffe{Faust}{1}{8}{14}{2W6+4}{Stumpf}\kreaturwaffe{Keule}{2}{8}{14}{4W6+6}{Stumpf, Niederwerfen ({-}4)}\kreaturkampfvorteile{Kraftvoller Kampf III, Niederwerfen, Sturmangriff}\trennlinie \kreaturattribute{GE 4, KK 22, KL {-}2, KO 20, MU 20}\kreaturfertigkeiten{Laufen 14, Pirschen 8, Wachsamkeit 6, Zähigkeit 16}\trennlinie \kreaturinfo{Quelle}{\href{https://ilarisblog.wordpress.com/downloads/}{Ilaris Regeln}}}}
\newcommand{\kreaturdetailogerkriegsoger}{\kreatur{Kriegsoger}{stinkender Vielfraß und Menschenfresser; großer Gegner}{gfx/kreaturen/tier}{\kreaturkampfwerte{14}{10}{8}{6}\trennlinie \kreaturwaffe{Faust}{1}{8}{14}{2W6+8}{Stumpf}\kreaturwaffe{Keule}{2}{8}{14}{4W6+8}{Stumpf, Niederwerfen ({-}4)}\kreaturkampfvorteile{Kraftvoller Kampf III, Niederwerfen, Sturmangriff}\trennlinie \kreaturattribute{GE 4, KK 22, KL {-}2, KO 20, MU 20}\kreaturfertigkeiten{Laufen 14, Pirschen 8, Wachsamkeit 6, Zähigkeit 16}\trennlinie \kreaturinfo{Quelle}{\href{https://ilarisblog.wordpress.com/downloads/}{Ilaris Regeln}}}}
\newcommand{\kreaturdetailogerstatue}{\kreatur{Ogerstatue}{behäbige Bronzestatue eines Ogers; großer Gegner}{gfx/kreaturen/tier}{\kreaturkampfwerte{10}{20}{1}{{-}1}\trennlinie \kreaturvorteile{Immunität (Gifte, Krankheiten), Resistenz II (Feuer, Stichwaffen)}\trennlinie \kreaturwaffe{Faust}{2}{4}{12}{2W20+6}{Niederwerfen ({-}4)}\kreaturkampfvorteile{Zusätzliche Attacke I, Unaufhaltsam}\trennlinie \kreaturattribute{KK 44, KO 48}\kreaturfertigkeiten{Laufen {-}2, Wachsamkeit 4}\trennlinie \kreaturinfo{Beschwörung}{Stein wandle!}\kreaturinfo{Dienste}{Lastentransport+4 (1 Tag), Wache+0 (1 Tag), Kampf{-}4 (1 Minute)}\trennlinie \kreaturinfo{Quelle}{\href{https://ilarisblog.wordpress.com/downloads/}{Ilaris Regeln}}}}
\newcommand{\kreaturdetailoghul}{\kreatur{Oghul}{ansteckender und gieriger großer Leichenfresser}{gfx/kreaturen/tier}{\kreaturkampfwerte{8}{8}{6}{4}\trennlinie \kreaturinfo{Empfindlichkeit II}{Sonnenlicht}\kreaturinfo{Immunität}{Einfluss}\kreaturinfo{Angepasst II (Dunkelheit)}{Angepasst I/II (Umgebung) Durch deine Spezies oder langjährige Erfahrung hast du dich an eine bestimmte Umgebung oder Umweltbedingung gewöhnt. Abzüge durch diese Umgebung (Beispiele auf S. 38), insbesondere im Kampf, sinken für dich um eine/zwei Stufen. Die Kosten für Angepasst legt der Spielleiter fest, wobei er sich an der Häufigkeit der Umgebung orientieren sollte. Zu allgemein gefasste Umgebungen wie „unsicherer Untergrund“ sollte er nicht zulassen. Beispiele für Angepasst sind: • Dunkelheit: verringert Abzüge durch schlechte Lichtverhältnisse (40 EP pro Stufe) • Schnee: verringert Abzüge durch schneebedeckten oder eisigen Untergrund (20 EP pro Stufe) • Wasser: verringert Abzüge durch knie{-} oder hüfttiefes Wasser und unter Wasser (20 EP pro Stufe) • Wald: verringert Abzüge durch Wurzeln, Gestrüpp und dichtes Unterholz (40 EP pro Stufe) Voraussetzungen: keine/Angepasst I Nachkauf: häufig/selten\newline%
}\kreaturvorteile{Lichtscheu, Schreckgestalt I}\trennlinie \kreaturwaffe{Faust}{1}{8}{14}{2W6+4}{Niederwerfen (–4), Stumpf, Hochansteckend}\kreaturwaffe{Zähne}{0}{8}{14}{3W6+2}{Ghulgift}\kreaturwaffe{Felsbrocken}{32}{}{}{4W6+4}{Niederwerfen ({-}8)}\kreaturkampfvorteile{Kraftvoller Kampf III, Niederwerfen, Sturmangriff}\trennlinie \kreaturattribute{GE 4, KK 44, KL {-}2, KO 40, MU 20}\kreaturfertigkeiten{Laufen 14, Wachsamkeit 6, Pirschen 8, Zähigkeit 16}\trennlinie \kreaturinfo{Info}{Varianten:  Mächtiger Oghul (WS 10/12, TP +4, Kampfwerte +2)}\trennlinie \kreaturinfo{Quelle}{\href{https://www.orkenspalter.de/filebase/index.php?file/2829-hilberts-bestiarium/}{Hilberts Bestarium}}}}
\newcommand{\kreaturdetailorciosil}{\kreatur{Orciosil}{Die brüllende Seele des Sturms; ein minderer Diener Agrimoths, kleiner Gegner}{gfx/kreaturen/daemon}{\kreaturkampfwerte{3}{8}{1}{{-}1}\trennlinie \kreaturvorteile{Rudel}\trennlinie \kreaturwaffe{Zunge}{1}{2}{10}{1W6+2}{}\trennlinie \kreaturfertigkeiten{Pirschen 18, Wachsamkeit 12}\trennlinie \kreaturinfo{AsP}{16}\kreaturinfo{Dämonisch}{10 (Aeolitus)}\trennlinie \kreaturinfo{Beschwörung}{Invocatio}\kreaturinfo{Dienste}{Segelboot antreiben+4 (1 Tag), Gegner auf Abstand halten+0 (1 Minute), Wachstum zu einem großen Orciosil{-}4 (1 Tag)}\trennlinie \kreaturinfo{Quelle}{\href{https://dsaforum.de/viewtopic.php?p=1887303\#p1887303}{Pandämonium}}}}
\newcommand{\kreaturdetailorkkrieger}{\kreatur{Ork{-}Krieger}{durchschnittlicher orkischer Kämpfer}{gfx/kreaturen/humanoid}{\kreaturkampfwerte{5}{5}{4}{3}\trennlinie \kreaturwaffe{Arbach}{1}{10}{10}{2W6+6}{Kopflastig}\kreaturwaffe{Kriegshammer}{1}{10}{10}{3W6+6}{Kopflastig, Zweihändig}\kreaturwaffe{Faust}{0}{9}{9}{1W6+2}{Kopflastig, Stumpf, Wendig, Zerbrechlich}\kreaturkampfvorteile{Kraftvoller Kampf II, Natürliche Rüstung, Niederwerfen, Offensiver Kampfstil, Rüstungsgewöhnung, Waffenloser Kampf}\trennlinie \kreaturattribute{CH 2, FF 4, GE 8, IN 6, KK 12, KL 2, KO 12, MU 14}\kreaturfertigkeiten{Laufen 12, Wachsamkeit 6, Zähigkeit 12}\trennlinie \kreaturinfo{Quelle}{\href{https://ilarisblog.wordpress.com/downloads/}{Ilaris Regeln}}}}
\newcommand{\kreaturdetailorkkundschafter}{\kreatur{Ork{-}Kundschafter}{wildniserfahrener orkischer Kundschafter oder Räuber}{gfx/kreaturen/humanoid}{\kreaturkampfwerte{5}{5}{5}{4}\trennlinie \kreaturwaffe{Skraja}{0}{8}{8}{2W6+2}{Rüstungsbrechend}\kreaturwaffe{Ork. Reiterbogen}{32}{}{}{2W6+3}{Zweihändig}\kreaturkampfvorteile{Natürliche Rüstung, Scharfschuss, Reflexschuss}\trennlinie \kreaturattribute{CH 2, FF 12, GE 8, IN 8, KK 10, KL 2, KO 12, MU 10}\kreaturfertigkeiten{Klettern 12, Laufen 12, Nordaventurien (Überleben) 13, Pirschen 12, Sinnenschärfe 10, Wachsamkeit 10, Zähigkeit 8}\trennlinie \kreaturinfo{Quelle}{\href{https://ilarisblog.wordpress.com/downloads/}{Ilaris Regeln}}}}
\newcommand{\kreaturdetailpanther}{\kreatur{Panther}{König der Dschungel und nördlichen Steppen; großer Gegner}{gfx/kreaturen/tier}{\kreaturkampfwerte{11}{6}{11}{10}\trennlinie \kreaturinfo{Angepasst II (Dunkelheit)}{Angepasst I/II (Umgebung) Durch deine Spezies oder langjährige Erfahrung hast du dich an eine bestimmte Umgebung oder Umweltbedingung gewöhnt. Abzüge durch diese Umgebung (Beispiele auf S. 38), insbesondere im Kampf, sinken für dich um eine/zwei Stufen. Die Kosten für Angepasst legt der Spielleiter fest, wobei er sich an der Häufigkeit der Umgebung orientieren sollte. Zu allgemein gefasste Umgebungen wie „unsicherer Untergrund“ sollte er nicht zulassen. Beispiele für Angepasst sind: • Dunkelheit: verringert Abzüge durch schlechte Lichtverhältnisse (40 EP pro Stufe) • Schnee: verringert Abzüge durch schneebedeckten oder eisigen Untergrund (20 EP pro Stufe) • Wasser: verringert Abzüge durch knie{-} oder hüfttiefes Wasser und unter Wasser (20 EP pro Stufe) • Wald: verringert Abzüge durch Wurzeln, Gestrüpp und dichtes Unterholz (40 EP pro Stufe) Voraussetzungen: keine/Angepasst I Nachkauf: häufig/selten\newline%
}\kreaturinfo{Blitzschnell}{erleidet keine Passierschläge wenn er sich aus dem Nahkampf zurückzieht}\kreaturvorteile{}\trennlinie \kreaturwaffe{Prankenhieb}{1}{12}{16}{2W6+4}{Doppelangriff, Niederwerfen ({-}4)}\kreaturwaffe{Biss}{0}{2}{16}{5W6+2}{Zerbrechlich}\kreaturkampfvorteile{Standfest, Sturmangriff}\trennlinie \kreaturattribute{GE 20, KK 20, KO 22, MU 18}\kreaturfertigkeiten{Laufen 16, Pirschen 18, Wachsamkeit 18, Zähigkeit 16}\trennlinie \kreaturinfo{Quelle}{\href{https://ilarisblog.wordpress.com/downloads/}{Ilaris Regeln}}}}
\newcommand{\kreaturdetailperldrache}{\kreatur{Perldrache}{intelligenter Drache und geschickter Fischer; großer Gegner}{gfx/kreaturen/mythen}{\kreaturkampfwerte{8}{12}{2}{6}\trennlinie \kreaturinfo{Angepasst I (Wasser)}{Angepasst I/II (Umgebung) Durch deine Spezies oder langjährige Erfahrung hast du dich an eine bestimmte Umgebung oder Umweltbedingung gewöhnt. Abzüge durch diese Umgebung (Beispiele auf S. 38), insbesondere im Kampf, sinken für dich um eine/zwei Stufen. Die Kosten für Angepasst legt der Spielleiter fest, wobei er sich an der Häufigkeit der Umgebung orientieren sollte. Zu allgemein gefasste Umgebungen wie „unsicherer Untergrund“ sollte er nicht zulassen. Beispiele für Angepasst sind: • Dunkelheit: verringert Abzüge durch schlechte Lichtverhältnisse (40 EP pro Stufe) • Schnee: verringert Abzüge durch schneebedeckten oder eisigen Untergrund (20 EP pro Stufe) • Wasser: verringert Abzüge durch knie{-} oder hüfttiefes Wasser und unter Wasser (20 EP pro Stufe) • Wald: verringert Abzüge durch Wurzeln, Gestrüpp und dichtes Unterholz (40 EP pro Stufe) Voraussetzungen: keine/Angepasst I Nachkauf: häufig/selten\newline%
}\kreaturinfo{Aura}{Hitze Zähigkeit (20) alle 4 INI:phasen 1 Wunde}\kreaturinfo{Resistenz II}{Feuer}\kreaturvorteile{Flugfähig, Schreckgestalt I}\trennlinie \kreaturwaffe{Biss}{2}{8}{12}{5W6+4}{}\kreaturwaffe{Klauenhieb}{1}{12}{12}{4W6+4}{}\kreaturwaffe{Schwanzschlag}{2}{8}{14}{2W6+2}{Flächenangriff (180° hinter dem Drachen)}\kreaturkampfvorteile{Niederwerfen, Standfest, Sturmangriff, Zerstörerisch I, Zusätzliche Attacke I}\trennlinie \kreaturattribute{GE 12, KK 70, KO 74, MU 20}\kreaturfertigkeiten{Einschüchtern 10, Fliegen 18, Menschenkenntnis 8, Wachsamkeit 10, Zähigkeit 24}\trennlinie \kreaturinfo{AsP}{24}\kreaturinfo{Hellsicht}{12 (Nach SL{-}Entscheid)}\kreaturinfo{Illusion}{12 (Nach SL{-}Entscheid)}\trennlinie \kreaturinfo{Quelle}{\href{https://ilarisblog.wordpress.com/downloads/}{Ilaris Regeln}}}}
\newcommand{\kreaturdetailperldrachewestwinddrache}{\kreatur{Westwinddrache}{intelligenter Drache und geschickter Fischer; großer Gegner}{gfx/kreaturen/mythen}{\kreaturkampfwerte{8}{12}{2}{6}\trennlinie \kreaturinfo{Angepasst I (Wasser)}{Angepasst I/II (Umgebung) Durch deine Spezies oder langjährige Erfahrung hast du dich an eine bestimmte Umgebung oder Umweltbedingung gewöhnt. Abzüge durch diese Umgebung (Beispiele auf S. 38), insbesondere im Kampf, sinken für dich um eine/zwei Stufen. Die Kosten für Angepasst legt der Spielleiter fest, wobei er sich an der Häufigkeit der Umgebung orientieren sollte. Zu allgemein gefasste Umgebungen wie „unsicherer Untergrund“ sollte er nicht zulassen. Beispiele für Angepasst sind: • Dunkelheit: verringert Abzüge durch schlechte Lichtverhältnisse (40 EP pro Stufe) • Schnee: verringert Abzüge durch schneebedeckten oder eisigen Untergrund (20 EP pro Stufe) • Wasser: verringert Abzüge durch knie{-} oder hüfttiefes Wasser und unter Wasser (20 EP pro Stufe) • Wald: verringert Abzüge durch Wurzeln, Gestrüpp und dichtes Unterholz (40 EP pro Stufe) Voraussetzungen: keine/Angepasst I Nachkauf: häufig/selten\newline%
}\kreaturinfo{Aura}{Hitze}\kreaturinfo{Zähigkeit}{(20) alle 4 INI:phasen 1 Wunde}\kreaturinfo{Resistenz II}{Feuer}\kreaturvorteile{Flugfähig, Schreckgestalt I}\trennlinie \kreaturwaffe{Biss}{2}{8}{12}{5W6+4}{}\kreaturwaffe{Klauenhieb}{1}{12}{12}{4W6+4}{}\kreaturwaffe{Schwanzschlag}{2}{8}{14}{2W6+2}{Flächenangriff (180° hinter dem Drachen)}\kreaturkampfvorteile{Niederwerfen, Standfest, Sturmangriff, Zerstörerisch I, Zusätzliche Attacke I}\trennlinie \kreaturattribute{GE 12, KK 70, KO 74, MU 20}\kreaturfertigkeiten{Einschüchtern 10, Fliegen 18, Menschenkenntnis 8, Wachsamkeit 10, Zähigkeit 24}\trennlinie \kreaturinfo{Quelle}{\href{https://ilarisblog.wordpress.com/downloads/}{Ilaris Regeln}}}}
\newcommand{\kreaturdetailpershirash}{\kreatur{Pershirash}{Der gierige Schlinger; minderer Diener Belshirashs, sehr kleiner Gegner}{gfx/kreaturen/daemon}{\kreaturkampfwerte{2}{10}{1}{2}\trennlinie \kreaturinfo{Auge des Pershirash legen}{Für ein ausgerissenes Auge legt Pershirash ein kristallenes Ei. Das Ei erleichtert FK{-}Angriffe des Beschwörers um +4/+6/+8 für 7 Tage/7 Monate/7 Jahre, je nachdem ob es von einem Tier/kulturschaffenden Wesen/dem Beschwörer selbst stammt. Die Effekte mehrere Eier sind nicht kumulativ, es zählt nur der höchste Bonus.)\newline%
}\kreaturvorteile{Flieger, Lebensraub, Verbindung zum Beschwörer}\trennlinie \kreaturwaffe{Krallen}{0}{8}{12}{1W6+5}{Erfrieren, Umklammern}\kreaturwaffe{Schnabel}{0}{2}{14}{2W6+4}{}\kreaturkampfvorteile{Sturmangriff}\trennlinie \kreaturfertigkeiten{Pirschen 16, Fliegen 18, Wachsamkeit 16, Zähigkeit 2}\trennlinie \kreaturinfo{AsP}{24}\kreaturinfo{Dämonisch}{10 (Axxeleratus, Exposami)}\trennlinie \kreaturinfo{Beschwörung}{Invocatio}\kreaturinfo{Dienste}{Jagd auf ein Tier+4 (1 Stunde), Auge ausreißen und Kristall{-}Ei erschaffen+0, Spionage{-}4 (1 Tag)}\trennlinie \kreaturinfo{Quelle}{\href{https://dsaforum.de/viewtopic.php?p=1887303\#p1887303}{Pandämonium}}}}
\newcommand{\kreaturdetailpestrattenschwarm}{\kreatur{Pestrattenschwarm}{ansteckender Schwarm aus Rattenkadavern; großer Gegner}{gfx/kreaturen/untot}{\kreaturkampfwerte{4}{2}{2}{0}\trennlinie \kreaturinfo{Astralsinn}{Astralsinn erlaubt es der Kreatur, ihre Umgebung magisch wahrzunehmen. Sie erleidet keine Abzüge durch schlechte Sicht. Der Astralsinn kann durch Antimagie, zum Beispiel Hellsicht trüben in der Modifikation Magie unterdrücken, gestört werden. Die Schwierigkeit dafür liegt mindestens bei 20, bei mächtigen Wesen deutlich höher.\newline%
}\kreaturinfo{Resistenz II}{Stichwaffen}\kreaturinfo{Verwundbarkeit IV}{Flächenschaden}\kreaturvorteile{Schmerzimmun II}\trennlinie \kreaturwaffe{Biss}{0}{6}{14}{2W6+0}{Hochansteckend (Jede erlittene Wunde erhöht das Risiko einer Ansteckung mit Wundbrand um 50\%. Eine Probe auf Gifte und Krankheiten (24) eliminiert das Risiko.)}\kreaturkampfvorteile{Zusätzliche Attacke I}\trennlinie \kreaturfertigkeiten{Laufen 4, Pirschen 4, Wachsamkeit 16}\trennlinie \kreaturinfo{Beschwörung}{Skelettarius, Totes handle!}\kreaturinfo{Dienste}{Stall leerfessen (1 Stunde, +4)+0,  andere Lebewesen fressen (1 Minute)+0,  nur bestimmtes Lebewesen fressen (1 Minute, {-}4)+0}\trennlinie \kreaturinfo{Quelle}{\href{https://ilarisblog.wordpress.com/downloads/}{Ilaris Regeln}}}}
\newcommand{\kreaturdetailpestrattenschwarmsehrgrosserschwarm}{\kreatur{Sehr großer Pestrattenschwarm}{ansteckender Schwarm aus Rattenkadavern; großer Gegner}{gfx/kreaturen/untot}{\kreaturkampfwerte{5}{2}{2}{0}\trennlinie \kreaturinfo{Astralsinn}{Astralsinn erlaubt es der Kreatur, ihre Umgebung magisch wahrzunehmen. Sie erleidet keine Abzüge durch schlechte Sicht. Der Astralsinn kann durch Antimagie, zum Beispiel Hellsicht trüben in der Modifikation Magie unterdrücken, gestört werden. Die Schwierigkeit dafür liegt mindestens bei 20, bei mächtigen Wesen deutlich höher.\newline%
}\kreaturinfo{Resistenz II}{Stichwaffen}\kreaturinfo{Verwundbarkeit IV}{Flächenschaden}\kreaturvorteile{Schmerzimmun II}\trennlinie \kreaturwaffe{Biss}{0}{6}{16}{2W6+0}{Hochansteckend (Jede erlittene Wunde erhöht das Risiko einer Ansteckung mit Wundbrand um 50\%. Eine Probe auf Gifte und Krankheiten (24) eliminiert das Risiko.)}\kreaturkampfvorteile{Zusätzliche Attacke II}\trennlinie \kreaturfertigkeiten{Laufen 4, Pirschen 4, Wachsamkeit 16}\trennlinie \kreaturinfo{Beschwörung}{Skelettarius, Totes handle!}\kreaturinfo{Dienste}{Stall leerfessen+4 (1 Stunde), Andere Lebewesen fressen+0 (1 Minute), nur bestimmtes Lebewesen fressen{-}4 (1 Minute)}\trennlinie \kreaturinfo{Quelle}{\href{https://ilarisblog.wordpress.com/downloads/}{Ilaris Regeln}}}}
\newcommand{\kreaturdetailpoltergeist}{\kreatur{Poltergeist}{nächtlicher Unruhestifter}{gfx/kreaturen/geist}{\kreaturkampfwerte{4}{8}{4}{4}\trennlinie \kreaturinfo{Telekinese{-}Feld}{In einer Reichweite von 8 Schritt kann der Poltergeist Gegenstände telekinetisch bewegen und werfen}\kreaturvorteile{Tarnung}\trennlinie \kreaturwaffe{Waffenlos}{0}{4}{8}{1W6+0}{Rüstungsbrechend}\kreaturwaffe{Kleines W.g.}{4}{}{}{2W6+1}{Meist Stumpf}\kreaturwaffe{Mittleres W.g.}{2}{}{}{2W6+3}{Meist Stumpf, Niederwerfen ({-}4)}\kreaturwaffe{Schweres W.g.}{1}{}{}{4W6+6}{Meist Stumpf, Niederwerfen ({-}8)}\trennlinie \kreaturattribute{CH 10, FF 12, GE 4, IN 6, KK 20, KL 10, KO 4, MU 10}\kreaturfertigkeiten{Einschüchtern 12, Sinnesschärfe 6, Pirschen 10, Willenskraft 10}\trennlinie \kreaturinfo{AsP}{20}\kreaturinfo{Verwandlung}{4 (Desintegratus Pulverstaub, Claudibus Clavistibor)}\trennlinie \kreaturinfo{Info}{Vorahnung: Gegen Fernkampfangriffe des Poltergeists kann mit IN (24, I) verteidigt werden.\newline%
Varianten:  Starker Poltergeist (32 AsP, Die Reichweite aller Fernkampfangriffe und des Telekinese{-}Felds ist verdoppelt. Schreckgestalt II. Verwandlung 8, Pirschen 14)\newline%
Mächtiger Poltergeist (40 AsP. Unsichtbarkeit, Schreckgestalt III. Die Reichweite aller Fernkampfangriffe und des Telekinese{-}Felds ist vervierfacht. Verwandlung 18 (Alle), Pirschen 16)\newline%
}\trennlinie \kreaturinfo{Quelle}{\href{https://www.orkenspalter.de/filebase/index.php?file/2829-hilberts-bestiarium/}{Hilberts Bestarium}}}}
\newcommand{\kreaturdetailqasaar}{\kreatur{Qasaar}{Der schnurrende Verschlinger; minderer Diener der Aphasmayra, sehr kleiner Gegner}{gfx/kreaturen/daemon}{\kreaturkampfwerte{2}{8}{8}{10}\trennlinie \kreaturinfo{Immunität}{profan}\kreaturinfo{Resistenz}{magisch, geweiht}\kreaturvorteile{Präsenz, Regeneration I}\trennlinie \kreaturwaffe{Biss}{0}{12}{12}{1W6+2}{}\kreaturkampfvorteile{Defensiver Kampfstil}\trennlinie \kreaturinfo{Beschwörung}{Invocatio}\kreaturinfo{Dienste}{Opfer terrorisieren+4 (1 Jahr)}\trennlinie \kreaturinfo{Quelle}{\href{https://dsaforum.de/viewtopic.php?p=1887303\#p1887303}{Pandämonium}}}}
\newcommand{\kreaturdetailraeuber}{\kreatur{Räuber}{ausgehungerte Halsabschneider auf der Suche nach leichter Beute}{gfx/kreaturen/humanoid}{\kreaturkampfwerte{4}{5}{4}{2}\trennlinie \kreaturvorteile{Zweihändig}\trennlinie \kreaturwaffe{Keule}{1}{8}{8}{2W6+0}{Stumpf, Kopflastig}\kreaturwaffe{Holzspeer}{2}{8}{8}{2W6+1}{Wendig}\kreaturwaffe{Kurzbogen}{16}{}{}{2W6+1}{Zweihändig}\trennlinie \kreaturattribute{CH 4, FF 8, GE 6, IN 4, KK 6, KL 6, KO 6, MU 8}\kreaturfertigkeiten{Pirschen 8, Wachsamkeit 6, Zähigkeit 4}\trennlinie \kreaturinfo{Info}{Der Räuber steht beispielhaft für schwache humanoide Gegner, die für einen echten Kämpfer kaum eine Gefahr darstellen.}\trennlinie \kreaturinfo{Quelle}{\href{https://ilarisblog.wordpress.com/downloads/}{Ilaris Regeln}}}}
\newcommand{\kreaturdetailreitpony}{\kreatur{Reitpony}{kampferprobtes Reittier aus den Ställen Elenvinas; großer Gegner}{gfx/kreaturen/tier}{\kreaturkampfwerte{11}{6}{10}{1}\trennlinie \kreaturwaffe{Tritt}{1}{5}{12}{2W6+0}{Niederwerfen}\kreaturwaffe{Biss}{0}{2}{12}{1W6+1}{Zerbrechlich}\kreaturkampfvorteile{Rüstungsgewöhnung, Standfest, im Reiterkampf}\trennlinie \kreaturattribute{GE 8, KK 26, KO 24, MU 12}\kreaturfertigkeiten{Wachsamkeit 10, Zähigkeit 10}\trennlinie \kreaturinfo{Info}{Im Reiterkampf sind hauptsächlich WS:, GS: und TP:  bedeutend, da AT: und VT: auf den PW Reiten abgelegt werden.}\trennlinie \kreaturinfo{Quelle}{\href{https://ilarisblog.wordpress.com/downloads/}{Ilaris Regeln}}}}
\newcommand{\kreaturdetailreittier}{\kreatur{Reitpferd}{kampferprobtes Reittier aus den Ställen Elenvinas; großer Gegner}{gfx/kreaturen/tier}{\kreaturkampfwerte{13}{6}{12}{1}\trennlinie \kreaturwaffe{Tritt}{1}{5}{12}{2W6+0}{Niederwerfen}\kreaturwaffe{Biss}{0}{2}{12}{1W6+1}{Zerbrechlich}\kreaturkampfvorteile{Rüstungsgewöhnung, Standfest, im Reiterkampf}\trennlinie \kreaturattribute{GE 8, KK 26, KO 24, MU 12}\kreaturfertigkeiten{Wachsamkeit 10, Zähigkeit 10}\trennlinie \kreaturinfo{Info}{Im Reiterkampf sind hauptsächlich WS:, GS: und TP:  bedeutend, da AT: und VT: auf den PW Reiten abgelegt werden.}\trennlinie \kreaturinfo{Quelle}{\href{https://ilarisblog.wordpress.com/downloads/}{Ilaris Regeln}}}}
\newcommand{\kreaturdetailriese}{\kreatur{Riese}{Unsterblicher Gigant; sehr großer Gegner}{gfx/kreaturen/mythen}{\kreaturkampfwerte{16}{16}{9}{6}\trennlinie \kreaturvorteile{Schreckgestalt II, Unbeugsamkeit, Willensstark II}\trennlinie \kreaturwaffe{Faust}{2}{12}{16}{2W20+5}{Niederwerfen ({-}12)}\kreaturwaffe{Baumstamm}{5}{12}{18}{3W20+10}{Flächenangriff (180° vor dem Riesen), Niederwerfen ({-}16), Zurückstoßen (wird nicht durch erfolgreiche VT: gestoppt)}\kreaturwaffe{Trampeln}{1}{6}{16}{4W20+0}{Flächenangriff (1 Schritt Umkreis), Niederwerfen ({-}16), Überrennen}\kreaturwaffe{Felsbrocken}{16}{}{14}{3W20+0}{Flächenangriff (1 Schritt Radius um das Hauptziel), Niederwerfen ({-}12)}\kreaturkampfvorteile{Zusätzliche Attacke I, Standfest, Sturmangriff, Unaufhaltsam, Zerstörerisch I, Zerstörerisch II}\trennlinie \kreaturattribute{GE 6, KK 150, KO 130, MU 20}\kreaturfertigkeiten{Laufen 16, Wachsamkeit 10, Zähigkeit 26}\trennlinie \kreaturinfo{Info}{kampfunfähigkeit: Der Riese flieht, wenn er 6 Wunden erlitten hat.}\trennlinie \kreaturinfo{schwachpunkte}{Im Nacken des Riesen, vom Boden unsichtbar, liegt eine armdicke Ader direkt unter der Haut. Sie kann entweder von einem Fernkämpfer in ähnlicher Höhe mit Sinnenschärfe (28) erspäht und mit einem FK: (28) getroffen werden, oder ein tollkühner Kletterer erklimmt den Riesen (Klettern (28)). Nach einem Treffer müssen sich nur noch alle Umstehenden in Sicherheit bringen, wenn sich der Riese in Schmerzen zu Boden wirft und alles niederwalzt (in 3 Schritt Umkreis GE: oder Akrobatik 30, sonst 3W20 TP:  ). Das kostet den Riesen 1 Aktion und gibt Umstehenden die Chance auf einen Passierschlag.}\kreaturinfo{vorgehen}{Eine Übermacht hält sich der Riese mit Befreiungsschlägen mit einem Baumstamm vom Leib, während er die Gegner unter sich niedertrampelt. Wegen des Vorteils Unaufhaltsam sollte seinen Angriffen unbedingt ausgewichen werden. Sollte ihm ein Fernkämpfer oder Zauberer gefährlich werden, stürmt er auf ihn zu (und trampelt alles im Weg nieder) oder bewirft ihn mit gewaltigen Felsbrocken.}\trennlinie \kreaturinfo{Quelle}{\href{https://ilarisblog.wordpress.com/downloads/}{Ilaris Regeln}}}}
\newcommand{\kreaturdetailriesenaffe}{\kreatur{Riesenaffe}{Erstaunlich Intelligenter und gefährlicher Dschungelbewohner}{gfx/kreaturen/tier}{\kreaturkampfwerte{11}{2}{6}{}\trennlinie \kreaturwaffe{Prankenhieb}{1}{4}{8}{3W6+4}{Stumpf}\kreaturwaffe{Knüppel}{2}{3}{7}{5W6+2}{Befreiungsschlag}\kreaturkampfvorteile{Niederwerfen}\trennlinie \kreaturattribute{GE 8, KK 40, KO 32, MU 20}\kreaturfertigkeiten{Wachsamkeit 8, Zähigkeit 16, Pirschen 10, Klettern 12, Laufen 14, Akrobatik 10}\trennlinie \kreaturinfo{Quelle}{\href{https://dsaforum.de/viewtopic.php?f=180&p=1738549\#p1738549}{Bestarium+}}}}
\newcommand{\kreaturdetailriesenamoebe}{\kreatur{Riesenamöbe}{allesfressender Schleimhaufen; großer Gegner}{gfx/kreaturen/tier}{\kreaturkampfwerte{6}{14}{1}{0}\trennlinie \kreaturinfo{Immunität}{Niederwerfen Umreißen und ähnliche Effekte}\kreaturinfo{Resistenz I}{Stichwaffen, stumpfe Waffen}\kreaturvorteile{Körperlosigkeit, Schmerzimmun II}\trennlinie \kreaturwaffe{Scheinarm}{1}{2}{6}{1W6+0}{Umklammern ({-}2, 16), Ätzend (Ein Opfer in der Umklammerung erleidet in jeder INI:phase 2W6 SP. Die Riesenamöbe kann nur ein Opfer Umklammern.)}\trennlinie \kreaturfertigkeiten{Laufen {-}8}\trennlinie \kreaturinfo{Quelle}{\href{None}{None}}}}
\newcommand{\kreaturdetailriesenfeuerkaefer}{\kreatur{Riesenfeuerkäfer}{flammendes Insekt}{gfx/kreaturen/elementar}{\kreaturkampfwerte{3}{13}{4}{4}\trennlinie \kreaturinfo{Überhitzung}{fallen bei der Bestimmung der Trefferpunkte mindestens zwei Sechsen, stirbt der Käfer}\kreaturvorteile{}\trennlinie \kreaturwaffe{Biss}{0}{6}{11}{3W6+0}{Nachbrennen, Niederwerfen}\trennlinie \kreaturattribute{CH 4, FF 2, GE 8, IN 2, KK 4, KL 2, KO 4, MU 4}\kreaturfertigkeiten{Willenskraft 4, Einschüchtern 6, Wachsamkeit 4}\trennlinie \kreaturinfo{Spezialfähigkeiten}{Zerberstender ({-}X, Explosion(XW6); maximal 4), Kochendes Blut ({-}4, Werden dem Riesenfeuerkäfer X Wunden zugefügt, muss jedes Wesen in Reichweite eine GE (16,I) ablegen, um nicht je 2XW6 SP zu erleiden)\newline%
}\trennlinie \kreaturinfo{Quelle}{\href{https://www.orkenspalter.de/filebase/index.php?file/2829-hilberts-bestiarium/}{Hilberts Bestarium}}}}
\newcommand{\kreaturdetailriesenkaiman}{\kreatur{Riesenkaiman}{gefährlicher Jäger der südlichen Gewässer}{gfx/kreaturen/tier}{\kreaturkampfwerte{13}{4}{4}{4}\trennlinie \kreaturinfo{Immunität}{Niederwerfen Umreißen und ähnliche Effekte}\kreaturinfo{Verwundbarkeit}{Eis}\kreaturvorteile{Amphibisch}\trennlinie \kreaturwaffe{Biss}{0}{6}{8}{4W6+4}{Zerbrechlich}\kreaturwaffe{Schwanz}{2}{2}{6}{1W6+2}{Flächenangriff (90° hinter dem Kaiman), Niederwerfen}\kreaturkampfvorteile{Zusätzliche Attacke I}\trennlinie \kreaturattribute{KK 20, KO 38, MU 10}\kreaturfertigkeiten{Pirschen 14 None, Schwimmen 10 None, Wachsamkeit 6 None}\trennlinie \kreaturinfo{Quelle}{\href{https://ilarisblog.wordpress.com/downloads/}{Ilaris Regeln}}}}
\newcommand{\kreaturdetailriesenkraken}{\kreatur{Riesenkraken}{Albtraum vieler Seefahrer; sehr großer Endgegner}{gfx/kreaturen/tier}{\kreaturkampfwerte{{-}1}{20}{6}{2}\trennlinie \kreaturvorteile{Schreckgestalt III, Wasserwesen, Willensstark II}\trennlinie \kreaturwaffe{Biss (Rumpf)}{1}{2}{14}{3W20+10}{Rüstungsbrechend}\kreaturwaffe{Tentakelhieb}{12}{8}{18}{4W6+2}{Flächenangriff (45°), Niederwerfen ({-}8), Zurückstoßen}\kreaturwaffe{Tentakelklammer}{10}{8}{15}{4W6+0}{Umklammern ({-}4, 32)}\kreaturkampfvorteile{Zusätzliche Attacke IV, Unaufhaltsam}\trennlinie \kreaturattribute{KK 22, KO 220}\kreaturfertigkeiten{Pirschen 14, Schwimmen 18, Wachsamkeit 12}\trennlinie \kreaturinfo{schwachpunkte}{Der Riesenkraken greift aus der Tiefe an, weswegen er hervorragend gegen Angriffen von oben geschützt ist. Seine Unterseite birgt jedoch ein verwundbares Verdauungsorgan, das allerdings erst erreicht werden muss (Schwimmen (20) und eine Aktion zum Hinabtauchen, Schwimmen (24) und eine weitere Aktion, um in Position zu kommen.) Auch eine Amphore Brandöl kann dem Kraken schwerste Verbrennungen zufügen, wenn sie direkt in das Maul geworfen wird. Für einen solchen Wurf ist eine IN{-}Probe (28) nötig, um die Bewegung des Mauls einzuschätzen. Der folgende Wurf erfordert eine KK{-}Probe (28 oder mehr). Die Verwendung von Telekinesemagie ersetzt die KK{-}Probe, nicht aber die IN{-}Probe. Die Verbrennungen führen zu einer Tobsucht, bei der der Kraken alle umklammerten Gegner loslässt und mit 6 Tentakeln und voller Offensive auf das Schiff einschlägt. Kampfunfähigkeit: Der Kraken flieht, wenn er 4 Wunden erlitten oder die Hälfte seiner Tentakel verloren hat.}\kreaturinfo{vorgehen}{Mit ungefähr der Hälfte seiner Tentakel hält der Riesenkraken das Schiff fest, während er mit den anderen das Deck verheert. Dazu nutzt er mindestens drei seiner zusätzlichen Attacken als Tentakelhiebe. Unterdessen versucht er, einzelne Opfer zu fressen: Zuerst umfasst er sie mit einer Tentakelklammer, in der nächsten Aktion zieht er sie zu seinem Maul und in der übernächsten Aktion beißt er zu.}\trennlinie \kreaturinfo{Quelle}{\href{https://ilarisblog.wordpress.com/downloads/}{Ilaris Regeln}}}}
\newcommand{\kreaturdetailriesenspringelschwarm}{\kreatur{Riesenspringelschwarm}{sehr kleines, scheußliches Sumpfgezücht}{gfx/kreaturen/tier}{\kreaturkampfwerte{2}{4}{2}{3}\trennlinie \kreaturinfo{Resistenz I}{Stichwaffen}\kreaturinfo{Verwundbarkeit IV}{Flächenschaden}\kreaturvorteile{Wasserwesen, Schmerzimmun II}\trennlinie \kreaturwaffe{Biss}{0}{4}{12}{2W6+0}{Rüstungsbrechend}\kreaturkampfvorteile{Zusätzliche Attacke I}\trennlinie \kreaturfertigkeiten{Schwimmen 6, Pirschen 6, Wachsamkeit 16}\trennlinie \kreaturinfo{Quelle}{\href{https://dsaforum.de/viewtopic.php?p=1887303\#p1887303}{Allgemeine Gegner}}}}
\newcommand{\kreaturdetailritter}{\kreatur{Ritter}{erfahrener Veteran vieler Gefechte gegen den Ork}{gfx/kreaturen/humanoid}{\kreaturkampfwerte{5}{5}{2}{8}\trennlinie \kreaturwaffe{Breitschwert}{1}{11}{11}{2W6+6}{Kopflastig}\kreaturwaffe{Großschild}{0}{13}{13}{2W6+1}{Schild, Stumpf}\kreaturwaffe{Andergaster}{2}{9}{9}{3W6+7}{Kopflastig, Zweihändig}\kreaturwaffe{Kriegslanze}{2}{0}{9}{3W6+4}{}\kreaturwaffe{Tralloper}{1}{15}{15}{2W6+6}{Niederwerfen}\kreaturkampfvorteile{Reiterkampf III, Schildkampf III, Kraftvoller Kampf III, Ausfall, Kommando: Haltet Stand!, Hammerschlag, Kampfreflexe, Muskelprotz, Niederwerfen, Offensiver Kampfstil, Rüstungsgewöhnung, Durchatmen}\trennlinie \kreaturattribute{CH 10, FF 6, GE 8, IN 8, KK 16, KL 8, KO 12, MU 14}\kreaturfertigkeiten{Anführen 12, Einschüchtern 12, Menschenkenntnis 10, Reiten 15, Wachsamkeit 7, Zähigkeit 12}\kreaturinfo{Profane Vorteile}{Abgehärtet I, Abgehärtet II, Eindrucksvoll I}\trennlinie \kreaturinfo{Info}{Kampfstile sind nicht in die Kampfwerte eingerechnet.}\trennlinie \kreaturinfo{Quelle}{\href{https://ilarisblog.wordpress.com/downloads/}{Ilaris Regeln}}}}
\newcommand{\kreaturdetailsaebelzahntiger}{\kreatur{Säbelzahntiger}{König der Dschungel und nördlichen Steppen; großer Gegner}{gfx/kreaturen/tier}{\kreaturkampfwerte{11}{6}{11}{10}\trennlinie \kreaturinfo{Angepasst II (Dunkelheit)}{Angepasst I/II (Umgebung) Durch deine Spezies oder langjährige Erfahrung hast du dich an eine bestimmte Umgebung oder Umweltbedingung gewöhnt. Abzüge durch diese Umgebung (Beispiele auf S. 38), insbesondere im Kampf, sinken für dich um eine/zwei Stufen. Die Kosten für Angepasst legt der Spielleiter fest, wobei er sich an der Häufigkeit der Umgebung orientieren sollte. Zu allgemein gefasste Umgebungen wie „unsicherer Untergrund“ sollte er nicht zulassen. Beispiele für Angepasst sind: • Dunkelheit: verringert Abzüge durch schlechte Lichtverhältnisse (40 EP pro Stufe) • Schnee: verringert Abzüge durch schneebedeckten oder eisigen Untergrund (20 EP pro Stufe) • Wasser: verringert Abzüge durch knie{-} oder hüfttiefes Wasser und unter Wasser (20 EP pro Stufe) • Wald: verringert Abzüge durch Wurzeln, Gestrüpp und dichtes Unterholz (40 EP pro Stufe) Voraussetzungen: keine/Angepasst I Nachkauf: häufig/selten\newline%
}\kreaturinfo{Blitzschnell}{erleidet keine Passierschläge wenn er sich aus dem Nahkampf zurückzieht}\kreaturvorteile{}\trennlinie \kreaturwaffe{Prankenhieb}{1}{12}{16}{2W6+4}{Doppelangriff, Niederwerfen ({-}4)}\kreaturwaffe{Biss}{0}{2}{16}{5W6+2}{Zerbrechlich}\kreaturkampfvorteile{Standfest, Sturmangriff}\trennlinie \kreaturattribute{GE 20, KK 20, KO 22, MU 18}\kreaturfertigkeiten{Laufen 16, Pirschen 18, Wachsamkeit 18, Zähigkeit 16}\trennlinie \kreaturinfo{Quelle}{\href{https://ilarisblog.wordpress.com/downloads/}{Ilaris Regeln}}}}
\newcommand{\kreaturdetailsandloewe}{\kreatur{Sandlöwe}{Stolzer König der Khômwüste}{gfx/kreaturen/tier}{\kreaturkampfwerte{8}{2}{8}{4}\trennlinie \kreaturwaffe{Biss}{0}{2}{12}{3W6+2}{Zerbrechlich}\kreaturwaffe{Prankenhieb}{1}{6}{14}{2W6+2}{Wendig}\kreaturkampfvorteile{Doppelangriff, Niederwerfen, Standfest, Sturmangriff}\trennlinie \kreaturattribute{GE 20, KK 24, KO 20, MU 16}\kreaturfertigkeiten{Laufen 20, Pirschen 18, Wachsamkeit 14, Zähigkeit 12}\trennlinie \kreaturinfo{Quelle}{\href{https://dsaforum.de/viewtopic.php?p=1887303\#p1887303}{Allgemeine Gegner}}}}
\newcommand{\kreaturdetailschattenloewe}{\kreatur{Sandlöwe}{Stolzer König der Khômwüste}{gfx/kreaturen/tier}{\kreaturkampfwerte{7}{2}{6}{4}\trennlinie \kreaturwaffe{Biss}{0}{2}{12}{3W6+2}{Zerbrechlich}\kreaturwaffe{Prankenhieb}{1}{6}{14}{2W6+2}{Wendig}\kreaturkampfvorteile{Doppelangriff, Niederwerfen, Standfest, Sturmangriff}\trennlinie \kreaturattribute{GE 20, KK 24, KO 20, MU 16}\kreaturfertigkeiten{Laufen 20, Pirschen 24, Wachsamkeit 14, Zähigkeit 12}\trennlinie \kreaturinfo{Quelle}{\href{https://dsaforum.de/viewtopic.php?p=1887303\#p1887303}{Allgemeine Gegner}}}}
\newcommand{\kreaturdetailschlangenmensch}{\kreatur{Schlangenmensch}{Mensch und Schlange}{gfx/kreaturen/daimonid}{\kreaturkampfwerte{10}{13}{5}{5}\trennlinie \kreaturwaffe{Biss}{0}{9}{9}{2W6+0}{}\kreaturwaffe{Waffe}{}{9}{9}{}{}\kreaturwaffe{Kurzbogen}{16}{}{}{2W6+1}{Zweihändig}\trennlinie \kreaturattribute{CH 6, FF 4, GE 6, IN 10, KK 20, KL 0, KO 20, MU 10}\kreaturfertigkeiten{Einschüchtern 10, Klettern 4, Athletik 7, Schwimmen 4, Wachsamkeit 8, Pirschen 6, Willenskraft 6}\trennlinie \kreaturinfo{Info}{Varianten: Giftschlange (Gift wie Schlange)}\trennlinie \kreaturinfo{Quelle}{\href{https://www.orkenspalter.de/filebase/index.php?file/2829-hilberts-bestiarium/}{Hilberts Bestarium}}}}
\newcommand{\kreaturdetailschlinger}{\kreatur{Schlinger}{Hungrige Riesenechse}{gfx/kreaturen/tier}{\kreaturkampfwerte{14}{8}{10}{4}\trennlinie \kreaturvorteile{Schreckgestallt I, Kältestarre}\trennlinie \kreaturwaffe{Biss}{1}{4}{16}{4W6+4}{}\kreaturwaffe{Schwanz}{2}{4}{12}{}{Befreiungsschlag, Niederwerfen}\kreaturkampfvorteile{Zusätzliche Attacke I, Standfest, Kalte Wut}\trennlinie \kreaturattribute{KK 36, KO 32}\kreaturfertigkeiten{Pirschen 4, Wachsamkeit 8}\trennlinie \kreaturinfo{Quelle}{\href{https://dsaforum.de/viewtopic.php?f=180&p=1738549\#p1738549}{Bestarium+}}}}
\newcommand{\kreaturdetailschroeter}{\kreatur{Schröter}{Landplage oder exotisches Haustier}{gfx/kreaturen/tier}{\kreaturkampfwerte{4}{13}{1}{0}\trennlinie \kreaturwaffe{Zange}{1}{4}{9}{2W6+2}{Rüstungsbrechend, Umklammern ({-}2, 18)}\kreaturwaffe{Biss}{0}{2}{12}{2W6+0}{Rüstungsbrechend}\kreaturkampfvorteile{Doppelangriff, Standfest}\trennlinie \kreaturattribute{GE 4, KK 8, KO 14, MU 4}\kreaturfertigkeiten{Wachsamkeit 12, Zähigkeit 10}\trennlinie \kreaturinfo{Quelle}{\href{None}{None}}}}
\newcommand{\kreaturdetailschwerthai}{\kreatur{Schwerthai}{Räuber der Meere}{gfx/kreaturen/tier}{\kreaturkampfwerte{11}{4}{8}{4}\trennlinie \kreaturvorteile{Wasserwesen}\trennlinie \kreaturwaffe{Schwert}{1}{3}{12}{3W6+4}{Umklammern (4), Rüstungsbrechend}\kreaturwaffe{Biss}{0}{4}{10}{2W6+2}{Zerbrechlich}\kreaturkampfvorteile{Sturmangriff}\trennlinie \kreaturattribute{GE 14, KK 20, KO 20, MU 12}\kreaturfertigkeiten{Pirschen 8, Schwimmen 16, Wachsamkeit 12, Zähigkeit 14}\trennlinie \kreaturinfo{Quelle}{\href{https://dsaforum.de/viewtopic.php?p=1887303\#p1887303}{Allgemeine Gegner}}}}
\newcommand{\kreaturdetailscylaphotai}{\kreatur{Scylaphotai}{Die Flammenqualle; minderer Diener Charyptoroths}{gfx/kreaturen/daemon}{\kreaturkampfwerte{6}{10}{}{4}\trennlinie \kreaturinfo{Aura}{Gestank, Zähigkeit (16) alle 4 INI: phasen, für 1 INI: phase handlungsunfähig}\kreaturinfo{Aura}{Kochendes Wasser, Zähigkeit (16), alle 4 INI: phasen, 1 Wunde}\kreaturvorteile{wasserwesen}\trennlinie \kreaturwaffe{Nesselbeschuss}{16}{}{}{2W6+2}{}\trennlinie \kreaturinfo{Beschwörung}{Invocatio}\kreaturinfo{Dienste}{Wasser erhitzen+4, Hafeneinfahrt oder Meerenge bewachen+0 (1 Woche), Kampf{-}4 (1 Minute)}\trennlinie \kreaturinfo{Quelle}{\href{https://dsaforum.de/viewtopic.php?p=1887303\#p1887303}{Pandämonium}}}}
\newcommand{\kreaturdetailsharbazz}{\kreatur{Sharbazz}{Der Weibel der Niederhöllen; zweigehörnter Diener Belhalhars, großer Gegner}{gfx/kreaturen/daemon}{\kreaturkampfwerte{14}{12}{4}{4}\trennlinie \kreaturvorteile{Flugfähig, Regeneration I, Zusätzliche Attacke I}\trennlinie \kreaturwaffe{Säbel}{1}{12}{16}{3W6+6}{}\kreaturwaffe{Tritt}{1}{4}{12}{2W6+4}{Niederwerfen ({-}4)}\kreaturkampfvorteile{Kommando: Formiert Euch!, Kommando: Keine Gefangenen!, Niederwerfen, Sturmangriff}\trennlinie \kreaturfertigkeiten{Anführen 20, Einschüchtern 20, Pirschen 0, Wachsamkeit 12}\trennlinie \kreaturinfo{AsP}{32}\kreaturinfo{Dämonisch}{12 (Corpofesso, Karnifilo)}\trennlinie \kreaturinfo{Beschwörung}{Invocatio}\kreaturinfo{Dienste}{Kontrolle von mehreren niederen Dienern Belhalhars+4 (1 Stunde), Planung einer Schlacht+0, Kontrolle von mehreren Shruufya{-}4 (1 Stunde)}\trennlinie \kreaturinfo{Quelle}{\href{https://dsaforum.de/viewtopic.php?p=1887303\#p1887303}{Pandämonium}}}}
\newcommand{\kreaturdetailsholtgothar}{\kreatur{Sholtgothar}{die zähe Glut}{gfx/kreaturen/elementar}{\kreaturkampfwerte{12}{13}{5}{4}\trennlinie \kreaturinfo{Aura}{Hitze, Zähigkeit (28) alle 4 Initiativephasen, 1 Wunde}\kreaturinfo{Tarnung}{Gestein}\kreaturvorteile{Schreckgestalt I}\trennlinie \kreaturwaffe{Gluthand}{0}{3}{19}{3W6+0}{Nachbrennen, Umklammern ({-}4), Rüstungsbrecher}\kreaturwaffe{Flammenaugen}{16}{}{}{1W6+0}{Nachbrennen (GE, 20)}\kreaturkampfvorteile{Niederwerfen, Standfest}\trennlinie \kreaturattribute{CH 2, FF 2, GE 8, IN 8, KK 30, KL 8, KO 32, MU 16}\kreaturfertigkeiten{Willenskraft 14, Einschüchtern 16, Wachsamkeit 8}\trennlinie \kreaturinfo{AsP}{64}\kreaturinfo{Feuer}{12 (alle)}\trennlinie \kreaturinfo{Feuriges Umklammern}{Sholtgothar ist heiß wie Glut (400{-}800 °C, Inervall 1, 1 Wunde, Berührung 3W6SP/Akt.).}\kreaturinfo{Waffen schmelzend}{Waffen, die Sholtgothar treffen, erhalten eine Beschädigung. Sholtgothars Paraden reichen dafür nicht aus.}\kreaturinfo{Spezialfähigkeiten}{Zerberstender ({-}X, Explosion(XW6))}\trennlinie \kreaturinfo{Quelle}{\href{https://www.orkenspalter.de/filebase/index.php?file/2829-hilberts-bestiarium/}{Hilberts Bestarium}}}}
\newcommand{\kreaturdetailskelett}{\kreatur{Skelett}{schwaches untotes Kanonenfutter}{gfx/kreaturen/untot}{\kreaturkampfwerte{4}{5}{3}{2}\trennlinie \kreaturinfo{Astralsinn}{Astralsinn erlaubt es der Kreatur, ihre Umgebung magisch wahrzunehmen. Sie erleidet keine Abzüge durch schlechte Sicht. Der Astralsinn kann durch Antimagie, zum Beispiel Hellsicht trüben in der Modifikation Magie unterdrücken, gestört werden. Die Schwierigkeit dafür liegt mindestens bei 20, bei mächtigen Wesen deutlich höher.\newline%
}\kreaturinfo{Resistenz II}{Stichwaffen}\kreaturvorteile{Rudel}\trennlinie \kreaturwaffe{Hände}{0}{5}{8}{1W6+2}{}\kreaturwaffe{Streitaxt}{1}{5}{8}{2W6+3}{}\kreaturwaffe{Stoßspeer}{2}{4}{7}{3W6+4}{}\trennlinie \kreaturattribute{GE 4, KK 8}\kreaturfertigkeiten{Pirschen 8, Untertauchen 8, Wachsamkeit 5}\trennlinie \kreaturinfo{Beschwörung}{Skelettarius, Totes handle!}\kreaturinfo{Dienste}{Burggraben auffüllen+4 (1 Stunde), Kampf+0 (1 Minute), Wache{-}4 (1 Tag (mit Totes handle! permanent))}\trennlinie \kreaturinfo{Quelle}{\href{https://ilarisblog.wordpress.com/downloads/}{Ilaris Regeln}}}}
\newcommand{\kreaturdetailskelettruestung}{\kreatur{Skelett mit Rüstung}{schwaches untotes Kanonenfutter}{gfx/kreaturen/untot}{\kreaturkampfwerte{5}{5}{2}{1}\trennlinie \kreaturinfo{Astralsinn}{Astralsinn erlaubt es der Kreatur, ihre Umgebung magisch wahrzunehmen. Sie erleidet keine Abzüge durch schlechte Sicht. Der Astralsinn kann durch Antimagie, zum Beispiel Hellsicht trüben in der Modifikation Magie unterdrücken, gestört werden. Die Schwierigkeit dafür liegt mindestens bei 20, bei mächtigen Wesen deutlich höher.\newline%
}\kreaturinfo{Resistenz II}{Stichwaffen}\kreaturvorteile{Rudel}\trennlinie \kreaturwaffe{Hände}{0}{5}{8}{1W6+2}{}\kreaturwaffe{Streitaxt}{1}{5}{8}{2W6+3}{}\kreaturwaffe{Stoßspeer}{2}{4}{7}{3W6+4}{}\trennlinie \kreaturattribute{GE 4, KK 8}\kreaturfertigkeiten{Pirschen 8, Untertauchen 8, Wachsamkeit 5}\trennlinie \kreaturinfo{Beschwörung}{Skelettarius, Totes handle!}\kreaturinfo{Dienste}{Burggraben auffüllen+4 (1 Stunde), Kampf+0 (1 Minute), Wache{-}4 (1 Tag (mit Totes handle! permanent))}\trennlinie \kreaturinfo{Quelle}{\href{https://ilarisblog.wordpress.com/downloads/}{Ilaris Regeln}}}}
\newcommand{\kreaturdetailsoeldner}{\kreatur{Söldner}{verlässlicher Mietling oder undisziplinierter Halsabschneider}{gfx/kreaturen/humanoid}{\kreaturkampfwerte{5}{5}{5}{3}\trennlinie \kreaturwaffe{Säbel}{1}{11}{11}{2W6+3}{}\kreaturwaffe{Streitaxt}{1}{11}{11}{2W6+3}{Kopflastig}\kreaturwaffe{Holzschild}{0}{12}{12}{1W6+0}{Schild, Stumpf}\kreaturwaffe{Speer}{2}{11}{11}{3W6+0}{Wendig, Zweihändig}\kreaturwaffe{Hellebarde}{2}{11}{11}{2W6+3}{Kopflastig, Rüstungsbrechend, Zweihändig}\kreaturkampfvorteile{Schildkampf I, Kraftvoller Kampf I, Schneller Kampf I, Niederwerfen, Rüstungsgewöhnung}\trennlinie \kreaturattribute{CH 6, FF 6, GE 8, IN 6, KK 10, KL 4, KO 10, MU 10}\kreaturfertigkeiten{Pirschen 8, Wachsamkeit 6, Zähigkeit 8}\trennlinie \kreaturinfo{Info}{Der Söldner steht beispielhaft für alle humanoiden Gegner mit gewisser Kampferfahrung. Er stellt für unerfahrene Kämpfercharak{-} tere eine gewisse Herausforderung dar und ist Nichtkämpfern überlegen. Kampfstile sind nicht in die Kampfwerte eingerechnet.}\trennlinie \kreaturinfo{Quelle}{\href{https://ilarisblog.wordpress.com/downloads/}{Ilaris Regeln}}}}
\newcommand{\kreaturdetailsoeldnerkommandant}{\kreatur{Söldnerkommandant}{charismatischer Befehlshaber einer größeren Einheit}{gfx/kreaturen/humanoid}{\kreaturkampfwerte{5}{13}{5}{10}\trennlinie \kreaturvorteile{Willensstark I, Willensstark II}\trennlinie \kreaturwaffe{Tuzakmesser}{2}{14}{16}{2W6+5}{Wendig, Zweihändig}\kreaturwaffe{Kurzschwert}{0}{14}{16}{2W6+0}{Wendig}\kreaturwaffe{Faust}{0}{8}{8}{1W6+1}{Stumpf, Zerbrechlich}\kreaturkampfvorteile{Schneller Kampf II, Defensiver Kampfstil, Kommando: Formiert Euch!, Kommando: Haltet Stand!, Kampfreflexe, Kommando: Keine Gefangenen!, Niederwerfen, Rüstungsgewöhnung, Standfest, Sturmangriff, Durchatmen}\trennlinie \kreaturattribute{CH 16, FF 6, GE 12, IN 12, KK 8, KL 8, KO 12, MU 14}\kreaturfertigkeiten{Anführen 15, Einschüchtern 15, Menschenkenntnis 12, Wachsamkeit 12, Überreden 13}\kreaturinfo{Profane Vorteile}{Abgehärtet I, Soziale Anpassungsfähigkeit, Eindrucksvoll I, Eindrucksvoll II, Vorausschauend I, Vorausschauend II}\trennlinie \kreaturinfo{Quelle}{\href{None}{None}}}}
\newcommand{\kreaturdetailsoeldnerveteran}{\kreatur{Söldnerveran}{verlässlicher Mietling oder undisziplinierter Halsabschneider}{gfx/kreaturen/humanoid}{\kreaturkampfwerte{5}{5}{5}{8}\trennlinie \kreaturwaffe{Säbel}{1}{12}{13}{2W6+4}{}\kreaturwaffe{Streitaxt}{1}{12}{13}{2W6+4}{Kopflastig}\kreaturwaffe{Holzschild}{0}{13}{14}{1W6+1}{Schild, Stumpf}\kreaturwaffe{Speer}{2}{12}{13}{3W6+1}{Wendig, Zweihändig}\kreaturwaffe{Hellebarde}{2}{11}{13}{2W6+3}{Kopflastig, Rüstungsbrechend, Zweihändig}\kreaturkampfvorteile{Schildkampf, Kraftvoller Kampf II, Schneller Kampf II, Niederwerfen, Rüstungsgewöhnung, Kampfreflexe, Standfest, Waffenloser Kampf, Durchatmen}\trennlinie \kreaturattribute{CH 8, FF 7, GE 8, IN 7, KK 11, KL 6, KO 10, MU 11}\kreaturfertigkeiten{Pirschen 8, Wachsamkeit 6, Zähigkeit 8}\trennlinie \kreaturinfo{Info}{Der Söldner steht beispielhaft für alle humanoiden Gegner mit gewisser Kampferfahrung. Er stellt für unerfahrene Kämpfercharak{-} tere eine gewisse Herausforderung dar und ist Nichtkämpfern überlegen. Kampfstile sind nicht in die Kampfwerte eingerechnet.}\trennlinie \kreaturinfo{Quelle}{\href{https://ilarisblog.wordpress.com/downloads/}{Ilaris Regeln}}}}
\newcommand{\kreaturdetailsordul}{\kreatur{Sordul}{Der Zersetzer; niederer Diener Belzhorashs}{gfx/kreaturen/daemon}{\kreaturkampfwerte{7}{6}{8}{6}\trennlinie \kreaturinfo{Aura}{Verseuchung, Zähigkeit (16) alle 4 INI: phasen, 1 Erschöpfung}\kreaturinfo{Dämonische Säure}{Waffen und Rüstungen, die Holz beinhalten, gelten gegen den Dämon als zerbrechlich.}\kreaturvorteile{}\trennlinie \kreaturwaffe{Krallen}{1}{10}{10}{2W6+4}{Nachbrennen (Säure)}\kreaturwaffe{Biss}{0}{2}{12}{3W6+0}{}\trennlinie \kreaturinfo{Beschwörung}{Invocatio}\kreaturinfo{Dienste}{Gegenstand auflösen+4, Wache+0 (1 Woche), Kampf{-}4 (1 Minute)}\trennlinie \kreaturinfo{Quelle}{\href{https://dsaforum.de/viewtopic.php?p=1887303\#p1887303}{Pandämonium}}}}
\newcommand{\kreaturdetailspektral}{\kreatur{Spektral}{mächtiger, eiskalter Kriegsgeist}{gfx/kreaturen/geist}{\kreaturkampfwerte{4}{16}{16}{10}\trennlinie \kreaturinfo{Eingeschränkt}{Kann nicht über größere Mengen an Eisen laufen {-} bspw. Nägel. Dadurch lässt sich auch eine Art Schutzkreis/Bannkreis bilden}\kreaturinfo{Immunität}{alles Schadensarten außer Feuer}\kreaturinfo{Unsichtbarkeit}{Ausnahme: Ziel der Jagd}\kreaturinfo{Verwundbarkeit IV}{Feuer, dass mindestens so heiß brennt wie ein Ignifaxius}\kreaturvorteile{}\trennlinie \kreaturwaffe{Berührung}{0}{8}{18}{3W6+0}{Erfrierend, Rüstungsbrechend, Stumpf, Todesstoß}\trennlinie \kreaturattribute{CH 4, FF 4, GE 20, IN 20, KK 30, KL 12, KO 4, MU 20}\kreaturfertigkeiten{Aktrobatik 18, Einschüchtern 10, Klettern 18, Wachsamkeit 22, Willenskraft 10}\trennlinie \kreaturinfo{Info}{Der Spektral entsteht nur bei der Erzeugung eines Meharus{-}Ogo{-}Distillat und kann nicht beschworen werden. Dabei wird ein lebendes Kulturschaffendes Wesen zum Spektral. Die kognitiven Fertigkeiten und Talente bleiben erhalten, jedoch ist der Spektral stets davon erfüllt, sein Revier mit allen Mitteln gegen Alle zu verteidigen.\newline%
}\trennlinie \kreaturinfo{Quelle}{\href{https://www.orkenspalter.de/filebase/index.php?file/2829-hilberts-bestiarium/}{Hilberts Bestarium}}}}
\newcommand{\kreaturdetailspinneschwarm}{\kreatur{Schwarmspinnen Schwarm}{Heimtückische Lauerer südlicher Wälder}{gfx/kreaturen/tier}{\kreaturkampfwerte{2}{0}{1}{1}\trennlinie \kreaturinfo{Resistenz II}{Stichwaffen}\kreaturinfo{Verwundbarkeit IV}{Flächenschaden}\kreaturvorteile{Schmerzimmun}\trennlinie \kreaturwaffe{Einspinnen}{0}{4}{12}{0W6+0}{Umklammern}\kreaturkampfvorteile{Zustätzliche Attacke I}\trennlinie \kreaturfertigkeiten{Wachsamkeit 16, Pirschen 12}\trennlinie \kreaturinfo{Info}{Durch Einspinnen können die Spinnen kumulativ Erschwernisse auf ihre Umklammerung legen, was darstellt dass sie ihr Opfer nach und nach Bewegungsunfähig machen. Ab einer Einschränkung von  {-}12 gilt der Gegner als Kampfunfähig und kann sich nicht mehr selbst befreien. Sobald der Gegner Kampfunfähig ist beginnt der Schwarm ihm langsam das Blut aus zu saugen, dies wird wie ein Gift oder eine Krankheit abgehandelt. Aussaugen: 20,  1 Std.,  1 Std, 1 Wunde, Bis Befreit/Tot\newline%
}\trennlinie \kreaturinfo{Quelle}{\href{https://dsaforum.de/viewtopic.php?f=180&p=1738549\#p1738549}{Bestarium+}}}}
\newcommand{\kreaturdetailstadtwache}{\kreatur{Stadtwache}{stets beschäftigter Hüter von Recht und Ordnung}{gfx/kreaturen/humanoid}{\kreaturkampfwerte{4}{5}{4}{6}\trennlinie \kreaturwaffe{Hellebarde}{2}{7}{8}{2W6+2}{Kopflastig, Rüstungsbrechend, Zweihändig}\kreaturwaffe{Kurzschwert}{0}{9}{10}{2W6+0}{Wendig}\kreaturwaffe{Leichte Armbrust}{20}{}{}{3W6+1}{Zweihändig}\kreaturkampfvorteile{Schneller Kampf I, Standfest}\trennlinie \kreaturattribute{CH 8, FF 6, GE 12, IN 12, KK 6, KL 8, KO 6, MU 8}\kreaturfertigkeiten{Einschüchtern 10, Gebräuche 8, Menschenkenntnis 11}\kreaturinfo{Profane Vorteile}{Vorausschauend I, Vorausschauend II, Scharfsinnig I}\trennlinie \kreaturinfo{Quelle}{\href{https://ilarisblog.wordpress.com/downloads/}{Ilaris Regeln}}}}
\newcommand{\kreaturdetailstreifenhai}{\kreatur{Streifenhai}{Räuber der Meere}{gfx/kreaturen/tier}{\kreaturkampfwerte{9}{4}{8}{4}\trennlinie \kreaturvorteile{Wasserwesen}\trennlinie \kreaturwaffe{Biss}{0}{4}{10}{2W6+2}{Zerbrechlich}\kreaturkampfvorteile{Sturmangriff}\trennlinie \kreaturattribute{GE 14, KK 20, KO 20, MU 12}\kreaturfertigkeiten{Pirschen 8, Schwimmen 16, Wachsamkeit 12, Zähigkeit 14}\trennlinie \kreaturinfo{Quelle}{\href{https://dsaforum.de/viewtopic.php?p=1887303\#p1887303}{Allgemeine Gegner}}}}
\newcommand{\kreaturdetailsumpfranze}{\kreatur{Sumpfranze}{feiges Affenwesen}{gfx/kreaturen/tier}{\kreaturkampfwerte{3}{2}{6}{3}\trennlinie \kreaturwaffe{Biss}{0}{6}{6}{1W6+4}{Zerbrechlich}\kreaturkampfvorteile{Standfest, Sturmangriff}\trennlinie \kreaturattribute{GE 12, KK 4, KO 8, MU 4}\kreaturfertigkeiten{Laufen 20, Pirschen 10, Wachsamkeit 14, Zähigkeit 4}\trennlinie \kreaturinfo{Quelle}{\href{https://dsaforum.de/viewtopic.php?p=1887303\#p1887303}{Allgemeine Gegner}}}}
\newcommand{\kreaturdetailsumupriester}{\kreatur{Sumupriester}{Beschützer der Natur, weiser Ratgeber, unerbittlicher Feind}{gfx/kreaturen/humanoid}{\kreaturkampfwerte{5}{5}{4}{8}\trennlinie \kreaturwaffe{Kampfstab}{2}{9}{9}{1W6+1}{Stumpf, Wendig, Zweihändig}\kreaturwaffe{Vulkanglasdolch}{0}{4}{4}{1W6+1}{Magisch, Unzerstörbar}\trennlinie \kreaturattribute{CH 8, FF 4, GE 4, IN 16, KK 4, KL 16, KO 8, MU 10}\kreaturfertigkeiten{Andergast \& Nostria 15, Einschüchtern 11, Elementarkunde 14, Gifte und Krankheiten 13, Götter und Kulte 15, Menschenkenntnis 10, Nordaventurien 9, Pirschen 8, Wachsamkeit 10, Willenskraft 8}\kreaturinfo{Profane Vorteile}{Scharfsinnig I, Scharfsinnig II}\trennlinie \kreaturinfo{AsP}{38}\kreaturinfo{Dolchrituale}{9 (Blut des Dolches, Ernte des Dolches, Lebenskraft des Dolches)}\kreaturinfo{Einfluss}{11 (Böser Blick)}\kreaturinfo{Hellsicht}{14 (Analys, Odem, Weg des Dolches, Weisung des Dolches)}\kreaturinfo{Humus}{15 (Balsam, Eins mit der Natur, Fesselranken, Herbeirufung des Humus, Zorn der Elemente)}\kreaturinfo{Umwelt}{11 (Nebelwand, Schutz des Dolches, Wettermeisterschaft)}\kreaturinfo{Magische Vorteile}{Astrale Regeneration, Effizientes Zaubern, Flexibles Zaubern, Kontrolliertes Zaubern, Kraftlinienmagie, Unitatio, Tradition der Druiden III}\trennlinie \kreaturinfo{Quelle}{\href{https://ilarisblog.wordpress.com/downloads/}{Ilaris Regeln}}}}
\newcommand{\kreaturdetailtaifelel}{\kreatur{Taifelel}{Die Täuschende Flamme; eingehörnter Diener Agrimoths}{gfx/kreaturen/daemon}{\kreaturkampfwerte{7}{10}{2}{4}\trennlinie \kreaturinfo{Aura}{Hitze, Zähigkeit (16) alle 4 INI: phasen, 1 Wunde}\kreaturvorteile{Immunität (Feuer), Verwundbarkeit I (Wasser)}\trennlinie \kreaturwaffe{Prankenhieb}{1}{12}{12}{2W6+4}{Nachbrennen}\trennlinie \kreaturfertigkeiten{Wachsamkeit 10}\trennlinie \kreaturinfo{AsP}{32}\kreaturinfo{Dämonisch}{12 (Caldofrigo, Ignimorpho)}\trennlinie \kreaturinfo{Beschwörung}{Invocatio}\kreaturinfo{Dienste}{Feuer formen+4, Gegenstand erhitzen+0, Wache{-}4 (1 Woche)}\trennlinie \kreaturinfo{Quelle}{\href{https://dsaforum.de/viewtopic.php?p=1887303\#p1887303}{Pandämonium}}}}
\newcommand{\kreaturdetailtatzelwurm}{\kreatur{Tatzelwurm}{stinkender Vertreter der niederen Drachen; großer Gegner}{gfx/kreaturen/tier}{\kreaturkampfwerte{10}{8}{3}{2}\trennlinie \kreaturinfo{Aura}{Gestank {-}2 auf körperliche Proben (kumulativ) für 1 Tag Zähigkeit (16) jede INI:phase"}\kreaturinfo{Resistenz I}{Feuer}\kreaturvorteile{Schreckgestalt I}\trennlinie \kreaturwaffe{Biss}{1}{4}{15}{4W6+2}{Zerbrechlich}\kreaturwaffe{Klauen}{1}{6}{12}{3W6+1}{Niederwerfen ({-}4)}\kreaturwaffe{Schwanz}{2}{4}{10}{2W6+1}{Niederwerfen, Flächenangriff (180° hinter dem Wurm)}\kreaturkampfvorteile{Zusätzliche Attacke I, Standfest}\trennlinie \kreaturattribute{KK 26, KO 32}\kreaturfertigkeiten{Pirschen 0, Wachsamkeit 8}\trennlinie \kreaturinfo{Quelle}{\href{https://ilarisblog.wordpress.com/downloads/}{Ilaris Regeln}}}}
\newcommand{\kreaturdetailthalon}{\kreatur{Thalon}{Das Schwarze Wiesel; niederer Diener Belshirashs}{gfx/kreaturen/daemon}{\kreaturkampfwerte{3}{10}{10}{8}\trennlinie \kreaturvorteile{Regeneration I, Zusätzliche Attacke I}\trennlinie \kreaturwaffe{Biss}{0}{8}{8}{2W6+4}{}\trennlinie \kreaturfertigkeiten{Laufen 16, Pirschen 16, Sinnenschärfe 14, Wachsamkeit 14}\trennlinie \kreaturinfo{AsP}{24}\kreaturinfo{Dämonisch}{10 (Axxeleratus, Visibili)}\trennlinie \kreaturinfo{Beschwörung}{Invocatio}\kreaturinfo{Dienste}{Wild jagen+4 (1 Tag), Spionage+0 (1 Tag), Opfer suchen und töten{-}4 (1 Tag)}\trennlinie \kreaturinfo{Quelle}{\href{https://dsaforum.de/viewtopic.php?p=1887303\#p1887303}{Pandämonium}}}}
\newcommand{\kreaturdetailthazlaraanji}{\kreatur{Thaz{-}Laranji}{Laraansbrut; niederer Diener Belshirashs}{gfx/kreaturen/daemon}{\kreaturkampfwerte{1}{12}{6}{8}\trennlinie \kreaturinfo{Unsichtbarkeit}{Ausnahme: Ziel der Jagd}\kreaturvorteile{}\trennlinie \kreaturwaffe{Ausweichen}{0}{2}{}{}{}\trennlinie \kreaturfertigkeiten{Untertauchen 12, Betören 16}\trennlinie \kreaturinfo{AsP}{24}\kreaturinfo{Dämonisch}{10 (Levthans Feuer, Traumgestalt)}\trennlinie \kreaturinfo{Beschwörung}{Invocatio}\kreaturinfo{Dienste}{Beschwörer erotische Träume bereiten+4, Regeneration des Opfers stehlen+0, Alpträume erzeugen{-}4 (1 Woche)}\trennlinie \kreaturinfo{Quelle}{\href{https://dsaforum.de/viewtopic.php?p=1887303\#p1887303}{Pandämonium}}}}
\newcommand{\kreaturdetailtigerhai}{\kreatur{Tigerhai}{Räuber der Meere}{gfx/kreaturen/tier}{\kreaturkampfwerte{12}{4}{8}{4}\trennlinie \kreaturvorteile{Wasserwesen, großer Gegner}\trennlinie \kreaturwaffe{Biss}{0}{4}{10}{4W6+2}{Zerbrechlich}\kreaturkampfvorteile{Sturmangriff}\trennlinie \kreaturattribute{GE 20, KK 26, KO 26, MU 18}\kreaturfertigkeiten{Pirschen 8, Schwimmen 16, Wachsamkeit 12, Zähigkeit 14}\trennlinie \kreaturinfo{Quelle}{\href{https://dsaforum.de/viewtopic.php?p=1887303\#p1887303}{Allgemeine Gegner}}}}
\newcommand{\kreaturdetailtrachrhabaar}{\kreatur{Trachrhabaar}{Die Näherin der Leiber; eingehörnte Dienerin Asfaloths, großer Gegner}{gfx/kreaturen/daemon}{\kreaturkampfwerte{13}{18}{6}{4}\trennlinie \kreaturinfo{Immunität}{profan}\kreaturvorteile{Regeneration II, Schreckgestalt II}\trennlinie \kreaturwaffe{Klauen}{1}{6}{14}{2W6+4}{Mutation}\kreaturwaffe{Tritt}{1}{2}{10}{2W6+4}{Niederwerfen}\trennlinie \kreaturfertigkeiten{Pirschen 10, Wachsamkeit 14}\trennlinie \kreaturinfo{AsP}{64}\kreaturinfo{Dämonisch}{14 (Balsam, Chimaeroform)}\trennlinie \kreaturinfo{Beschwörung}{Invocatio}\kreaturinfo{Dienste}{Zusammennähen einer schwachen Chimäre+4, Zusammennähen einer nützlichen Chimäre+0, Heilung des Beschwörers mit Milch{-}4}\trennlinie \kreaturinfo{Quelle}{\href{https://dsaforum.de/viewtopic.php?p=1887303\#p1887303}{Pandämonium}}}}
\newcommand{\kreaturdetailtralloper}{\kreatur{Tralloper}{kampferprobtes Reittier aus den Ställen Elenvinas; großer Gegner}{gfx/kreaturen/tier}{\kreaturkampfwerte{16}{6}{8}{1}\trennlinie \kreaturwaffe{Tritt}{1}{5}{12}{2W6+4}{Niederwerfen}\kreaturwaffe{Biss}{0}{2}{12}{1W6+5}{Zerbrechlich}\kreaturkampfvorteile{Rüstungsgewöhnung, Standfest, im Reiterkampf}\trennlinie \kreaturattribute{GE 8, KK 26, KO 24, MU 12}\kreaturfertigkeiten{Wachsamkeit 10, Zähigkeit 10}\trennlinie \kreaturinfo{Info}{Im Reiterkampf sind hauptsächlich WS:, GS: und TP:  bedeutend, da AT: und VT: auf den PW Reiten abgelegt werden.}\trennlinie \kreaturinfo{Quelle}{\href{https://ilarisblog.wordpress.com/downloads/}{Ilaris Regeln}}}}
\newcommand{\kreaturdetailtroll}{\kreatur{Troll}{jähzorniger Überlebender einer uralten Kultur; großer Gegner}{gfx/kreaturen/humanoid}{\kreaturkampfwerte{12}{8}{7}{3}\trennlinie \kreaturwaffe{Faust}{1}{9}{15}{4W6+1}{Niederwerfen ({-}4), Stumpf}\kreaturwaffe{Keule}{2}{8}{14}{5W6+4}{Niederwerfen ({-}8), Stumpf}\kreaturwaffe{Axt}{2}{9}{15}{6W6+4}{Niederwerfen ({-}8)}\kreaturkampfvorteile{Ausfall, Kraftvoller Kampf III, Atemtechnik, Sturmangriff}\trennlinie \kreaturattribute{KK 32, KL 6, KO 34, MU 12}\kreaturfertigkeiten{Laufen 14, Pirschen 8, Wachsamkeit 12}\trennlinie \kreaturinfo{Quelle}{\href{https://ilarisblog.wordpress.com/downloads/}{Ilaris Regeln}}}}
\newcommand{\kreaturdetailtuuramash}{\kreatur{Tuur{-}Amash}{Die Schwarze Kröte; siebengehörnter Diener Agrimoth, großer Gegner}{gfx/kreaturen/daemon}{\kreaturkampfwerte{12}{18}{2}{10}\trennlinie \kreaturinfo{Krötensekret}{Wenn du dem Tuur{-}Amash mit einem Nahkampfangriff mindestens 2 Kratzer zufügt, musst du eine GE{-}Probe (20) ablegen. Misslingt die Probe, erleidest du 4W6 SP.\newline%
}\kreaturvorteile{Regeneration I}\trennlinie \kreaturwaffe{Sprungangriff}{16}{10}{20}{4W6+2}{Mutation, Niederwerfen ({-}8)}\kreaturwaffe{Zunge}{8}{10}{20}{3W6+2}{Fesseln, Mutation, Umklammern ({-}4, 20)}\kreaturkampfvorteile{Sturmangriff}\trennlinie \kreaturfertigkeiten{Sinnenschärfe 20, Wachsamkeit 20}\trennlinie \kreaturinfo{AsP}{64}\kreaturinfo{Dämonisch}{16 (Chimaeroform, Haselbusch, Krötensprung, Rikais Fluch, Tlalucs Odem, Sumpfstrudel)}\trennlinie \kreaturinfo{Beschwörung}{Invocatio}\kreaturinfo{Dienste}{Giftige Dornenhecken wuchern lassen+4, Felder pervertieren+0, Erschaffung von Daimoniden{-}4}\trennlinie \kreaturinfo{Quelle}{\href{https://dsaforum.de/viewtopic.php?p=1887303\#p1887303}{Pandämonium}}}}
\newcommand{\kreaturdetailtuzaker}{\kreatur{Tuzaker}{loyale Hundgefährten}{gfx/kreaturen/tier}{\kreaturkampfwerte{4}{5}{8}{4}\trennlinie \kreaturinfo{Tierempathie}{Maraskantarantel}\kreaturinfo{Immunität}{Maraskantarantel{-}Gift}\kreaturvorteile{}\trennlinie \kreaturwaffe{Biss}{0}{4}{10}{2W6+0}{}\kreaturkampfvorteile{Niederwerfen, Standfest, Sturmangriff}\trennlinie \kreaturattribute{CH 8, FF 0, GE 10, IN 8, KK 8, KL {-}10, KO 8, MU 12}\kreaturfertigkeiten{Laufen 24, Akrobatik 16, Zähigkeit 4, Pirschen 12, Einschüchtern 6, Menschenkenntnis 7, Sinnesschärfe 16, Wachsamkeit 16}\trennlinie \kreaturinfo{Quelle}{\href{https://www.orkenspalter.de/filebase/index.php?file/2829-hilberts-bestiarium/}{Hilberts Bestarium}}}}
\newcommand{\kreaturdetailudapoth}{\kreatur{Udapoth}{Das Grauen aus dem Wasser, der Tentakelkopf; ein niederer Diener Charyptoroths}{gfx/kreaturen/daemon}{\kreaturkampfwerte{6}{12}{4}{6}\trennlinie \kreaturinfo{Verwundbarkeit I}{Feuer}\kreaturvorteile{Amphibisches Wesen, Regeneration I}\trennlinie \kreaturwaffe{Hieb}{1}{10}{12}{2W6+1}{Umklammern ({-}4, 12)}\kreaturwaffe{Biss}{0}{2}{10}{3W6+1}{}\kreaturkampfvorteile{Niederwerfen}\trennlinie \kreaturfertigkeiten{Pirschen 10, Wachsamkeit 10}\trennlinie \kreaturinfo{AsP}{24}\kreaturinfo{Dämonisch}{10 (Aquafaxius)}\trennlinie \kreaturinfo{Beschwörung}{Invocatio}\kreaturinfo{Dienste}{Wache+4 (1 Woche), Schutz des Beschwörers+0 (1 Tag), Opfer suchen und zum Beschwörer verschleppen{-}4 (1 Tag)}\trennlinie \kreaturinfo{Quelle}{\href{https://dsaforum.de/viewtopic.php?p=1887303\#p1887303}{Pandämonium}}}}
\newcommand{\kreaturdetailugrabaan}{\kreatur{Ugrabaan}{Der Chimärenschlund; dreigehörnter Diener Asfaloths, großer Gegner}{gfx/kreaturen/daemon}{\kreaturkampfwerte{13}{18}{6}{6}\trennlinie \kreaturvorteile{Zusätzliche Attacke II}\trennlinie \kreaturwaffe{Pseudopode}{2}{6}{12}{2W6+4}{Umklammern ({-}4, 12)}\kreaturwaffe{Biss}{0}{2}{14}{4W6+4}{}\kreaturkampfvorteile{Niederwerfen}\trennlinie \kreaturfertigkeiten{Pirschen 18, Wachsamkeit 14}\trennlinie \kreaturinfo{AsP}{64}\kreaturinfo{Dämonisch}{14 (Chimaeroform)}\trennlinie \kreaturinfo{Beschwörung}{Invocatio}\kreaturinfo{Dienste}{Einem Opfer auflauern und es verschlingen+4 (1 Tag), Zwei Lebewesen fressen und zu einer nützlichen Chimäre verschmelzen+0, Ein Dorf überfallen und alles auf seinem Weg fressen{-}4 (1 Stunde)}\trennlinie \kreaturinfo{Quelle}{\href{https://dsaforum.de/viewtopic.php?p=1887303\#p1887303}{Pandämonium}}}}
\newcommand{\kreaturdetailumdoreel}{\kreatur{Umdoreel}{Der Meister der Jagd; dreigehörnter Diener Belshirashs, großer Gegner}{gfx/kreaturen/daemon}{\kreaturkampfwerte{10}{18}{7}{4}\trennlinie \kreaturinfo{Aura}{Kälte, Zähigkeit (20), alle 4 INI: phasen, 1Wunde}\kreaturvorteile{Regeneration I, Schreckgestalt I, Zusätzliche Attacke II}\trennlinie \kreaturwaffe{Krallen}{1}{9}{15}{4W6+1}{Niederwerfen ({-}4)}\kreaturwaffe{Keule}{2}{8}{14}{5W6+4}{Niederwerfen ({-}8), Stumpf}\kreaturwaffe{Biss}{0}{2}{16}{5W6+2}{Fesseln, Mutation, Umklammern ({-}4, 20)}\kreaturkampfvorteile{Befreiungsschlag, Kommando: Formiert Euch!, Kommando: Keine Gefangenen!, Sturmangriff}\trennlinie \kreaturfertigkeiten{Anführen 16, Pirschen 16, Wachsamkeit 18}\trennlinie \kreaturinfo{Beschwörung}{Invocatio}\kreaturinfo{Dienste}{Ziel suchen und töten{-}4 (1 Tag), Mit der Wilden Jagd Angst und Schrecken verbreiten+0 (1 Tag), Kampf+4 (1 Minute)}\trennlinie \kreaturinfo{Quelle}{\href{https://dsaforum.de/viewtopic.php?p=1887303\#p1887303}{Pandämonium}}}}
\newcommand{\kreaturdetailuntoterdrachenbandwurm}{\kreatur{Untoter Drachenbandwurm}{Cestoda Dracophagus}{gfx/kreaturen/untot}{\kreaturkampfwerte{2}{12}{1}{1}\trennlinie \kreaturinfo{Resistenz I}{Feuer, Stich}\kreaturinfo{Magieabsorption}{Zauberern wird 4 AsP pro Initiativephase entzogen. Die entzogenen AsP sind um 1 je volle 8 Schritt Entfernung zum Absorber reduziert}\kreaturinfo{Magienahrung}{regeneriert 1 Wunde pro 2 zugenommene AsP}\kreaturvorteile{}\trennlinie \kreaturwaffe{Biss}{0}{4}{11}{1W6+2}{Entzieht dem Ziel 3W6 AsP pro angerichtete Wunde}\trennlinie \kreaturinfo{Quelle}{\href{https://www.orkenspalter.de/filebase/index.php?file/2829-hilberts-bestiarium/}{Hilberts Bestarium}}}}
\newcommand{\kreaturdetailuridabash}{\kreatur{Uridabash}{Die Hand mit dem Stab; ein vielgehörnter Diener Amazeroths}{gfx/kreaturen/daemon}{\kreaturkampfwerte{12}{24}{4}{8}\trennlinie \kreaturvorteile{Ausweichen in den Limbus, Tarnung, Zusätzliche Attacke I}\trennlinie \kreaturwaffe{Stab}{2}{16}{14}{2W6+6}{Verwirrung (Wundschmerzeffekt, KL (20, I), alle Proben {-}2 für 8 INI{-}Phasen)}\trennlinie \kreaturfertigkeiten{Magiekunde 22, Mythenkunde 22, Menschenkenntnis 16, Rhetorik 20, Überreden 20, Untertauchen 10, Wachsamkeit 18}\trennlinie \kreaturinfo{AsP}{128}\kreaturinfo{Antimagie}{24 (Alle Zauber)}\kreaturinfo{Illusion}{24 (Alle Zauber)}\kreaturinfo{Kraft}{24 (Alle Zauber)}\trennlinie \kreaturinfo{Beschwörung}{Invocatio}\kreaturinfo{Dienste}{Lehrmeister für Zauber+4, Artefakt entzaubern+0, Lehrmeister für magischen Vorteil{-}4}\trennlinie \kreaturinfo{Quelle}{\href{https://dsaforum.de/viewtopic.php?p=1887303\#p1887303}{Pandämonium}}}}
\newcommand{\kreaturdetailusuzoreel}{\kreatur{Usuzoreel}{Der-dich-in-den-Tod-treibt; ein minderer Diener Belshirashs}{gfx/kreaturen/daemon}{\kreaturkampfwerte{4}{10}{}{8}\trennlinie \kreaturinfo{Empfindlichkeit I}{Feuer}\kreaturvorteile{Flieger, Lichtscheu, Regeneration I, Rudel, Schreckgestalt II}\trennlinie \kreaturwaffe{Ausweichen}{0}{8}{}{}{}\trennlinie \kreaturinfo{Beschwörung}{Invocatio}\kreaturinfo{Dienste}{Gegner erschrecken+4 (1 Stunde), Opfer zu Tode hetzen+0 (1 Tag)}\trennlinie \kreaturinfo{Quelle}{\href{https://dsaforum.de/viewtopic.php?p=1887303\#p1887303}{Pandämonium}}}}
\newcommand{\kreaturdetailuttaravha}{\kreatur{UTTARA'VHA}{Die diebische Elster; ein niederer Diener Tasfarelels, sehr kleiner Gegner}{gfx/kreaturen/daemon}{\kreaturkampfwerte{2}{6}{1}{2}\trennlinie \kreaturvorteile{Flieger}\trennlinie \kreaturwaffe{Krallen}{0}{6}{12}{1W6+2}{}\trennlinie \kreaturfertigkeiten{Untertauchen 12, Wachsamkeit 10}\trennlinie \kreaturinfo{AsP}{24}\kreaturinfo{Dämonisch}{10 (Axxeleratus, Gefunden!)}\trennlinie \kreaturinfo{Beschwörung}{Invocatio}\kreaturinfo{Dienste}{Gegenstand suchen und stehlen+4 (1 Tag), Spionage{-}4 (1 Tag)}\trennlinie \kreaturinfo{Quelle}{\href{https://dsaforum.de/viewtopic.php?p=1887303\#p1887303}{Pandämonium}}}}
\newcommand{\kreaturdetailvhatacheor}{\kreatur{Vhatacheor}{Meister des brennenden Wassers; achtgehörnter Diener Charyptoroths, großer Gegner}{gfx/kreaturen/daemon}{\kreaturkampfwerte{13}{20}{}{6}\trennlinie \kreaturvorteile{Regeneration I, Schreckgestalt II, Wasserwesen, Zusätzliche Attacke IV}\trennlinie \kreaturwaffe{Krallen}{8}{16}{20}{4W6+2}{Ertränken, Nachbrennen}\kreaturwaffe{Biss}{0}{2}{14}{3W20+10}{}\trennlinie \kreaturfertigkeiten{Pirschen 16, Schwimmen 24, Wachsamkeit 16}\trennlinie \kreaturinfo{AsP}{128}\kreaturinfo{Dämonisch}{24 (Brenne, Wand aus Flammen, Wasserwand)}\trennlinie \kreaturinfo{Beschwörung}{Invocatio}\kreaturinfo{Dienste}{Schiffbrüchige verbrennen+4, Schiff in Brand setzen+0, Kampf+4 (1 Minute)}\trennlinie \kreaturinfo{Quelle}{\href{https://dsaforum.de/viewtopic.php?p=1887303\#p1887303}{Pandämonium}}}}
\newcommand{\kreaturdetailwaldloewe}{\kreatur{Waldlöwe}{Stolzer König der Khômwüste}{gfx/kreaturen/tier}{\kreaturkampfwerte{5}{2}{8}{4}\trennlinie \kreaturwaffe{Biss}{0}{2}{12}{3W6+2}{Zerbrechlich}\kreaturwaffe{Prankenhieb}{1}{6}{14}{2W6+2}{Wendig}\kreaturkampfvorteile{Doppelangriff, Niederwerfen, Standfest, Sturmangriff}\trennlinie \kreaturattribute{GE 18, KK 22, KO 18, MU 14}\kreaturfertigkeiten{Laufen 20, Pirschen 18, Wachsamkeit 14, Zähigkeit 12}\trennlinie \kreaturinfo{Quelle}{\href{https://dsaforum.de/viewtopic.php?p=1887303\#p1887303}{Allgemeine Gegner}}}}
\newcommand{\kreaturdetailwaldschrat}{\kreatur{Waldschrat}{baumartiger Hüter der Wälder; großer Gegner}{gfx/kreaturen/humanoid}{\kreaturkampfwerte{10}{10}{6}{3}\trennlinie \kreaturwaffe{Asthieb}{1}{6}{15}{5W6+0}{Niederwerfen ({-}4)}\kreaturkampfvorteile{Kraftvoller Kampf III, Atemtechnik, Standfest}\trennlinie \kreaturattribute{KK 32, KO 34, MU 12}\kreaturfertigkeiten{Laufen 12, Pirschen 24, Wachsamkeit 14}\trennlinie \kreaturinfo{AsP}{16}\kreaturinfo{Humus}{12 (Balsam, Fesselranken, Leib des Humus, Ruhe Körper, Wand aus Dornen, Weisheit der Bäume)}\trennlinie \kreaturinfo{Quelle}{\href{https://ilarisblog.wordpress.com/downloads/}{Ilaris Regeln}}}}
\newcommand{\kreaturdetailwaldschratnichtmagisch}{\kreatur{Nichtmagischer Waldschrat}{baumartiger Hüter der Wälder; großer Gegner}{gfx/kreaturen/tier}{\kreaturkampfwerte{10}{10}{6}{3}\trennlinie \kreaturinfo{Tarnung}{im Wald}\kreaturinfo{Verwundbarkeit I}{Feuer}\kreaturvorteile{}\trennlinie \kreaturwaffe{Asthieb}{1}{6}{15}{5W6+0}{Niederwerfen ({-}4)}\kreaturkampfvorteile{Kraftvoller Kampf III, Atemtechnik, Standfest}\trennlinie \kreaturattribute{KK 32, KO 34, MU 12}\kreaturfertigkeiten{Laufen 12, Pirschen 24, Wachsamkeit 14}\trennlinie \kreaturinfo{Quelle}{\href{https://ilarisblog.wordpress.com/downloads/}{Ilaris Regeln}}}}
\newcommand{\kreaturdetailwaldspinne}{\kreatur{Waldspinne}{Große Jagdspinne aus Kavernen und Wäldern}{gfx/kreaturen/tier}{\kreaturkampfwerte{5}{10}{1}{2}\trennlinie \kreaturinfo{Angepasst I (Dunkelheit)}{Angepasst I/II (Umgebung) Durch deine Spezies oder langjährige Erfahrung hast du dich an eine bestimmte Umgebung oder Umweltbedingung gewöhnt. Abzüge durch diese Umgebung (Beispiele auf S. 38), insbesondere im Kampf, sinken für dich um eine/zwei Stufen. Die Kosten für Angepasst legt der Spielleiter fest, wobei er sich an der Häufigkeit der Umgebung orientieren sollte. Zu allgemein gefasste Umgebungen wie „unsicherer Untergrund“ sollte er nicht zulassen. Beispiele für Angepasst sind: • Dunkelheit: verringert Abzüge durch schlechte Lichtverhältnisse (40 EP pro Stufe) • Schnee: verringert Abzüge durch schneebedeckten oder eisigen Untergrund (20 EP pro Stufe) • Wasser: verringert Abzüge durch knie{-} oder hüfttiefes Wasser und unter Wasser (20 EP pro Stufe) • Wald: verringert Abzüge durch Wurzeln, Gestrüpp und dichtes Unterholz (40 EP pro Stufe) Voraussetzungen: keine/Angepasst I Nachkauf: häufig/selten\newline%
}\kreaturvorteile{}\trennlinie \kreaturwaffe{Biss}{0}{4}{12}{2W6+0}{Zerbrechlich, Gift}\kreaturkampfvorteile{Standfest}\trennlinie \kreaturattribute{GE 6, KK 8, KO 10, MU 4}\kreaturfertigkeiten{Wachsamkeit 12, Pirschen 12, Laufen 8, Zähigkeit 8, Klettern 10, Akrobatik 8}\trennlinie \kreaturinfo{Quelle}{\href{https://dsaforum.de/viewtopic.php?f=180&p=1738549\#p1738549}{Bestarium+}}}}
\newcommand{\kreaturdetailwalkuer}{\kreatur{Walkür}{pflichtbewusster halb-materieller Schutzgeist}{gfx/kreaturen/geist}{\kreaturkampfwerte{5}{6}{5}{10}\trennlinie \kreaturinfo{Resistenz}{profan}\kreaturvorteile{}\trennlinie \kreaturwaffe{Schwert}{1}{10}{10}{2W6+2}{Wendig}\kreaturwaffe{Schild}{0}{12}{15}{1W6{-}1}{Schild, Stumpf}\kreaturkampfvorteile{Ausfall, Niederwerfen, Schildkampf III}\trennlinie \kreaturattribute{CH 4, FF 3, GE 14, IN 3, KK 8, KL 2, KO 6, MU 10}\kreaturfertigkeiten{Einschüchtern 10, Pirschen 6, Willenskraft 8}\trennlinie \kreaturinfo{Info}{Beseelte Waffe: Der Walkür kann für eine Stunde in eine Waffe einfahren (WM +1, TP +1, 1x kostenlos Wurf wiederholen, magisch)\newline%
Varianten:  \newline%
Großer starker Geisterkrieger (WS 7/8, +4 TP, +2 AT/VT IN: 16 MU: 16, Gegenhalten, Aufmerksamkeit, Groß)\newline%
Keine Immunität (profan)}\trennlinie \kreaturinfo{Quelle}{\href{https://www.orkenspalter.de/filebase/index.php?file/2829-hilberts-bestiarium/}{Hilberts Bestarium}}}}
\newcommand{\kreaturdetailweisserhetzer}{\kreatur{Weißer Hetzer}{zweiköpfige Chimäre aus Wolf und Dämon}{gfx/kreaturen/daimonid}{\kreaturkampfwerte{4}{7}{8}{5}\trennlinie \kreaturinfo{Angepasst II (Dunkelheit)}{Angepasst I/II (Umgebung) Durch deine Spezies oder langjährige Erfahrung hast du dich an eine bestimmte Umgebung oder Umweltbedingung gewöhnt. Abzüge durch diese Umgebung (Beispiele auf S. 38), insbesondere im Kampf, sinken für dich um eine/zwei Stufen. Die Kosten für Angepasst legt der Spielleiter fest, wobei er sich an der Häufigkeit der Umgebung orientieren sollte. Zu allgemein gefasste Umgebungen wie „unsicherer Untergrund“ sollte er nicht zulassen. Beispiele für Angepasst sind: • Dunkelheit: verringert Abzüge durch schlechte Lichtverhältnisse (40 EP pro Stufe) • Schnee: verringert Abzüge durch schneebedeckten oder eisigen Untergrund (20 EP pro Stufe) • Wasser: verringert Abzüge durch knie{-} oder hüfttiefes Wasser und unter Wasser (20 EP pro Stufe) • Wald: verringert Abzüge durch Wurzeln, Gestrüpp und dichtes Unterholz (40 EP pro Stufe) Voraussetzungen: keine/Angepasst I Nachkauf: häufig/selten\newline%
}\kreaturvorteile{}\trennlinie \kreaturwaffe{Biss}{1}{9}{11}{2W6+2}{}\kreaturkampfvorteile{Zusätzliche Attacke I, Niederwerfen, Standfest, Sturmangriff}\trennlinie \kreaturattribute{GE 16, KK 8, KO 4}\kreaturfertigkeiten{Laufen 24, Pirschen 12, Wachsamkeit 18}\trennlinie \kreaturinfo{Beschwörung}{Chimaeroform}\kreaturinfo{Dienste}{Spur verfolgen+4 (1 Stunde), Kampf+0 (1 Minute), Hetzen und Töten (1 Stunde, {-}4)+0}\trennlinie \kreaturinfo{Quelle}{\href{https://ilarisblog.wordpress.com/downloads/}{Ilaris Regeln}}}}
\newcommand{\kreaturdetailwidderhyaene}{\kreatur{Widderhyäne}{Widder und Hyäne}{gfx/kreaturen/daimonid}{\kreaturkampfwerte{7}{8}{11}{10}\trennlinie \kreaturvorteile{Meute I}\trennlinie \kreaturwaffe{Biss}{0}{5}{10}{2W6+4}{None}\kreaturwaffe{Hörner}{1}{5}{10}{2W6+2}{Rüstungsbrechend}\kreaturkampfvorteile{Sturmangriff, Niederwerfen}\trennlinie \kreaturattribute{CH 6, FF 4, GE 10, IN 12, KK 8, KL {-}12, KO 9, MU 10}\kreaturfertigkeiten{Laufen 7, Klettern 12, Akrobatik 12, Pirschen 8, Willenskraft 5, Zähigkeit 5, Einschüchtern 6, Sinnesschärfe 9, Wachsamkeit 8}\trennlinie \kreaturinfo{Info}{Fortpflanzungsfähige Widderhyänen können mit einer Erschwernis von {-}16 erzeugt werden.}\trennlinie \kreaturinfo{Quelle}{\href{https://www.orkenspalter.de/filebase/index.php?file/2829-hilberts-bestiarium/}{Hilberts Bestarium}}}}
\newcommand{\kreaturdetailwildschwein}{\kreatur{Wildschwein}{gefräßiger Schädling und goblinisches Reittier}{gfx/kreaturen/tier}{\kreaturkampfwerte{9}{3}{7}{3}\trennlinie \kreaturwaffe{Stoß}{0}{3}{10}{2W6+4}{Niederwerfen}\kreaturkampfvorteile{Standfest, Sturmangriff}\trennlinie \kreaturattribute{GE 8, KK 18, KO 22}\kreaturfertigkeiten{Laufen 12, Pirschen 4, Wachsamkeit 10, Zähigkeit 4}\trennlinie \kreaturinfo{Quelle}{\href{https://ilarisblog.wordpress.com/downloads/}{Ilaris Regeln}}}}
\newcommand{\kreaturdetailwolf}{\kreatur{Wolf}{verbreitetes Rudeltier}{gfx/kreaturen/tier}{\kreaturkampfwerte{4}{2}{8}{2}\trennlinie \kreaturinfo{Angepasst II (Dunkelheit)}{Angepasst I/II (Umgebung) Durch deine Spezies oder langjährige Erfahrung hast du dich an eine bestimmte Umgebung oder Umweltbedingung gewöhnt. Abzüge durch diese Umgebung (Beispiele auf S. 38), insbesondere im Kampf, sinken für dich um eine/zwei Stufen. Die Kosten für Angepasst legt der Spielleiter fest, wobei er sich an der Häufigkeit der Umgebung orientieren sollte. Zu allgemein gefasste Umgebungen wie „unsicherer Untergrund“ sollte er nicht zulassen. Beispiele für Angepasst sind: • Dunkelheit: verringert Abzüge durch schlechte Lichtverhältnisse (40 EP pro Stufe) • Schnee: verringert Abzüge durch schneebedeckten oder eisigen Untergrund (20 EP pro Stufe) • Wasser: verringert Abzüge durch knie{-} oder hüfttiefes Wasser und unter Wasser (20 EP pro Stufe) • Wald: verringert Abzüge durch Wurzeln, Gestrüpp und dichtes Unterholz (40 EP pro Stufe) Voraussetzungen: keine/Angepasst I Nachkauf: häufig/selten\newline%
}\kreaturvorteile{}\trennlinie \kreaturwaffe{Biss}{0}{6}{6}{2W6+0}{Zerbrechlich}\kreaturkampfvorteile{Niederwerfen, Standfest, Sturmangriff}\trennlinie \kreaturattribute{GE 14, KK 6, KO 6, MU 6}\kreaturfertigkeiten{Laufen 24, Pirschen 12, Wachsamkeit 18, Zähigkeit 4}\trennlinie \kreaturinfo{Quelle}{\href{https://ilarisblog.wordpress.com/downloads/}{Ilaris Regeln}}}}
\newcommand{\kreaturdetailwolfrudelfuehrer}{\kreatur{Wolf Rudelführer}{verbreitetes Rudeltier}{gfx/kreaturen/tier}{\kreaturkampfwerte{6}{2}{8}{2}\trennlinie \kreaturinfo{Angepasst II (Dunkelheit)}{Angepasst I/II (Umgebung) Durch deine Spezies oder langjährige Erfahrung hast du dich an eine bestimmte Umgebung oder Umweltbedingung gewöhnt. Abzüge durch diese Umgebung (Beispiele auf S. 38), insbesondere im Kampf, sinken für dich um eine/zwei Stufen. Die Kosten für Angepasst legt der Spielleiter fest, wobei er sich an der Häufigkeit der Umgebung orientieren sollte. Zu allgemein gefasste Umgebungen wie „unsicherer Untergrund“ sollte er nicht zulassen. Beispiele für Angepasst sind: • Dunkelheit: verringert Abzüge durch schlechte Lichtverhältnisse (40 EP pro Stufe) • Schnee: verringert Abzüge durch schneebedeckten oder eisigen Untergrund (20 EP pro Stufe) • Wasser: verringert Abzüge durch knie{-} oder hüfttiefes Wasser und unter Wasser (20 EP pro Stufe) • Wald: verringert Abzüge durch Wurzeln, Gestrüpp und dichtes Unterholz (40 EP pro Stufe) Voraussetzungen: keine/Angepasst I Nachkauf: häufig/selten\newline%
}\kreaturvorteile{}\trennlinie \kreaturwaffe{Biss}{0}{10}{10}{2W6+2}{Zerbrechlich}\kreaturkampfvorteile{Niederwerfen, Standfest, Sturmangriff, Kommando: Haltet Stand!, Kommando: Keine Gefangenen!}\trennlinie \kreaturattribute{GE 14, KK 6, KO 6, MU 6}\kreaturfertigkeiten{Laufen 24, Pirschen 12, Wachsamkeit 18, Zähigkeit 4, Anführen 15}\trennlinie \kreaturinfo{Quelle}{\href{https://ilarisblog.wordpress.com/downloads/}{Ilaris Regeln}}}}
\newcommand{\kreaturdetailwolfsechse}{\kreatur{Wolfsechse}{Sandwolf und Panzerechse}{gfx/kreaturen/daimonid}{\kreaturkampfwerte{6}{13}{8}{3}\trennlinie \kreaturinfo{Angepasst II (Dunkelheit)}{Angepasst I/II (Umgebung) Durch deine Spezies oder langjährige Erfahrung hast du dich an eine bestimmte Umgebung oder Umweltbedingung gewöhnt. Abzüge durch diese Umgebung (Beispiele auf S. 38), insbesondere im Kampf, sinken für dich um eine/zwei Stufen. Die Kosten für Angepasst legt der Spielleiter fest, wobei er sich an der Häufigkeit der Umgebung orientieren sollte. Zu allgemein gefasste Umgebungen wie „unsicherer Untergrund“ sollte er nicht zulassen. Beispiele für Angepasst sind: • Dunkelheit: verringert Abzüge durch schlechte Lichtverhältnisse (40 EP pro Stufe) • Schnee: verringert Abzüge durch schneebedeckten oder eisigen Untergrund (20 EP pro Stufe) • Wasser: verringert Abzüge durch knie{-} oder hüfttiefes Wasser und unter Wasser (20 EP pro Stufe) • Wald: verringert Abzüge durch Wurzeln, Gestrüpp und dichtes Unterholz (40 EP pro Stufe) Voraussetzungen: keine/Angepasst I Nachkauf: häufig/selten\newline%
}\kreaturinfo{Empfindlichkeit I}{Kälte}\kreaturinfo{Tarnung}{Dschungel, Sumpf}\kreaturinfo{Verwundbarkeit I}{Eis}\kreaturvorteile{Meute I}\trennlinie \kreaturwaffe{Biss}{0}{4}{12}{2W6+3}{}\kreaturwaffe{Pranke}{1}{4}{12}{2W6+1}{}\kreaturkampfvorteile{Niederwerfen, Standfest, Sturmangriff}\trennlinie \kreaturattribute{CH 6, FF 4, GE 10, IN 6, KK 8, KL {-}12, KO 8, MU 4}\kreaturfertigkeiten{Laufen 10, Klettern 2, Schwimmen 4, Akrobatik 10, Pirschen 11, Willenskraft 7, Zähigkeit 8, Einschüchtern 10, Sinnesschärfe 8, Wachsamkeit 8}\trennlinie \kreaturinfo{Quelle}{\href{https://www.orkenspalter.de/filebase/index.php?file/2829-hilberts-bestiarium/}{Hilberts Bestarium}}}}
\newcommand{\kreaturdetailwuerger}{\kreatur{Würger}{ansteckender und gieriger spuckender Leichenfresser}{gfx/kreaturen/untot}{\kreaturkampfwerte{5}{9}{6}{6}\trennlinie \kreaturinfo{Empfindlichkeit II}{Sonnenlicht}\kreaturinfo{Immunität}{Einfluss}\kreaturinfo{Angepasst II (Dunkelheit)}{Angepasst I/II (Umgebung) Durch deine Spezies oder langjährige Erfahrung hast du dich an eine bestimmte Umgebung oder Umweltbedingung gewöhnt. Abzüge durch diese Umgebung (Beispiele auf S. 38), insbesondere im Kampf, sinken für dich um eine/zwei Stufen. Die Kosten für Angepasst legt der Spielleiter fest, wobei er sich an der Häufigkeit der Umgebung orientieren sollte. Zu allgemein gefasste Umgebungen wie „unsicherer Untergrund“ sollte er nicht zulassen. Beispiele für Angepasst sind: • Dunkelheit: verringert Abzüge durch schlechte Lichtverhältnisse (40 EP pro Stufe) • Schnee: verringert Abzüge durch schneebedeckten oder eisigen Untergrund (20 EP pro Stufe) • Wasser: verringert Abzüge durch knie{-} oder hüfttiefes Wasser und unter Wasser (20 EP pro Stufe) • Wald: verringert Abzüge durch Wurzeln, Gestrüpp und dichtes Unterholz (40 EP pro Stufe) Voraussetzungen: keine/Angepasst I Nachkauf: häufig/selten\newline%
}\kreaturvorteile{Lichtscheu, Schreckgestalt I}\trennlinie \kreaturwaffe{Unbewaffnet}{1}{5}{5}{1W6+0}{Stumpf, Zerbrechlich, Hochansteckend}\kreaturwaffe{Zähne}{0}{5}{5}{2W6+0}{Zerbrechlich, Ghulgift}\kreaturwaffe{Würgen}{2}{}{}{1W6+1}{Säure (KO(I,20) sonst 1 Wunde), einmalig}\trennlinie \kreaturattribute{GE 8, KK 16, KL 2, KO 20, MU 8}\kreaturfertigkeiten{Pirschen 8, Wachsamkeit 6}\trennlinie \kreaturinfo{Quelle}{\href{https://www.orkenspalter.de/filebase/index.php?file/2829-hilberts-bestiarium/}{Hilberts Bestarium}}}}
\newcommand{\kreaturdetailxamanoth}{\kreatur{Xamanoth}{Das Verbotene Wissen; siebengehörnter Diener Amazeroths}{gfx/kreaturen/daemon}{\kreaturkampfwerte{10}{24}{1}{12}\trennlinie \kreaturvorteile{Ausweichen in den Limbus, Flieger, Lebensraub, Limbusreisender, Zusätzliche Attacke III}\trennlinie \kreaturwaffe{Tentakel}{2}{18}{10}{2W6+4}{}\trennlinie \kreaturfertigkeiten{Magiekunde 24, Mythenkunde 24, Untertauchen 18, Wachsamkeit 16}\kreaturfertigkeiten{Lehrer 3}\trennlinie \kreaturinfo{AsP}{128}\kreaturinfo{Fast Alle}{24}\trennlinie \kreaturinfo{Beschwörung}{Invocatio}\kreaturinfo{Dienste}{Lehrmeister für einen Zauber in borbaradianischer Tradition+4, Weitergabe von geheimem Wissen+0, Lehrmeister für die borbaradianische Tradition{-}4}\trennlinie \kreaturinfo{Info}{Xamanoth beherrscht nahezu alle bekannten Zauberformeln mit einem PW von 24. Jeder Dienst erfordert die Opferung von Eigenblut des Beschwörers. }\trennlinie \kreaturinfo{Quelle}{\href{https://dsaforum.de/viewtopic.php?p=1887303\#p1887303}{Pandämonium}}}}
\newcommand{\kreaturdetailxarrmalk}{\kreatur{Xarrmalk}{Der Vielleibige Aufhetzer; ein sechsgehörnter Diener Lolgramoths}{gfx/kreaturen/daemon}{\kreaturkampfwerte{12}{20}{6}{10}\trennlinie \kreaturvorteile{Ausweichen in den Limbus}\trennlinie \kreaturwaffe{Krallen}{1}{12}{16}{2W6+6}{}\trennlinie \kreaturfertigkeiten{Einschüchtern 16, Gebräuche 20, Menschenkenntnis 18, Rhetorik 24, Überreden 24, Wachsamkeit 16}\trennlinie \kreaturinfo{AsP}{64}\kreaturinfo{Einfluss}{20 (Bannbaladin, Große Gier, Imperavi, Seidenzunge))}\trennlinie \kreaturinfo{Beschwörung}{Invocatio}\kreaturinfo{Dienste}{Menschenmenge aufwiegeln+4 (1 Stunde), Häretische Lehren verbreiten+0 (1 Woche), Umsturz einleiten{-}4 (1 Woche)}\trennlinie \kreaturinfo{Quelle}{\href{https://dsaforum.de/viewtopic.php?p=1887303\#p1887303}{Pandämonium}}}}
\newcommand{\kreaturdetailyaryuraam}{\kreatur{YAR'YURAAM}{Die rastlose Bewegung, der ewige Läufer; zweigehörnter Diener Lolgramoths}{gfx/kreaturen/daemon}{\kreaturkampfwerte{1}{10}{4}{8}\trennlinie \kreaturinfo{Unsichtbarkeit}{Ausnahme: Ziel der Jagd}\kreaturvorteile{Limbusreisender, Besessenheit}\trennlinie \kreaturwaffe{Ausweichen}{0}{4}{}{}{}\trennlinie \kreaturinfo{AsP}{32}\kreaturinfo{Dämonisch}{12 (Axxeleratus)}\trennlinie \kreaturinfo{Beschwörung}{Invocatio}\kreaturinfo{Dienste}{Tier zu Tode hetzen+4, Objekt beseelen und Beschwörer transportieren+0 (1 Woche), Menschliches Opfer suchen und zu Tode hetzen{-}4 (1 Tag)}\trennlinie \kreaturinfo{Info}{Besessenheit: Der Dämon kann von einem Gegenstand wie z.B. eine Hütte oder einem Wesen, meist einem Tier, Besitz ergreifen. Er verliert Limbusreisender, Unsichtbarkeit, Besessenheit, WS, GS und erhält die Eigenschaften/Attacken des Ziels (bei einem Wesen). Einem Objekt wachsen Beine und es erhält eine GS von 8. Bei einem Wesen steigt dessen GS um +4 und es beginnt sofort im Dauerlauf in eine beliebige Richtung zu laufen, bis es an Erschöpfung stirbt.\newline%
}\trennlinie \kreaturinfo{Quelle}{\href{https://dsaforum.de/viewtopic.php?p=1887303\#p1887303}{Pandämonium}}}}
\newcommand{\kreaturdetailyashnatam}{\kreatur{YASH'NATAM}{Das gläserne Klirren auf dem Eis, das Ross Nagrachs; eingehörnter Diener Belshirashs, großer Gegner}{gfx/kreaturen/daemon}{\kreaturkampfwerte{15}{20}{10}{10}\trennlinie \kreaturvorteile{Regeneration I}\trennlinie \kreaturwaffe{Tritt}{1}{6}{14}{3W6+4}{Niederwerfen}\kreaturwaffe{Biss}{0}{2}{12}{2W6+3}{Erfrieren}\kreaturwaffe{Schwanz}{2}{4}{12}{3W6+0}{Niederwerfen}\kreaturkampfvorteile{Sturmangriff, Überrennen}\trennlinie \kreaturfertigkeiten{Laufen 16, Pirschen 12, Wachsamkeit 10}\trennlinie \kreaturinfo{AsP}{32}\kreaturinfo{Dämonisch}{12 (Corpofrigo, Kulminatio)}\trennlinie \kreaturinfo{Beschwörung}{Invocatio}\kreaturinfo{Dienste}{Beschwörer transportieren+4 (1 Tag), Beschwörer in den Kampf tragen+0 (1 Stunde)}\trennlinie \kreaturinfo{Quelle}{\href{https://dsaforum.de/viewtopic.php?p=1887303\#p1887303}{Pandämonium}}}}
\newcommand{\kreaturdetailyashoreel}{\kreatur{Yash'oreel}{Der Zermalmer, die Eiskalte Seele, die Kristallene Mordlust; zweigehörnter Diener Belshirashs}{gfx/kreaturen/daemon}{\kreaturkampfwerte{8}{12}{12}{8}\trennlinie \kreaturinfo{Unsichtbarkeit}{Ausnahme: Ziel der Jagd}\kreaturvorteile{Präsenz, Regeneration I}\trennlinie \kreaturwaffe{Ausweichen}{0}{4}{}{}{}\trennlinie \kreaturfertigkeiten{Pirschen 16, Wachsamkeit 10}\trennlinie \kreaturinfo{AsP}{64}\kreaturinfo{Dämonisch}{14 (Applicatus, Frigifaxius, Glacioflumen, Leib des Eises, Metamorpho, Motoricus, Stillstand)}\trennlinie \kreaturinfo{Beschwörung}{Invocatio}\kreaturinfo{Dienste}{Theriaknadel beseelen und antreiben{-}4 (1 Jahr), Magische Eisfalle auslösen+0, Lawine auslösen+4}\trennlinie \kreaturinfo{Quelle}{\href{https://dsaforum.de/viewtopic.php?p=1887303\#p1887303}{Pandämonium}}}}
\newcommand{\kreaturdetailyelarizel}{\kreatur{YEL'ARIZEL}{Der schlechte Ratgeber, Sämann der Zwietracht; niederer Diener Lolgramoths}{gfx/kreaturen/daemon}{\kreaturkampfwerte{1}{12}{6}{8}\trennlinie \kreaturinfo{Unsichtbarkeit}{Ausnahme: Ziel der Jagd}\kreaturvorteile{Besessenheit}\trennlinie \kreaturwaffe{Ausweichen}{0}{2}{}{}{}\trennlinie \kreaturfertigkeiten{Überreden 14, Menschenkenntnis 12}\trennlinie \kreaturinfo{AsP}{32}\kreaturinfo{Einfluss}{12 (Ängste mehren, Große Gier, Seidenzunge)}\trennlinie \kreaturinfo{Beschwörung}{Invocatio}\kreaturinfo{Dienste}{Beratung von Verbündeten stören+4 (1 Tag), Freunde entzweien+0 (1 Woche)}\trennlinie \kreaturinfo{Info}{Der Dämon kann von einem Wesen Besitz ergreifen. Er verliert WS, GS, Ausweichen, Unsichtbarkeit und erhält die Eigenschaften/Attacken des Opfers. Er flüstert dem Opfer schlechte Ratschläge ein und sorgt für Zwist und Hader. Alternativ nutzt der Dämon eine Eigenheit des Opfers aus, wenn diesem keine Konterprobe Willenskraft (16) gelingt.\newline%
}\trennlinie \kreaturinfo{Quelle}{\href{https://dsaforum.de/viewtopic.php?p=1887303\#p1887303}{Pandämonium}}}}
\newcommand{\kreaturdetailyeolkhardas}{\kreatur{Yeol{-}Khardas}{Der Meister des Bades; viergehörnter Diener Belkelels}{gfx/kreaturen/daemon}{\kreaturkampfwerte{11}{12}{8}{6}\trennlinie \kreaturinfo{Tarnung}{im Wasser}\kreaturvorteile{Amphibisch, Regeneration I, Zusätzliche Attacke I}\trennlinie \kreaturwaffe{Tentakel}{2}{8}{16}{2W6+0}{Umklammern ({-}4, 16), Ersäufen (Ein umklammertes Ziel wird unter Wasser gezogen und erleidet in jeder INI: phase 1 Punkt Erschöpfung. Der Dämon kann bis zu zwei Ziele gleichzeitig unter Wasser halten.\newline%
}\kreaturwaffe{Wasserstrahl}{8}{}{}{2W6+4}{Zurückstoßen, Ertränken}\trennlinie \kreaturfertigkeiten{Gebräuche 16, Menschenkenntnis 14, Pirschen 14, Schwimmen 20, Wachsamkeit 12}\kreaturfertigkeiten{Bader 3}\trennlinie \kreaturinfo{AsP}{64}\kreaturinfo{Dämonisch}{16 (Aquafaxius, Aquasphaero, Caldofrigo, Hartes Schmelze, Macht über den Regen, Mahlstrom, Manifesto, Nebelwand, Nebelleib, Sapefacta, Wand aus Wasser, Wellenlauf)}\trennlinie \kreaturinfo{Beschwörung}{Invocatio}\kreaturinfo{Dienste}{Verführerische Konversation mit den Badenden+4 (1 Stunde), Erotische Wassermassage für mehrere Badende+0 (1 Stunde), Fluten eines Bades mit warmem Wasser{-}4}\trennlinie \kreaturinfo{Quelle}{\href{https://dsaforum.de/viewtopic.php?p=1887303\#p1887303}{Pandämonium}}}}
\newcommand{\kreaturdetailyeti}{\kreatur{Yeti}{großer, haariger Schneeschrat}{gfx/kreaturen/humanoid}{\kreaturkampfwerte{9}{2}{6}{4}\trennlinie \kreaturinfo{Angepasst I (Dunkelheit)}{Angepasst I/II (Umgebung) Durch deine Spezies oder langjährige Erfahrung hast du dich an eine bestimmte Umgebung oder Umweltbedingung gewöhnt. Abzüge durch diese Umgebung (Beispiele auf S. 38), insbesondere im Kampf, sinken für dich um eine/zwei Stufen. Die Kosten für Angepasst legt der Spielleiter fest, wobei er sich an der Häufigkeit der Umgebung orientieren sollte. Zu allgemein gefasste Umgebungen wie „unsicherer Untergrund“ sollte er nicht zulassen. Beispiele für Angepasst sind: • Dunkelheit: verringert Abzüge durch schlechte Lichtverhältnisse (40 EP pro Stufe) • Schnee: verringert Abzüge durch schneebedeckten oder eisigen Untergrund (20 EP pro Stufe) • Wasser: verringert Abzüge durch knie{-} oder hüfttiefes Wasser und unter Wasser (20 EP pro Stufe) • Wald: verringert Abzüge durch Wurzeln, Gestrüpp und dichtes Unterholz (40 EP pro Stufe) Voraussetzungen: keine/Angepasst I Nachkauf: häufig/selten\newline%
}\kreaturinfo{Resistenz}{gegen Kälte}\kreaturvorteile{}\trennlinie \kreaturwaffe{Faust}{1}{6}{12}{2W6+2}{Stumpf}\kreaturwaffe{Keule}{2}{8}{14}{3W6+4}{Stumpf}\kreaturkampfvorteile{Natürliche Rüstung, Niederwerfen, Kraftvoller Kampf III}\trennlinie \kreaturattribute{CH 4, FF 4, GE 8, IN 8, KK 24, KL 2, KO 20, MU 16}\kreaturfertigkeiten{Pirschen 14, Laufen 12, Hoher Norden 16, Klettern 10, Sinnenschärfe 10, Wachsamkeit 14, Zähigkeit 16}\kreaturinfo{Profane Vorteile}{Zerstörerisch I, II}\trennlinie \kreaturinfo{Quelle}{\href{https://dsaforum.de/viewtopic.php?p=1887303\#p1887303}{Allgemeine Gegner}}}}
\newcommand{\kreaturdetailyishazrhi}{\kreatur{Yish'Azrhi}{Die niederhöllische Spottdrossel; zweigehörnte Dienerin Lolgramoths, sehr kleiner Gegner}{gfx/kreaturen/daemon}{\kreaturkampfwerte{1}{6}{1}{8}\trennlinie \kreaturinfo{Unsichtbarkeit}{Ausnahme: Ziel der Jagd}\kreaturvorteile{Flieger, Kritische Konsistenz}\trennlinie \kreaturwaffe{Schnabel}{0}{16}{8}{1W6+0}{}\trennlinie \kreaturfertigkeiten{Fliegen 12, Untertauchen 16, Wachsamkeit 14, Überreden 18}\trennlinie \kreaturinfo{AsP}{32}\kreaturinfo{Einfluss}{12 (Papperlapapp, Schabernack)}\trennlinie \kreaturinfo{Beschwörung}{Invocatio}\kreaturinfo{Dienste}{Vortrag eines Konkurrenten stören+4 (1 Stunde), Ziel zur Weißglut treiben+0 (1 Stunde)}\trennlinie \kreaturinfo{Quelle}{\href{https://dsaforum.de/viewtopic.php?p=1887303\#p1887303}{Pandämonium}}}}
\newcommand{\kreaturdetailylmadath}{\kreatur{Ylmadath}{Das sternenlose Leuchten in der Finsternis; ein niederer Diener Amazeroths}{gfx/kreaturen/daemon}{\kreaturkampfwerte{1}{8}{}{6}\trennlinie \kreaturvorteile{Flieger, Macht der Zauberei, Prophezeien}\trennlinie \kreaturwaffe{Funke*}{2}{2}{8}{2W6+2}{Nachbrennen}\trennlinie \kreaturfertigkeiten{Fliegen 6, Pirschen 10, Wachsamkeit 8, Willenskraft 12}\trennlinie \kreaturinfo{AsP}{24}\kreaturinfo{Dämonisch}{10 (Flirrender Funkelglanz, Madas Spiegel, Orbitarium)}\trennlinie \kreaturinfo{Beschwörung}{Invocatio}\kreaturinfo{Dienste}{Bekannte Person an einem fernen Ort zeigen+4, Unterstützung bei einer Prophezeiung+0 (1 Stunde), Person die Macht der Zauberei verleihen{-}4 (1 Tag)}\trennlinie \kreaturinfo{Info}{Der Dämon kann von einem Wesen Besitz ergreifen. Er verliert Macht der Zauberei, fliegend, WS, Funke und erhält die Eigenschaften/Attacken des Ziels, hat jedoch keinerlei Kontrolle über das Ziel. Das Ziel erhält den Vorteil Zauberer II, die Tradition der Magiedilettanten sowie drei Zauber mit PW 8.\newline%
}\trennlinie \kreaturinfo{Quelle}{\href{https://dsaforum.de/viewtopic.php?p=1887303\#p1887303}{Pandämonium}}}}
\newcommand{\kreaturdetailyonahoh}{\kreatur{YO'NAHOH}{Die Multipode oder der Neunkrake, die Arme aus der Tiefe; zehngehörnter und einzigartiger Diener Charyptoroths, sehr großer Gegner}{gfx/kreaturen/daemon}{\kreaturkampfwerte{{-}1}{24}{}{4}\trennlinie \kreaturinfo{Zusätzliche Attacke IV}{nur Tentakel}\kreaturvorteile{Regeneration II, Schreckgestalt III, Wasserwesen}\trennlinie \kreaturwaffe{Biss (Rumpf)}{1}{4}{16}{3W20+10}{Rüstungsbrechend}\kreaturwaffe{Tentakelhieb}{12}{10}{20}{4W6+2}{Flächenangriff (45°), Niederwerfen ({-}8), Zurückstoßen}\kreaturwaffe{Tentakelklammer}{10}{10}{18}{4W6+0}{Umklammern ({-}4, 32)}\kreaturkampfvorteile{Unaufhaltsam}\trennlinie \kreaturfertigkeiten{Pirschen 16, Schwimmen 20, Wachsamkeit 16}\trennlinie \kreaturinfo{Beschwörung}{Invocatio}\kreaturinfo{Dienste}{Schiff festhalten+4, Schiff oder großes Meerestier vernichten+0, Ort bewachen+4 (1 Woche)}\trennlinie \kreaturinfo{Quelle}{\href{https://dsaforum.de/viewtopic.php?p=1887303\#p1887303}{Pandämonium}}}}
\newcommand{\kreaturdetailyougghatugythot}{\kreatur{Yo'ugghatugythot}{Der Zerreißer der Sphären; ein dreigehörnter, unabhängiger Dämon}{gfx/kreaturen/daemon}{\kreaturkampfwerte{6}{16}{6}{6}\trennlinie \kreaturvorteile{Ausweichen in den Limbus, Limbusreisender}\trennlinie \kreaturwaffe{Krallenhieb}{1}{14}{14}{2W6+4}{}\trennlinie \kreaturfertigkeiten{Pirschen 16, Wachsamkeit 14}\trennlinie \kreaturinfo{AsP}{64}\kreaturinfo{Kraft}{14 (Auge des Limbus, Limbus versiegeln, Oculus Astralis, Planastrale, Transversalis, Verschwindibus)}\trennlinie \kreaturinfo{Beschwörung}{Invocatio}\kreaturinfo{Dienste}{Beschwörer durch den Limbus transportieren+4 (1 Stunde), Gegenstand stehlen und in den Limbus befördern+0, Limbusportal öffnen{-}4}\trennlinie \kreaturinfo{Info}{Im Limubs GS und INI 12 und Regeneration II.}\trennlinie \kreaturinfo{Quelle}{\href{https://dsaforum.de/viewtopic.php?p=1887303\#p1887303}{Pandämonium}}}}
\newcommand{\kreaturdetailystphogorthu}{\kreatur{Yst{-}Phogorthu}{Das hohe Ross; ein zweigehörnter, unabhängiger Dämon, großer Gegner}{gfx/kreaturen/daemon}{\kreaturkampfwerte{15}{10}{10}{10}\trennlinie \kreaturvorteile{Ausweichen in den Limbus}\trennlinie \kreaturwaffe{Hornstoß}{0}{4}{16}{2W6+7}{}\kreaturwaffe{Tritt}{1}{6}{14}{3W6+4}{Niederwerfen}\kreaturwaffe{Biss}{0}{2}{12}{2W6+3}{}\kreaturkampfvorteile{Sturmangriff, Überrennen}\trennlinie \kreaturfertigkeiten{Laufen 16, Wachsamkeit 10}\trennlinie \kreaturinfo{AsP}{32}\kreaturinfo{Dämonisch}{ (Armatrutz, Gardianum, Leidensbund)}\trennlinie \kreaturinfo{Beschwörung}{Invocatio}\kreaturinfo{Dienste}{Beschwörer transportieren+4 (1 Stunde), Beschwörer im Kampf mit Zaubern schützen+0 (1 Stunde)}\trennlinie \kreaturinfo{Info}{Mit entsprechenden Donaria kann Yst{-}Phogorthu Eigenschaften niederhöllischer Domänen erhalten. Blakharaz: Höllisches Streitross (AT, VT, \newline%
  TP: +2)\newline%
Lolgramoth: Rasendes Ross (GS 20) Thargunitoth: Nachtmähre (Schreckgestalt II) Agrimoth: Lauffeuerfuchs (Immunität (Feuer), Biss und Hornstoß erhalten die Waffeneigenschaft Nachbrennen)\newline%
}\trennlinie \kreaturinfo{Quelle}{\href{https://dsaforum.de/viewtopic.php?p=1887303\#p1887303}{Pandämonium}}}}
\newcommand{\kreaturdetailzazamotlgnakhyaa}{\kreatur{Zazamotl'gnakhyaa}{Der gefräßige Söldner der niedersten Höllen; ein zweigehörnter, unabhängiger Dämon, kleiner Gegner}{gfx/kreaturen/daemon}{\kreaturkampfwerte{4}{10}{4}{8}\trennlinie \kreaturinfo{Resistenz}{dämonisch, magisch}\kreaturvorteile{Zusätzliche Attacke I}\trennlinie \kreaturwaffe{Klauen}{0}{12}{12}{2W6+0}{Doppelangriff}\kreaturwaffe{Stachel}{0}{4}{16}{2W6+5}{Gift (Zazamotoxin (W,K), wirkt nur gegen Dämonen, umgeht deren Immunität (Gift), Stufe 20 (MR statt KO für Gegenprobe), Verzögerung 0, Intervall 1 Aktion, Schaden 1 Wunde, Proben {-}8, Dauer 4 Aktionen)\newline%
}\trennlinie \kreaturinfo{Beschwörung}{Invocatio}\kreaturinfo{Dienste}{Kampf gegen Dämon+4 (1 Minute), Dämon suchen und vernichten+0 (1 Tag), 1 Dosis Zazamotoxin zur Verfügung stellen{-}4}\trennlinie \kreaturinfo{Quelle}{\href{https://dsaforum.de/viewtopic.php?p=1887303\#p1887303}{Pandämonium}}}}
\newcommand{\kreaturdetailzilit}{\kreatur{Zilit}{Kleiner amphibischer Humanoide}{gfx/kreaturen/humanoid}{\kreaturkampfwerte{5}{4}{5}{2}\trennlinie \kreaturinfo{Angepasst II (Wasser, Dunkelheit)}{Angepasst I/II (Umgebung) Durch deine Spezies oder langjährige Erfahrung hast du dich an eine bestimmte Umgebung oder Umweltbedingung gewöhnt. Abzüge durch diese Umgebung (Beispiele auf S. 38), insbesondere im Kampf, sinken für dich um eine/zwei Stufen. Die Kosten für Angepasst legt der Spielleiter fest, wobei er sich an der Häufigkeit der Umgebung orientieren sollte. Zu allgemein gefasste Umgebungen wie „unsicherer Untergrund“ sollte er nicht zulassen. Beispiele für Angepasst sind: • Dunkelheit: verringert Abzüge durch schlechte Lichtverhältnisse (40 EP pro Stufe) • Schnee: verringert Abzüge durch schneebedeckten oder eisigen Untergrund (20 EP pro Stufe) • Wasser: verringert Abzüge durch knie{-} oder hüfttiefes Wasser und unter Wasser (20 EP pro Stufe) • Wald: verringert Abzüge durch Wurzeln, Gestrüpp und dichtes Unterholz (40 EP pro Stufe) Voraussetzungen: keine/Angepasst I Nachkauf: häufig/selten\newline%
}\kreaturvorteile{Natürliche Rüstung I, Amphibisch}\trennlinie \kreaturwaffe{Holzspeer}{}{}{}{2W6{-}1}{Wendig}\kreaturwaffe{Holzspeer}{8}{}{}{1W6+2}{Wurfwaffe?}\trennlinie \kreaturattribute{CH 4, FF 8, GE 8, IN 4, KK 2, KL 6, KO 4, MU 8}\kreaturfertigkeiten{Pirschen 8, Schwimmen 10, Wachsamkeit 6}\trennlinie \kreaturinfo{Quelle}{\href{https://dsaforum.de/viewtopic.php?f=180&p=1738549\#p1738549}{Bestarium+}}}}
\newcommand{\kreaturdetailabualazilat}{\kreatur{Abu al‘Azilat}{verführerischer, von Rosen bewachsener Humusdschinn}{gfx/kreaturen/elementar}{\kreaturkampfwerte{13}{12}{8}{6}\trennlinie \kreaturvorteile{Regeneration I}\trennlinie \kreaturwaffe{Giftranke}{1}{12}{16}{2W6{-}2}{Fesseln, Giftig (Goldleim, Stufe 24, siehe S. 34)}\trennlinie \kreaturfertigkeiten{Betören 16, Tulamidenlande (Gebräuche) 14, Menschen­kenntnis 14, Überreden 12, Wachsamkeit 10, Willenskraft 12}\trennlinie \kreaturinfo{AsP}{64}\kreaturinfo{Humus}{16 (Fesselranken, Haselbusch, Kraft des Humus, Lach dich Gesund, Leib der Erde, Leidensbund, Manifesto, Ruhe Körper, Satuarias Herrlichkeit, Wand aus Dornen)}\trennlinie \kreaturinfo{Beschwörung}{Herbeirufung des Humus}\kreaturinfo{Dienste}{Wächterinnen ablenken (4 Minuten, +4)+0, prächtiger Blumenschmuck für ein Fest+0, einer Frau den Kopf verdrehen (1 Woche, {-}4)+0}\trennlinie \kreaturinfo{Info}{Varianten: Selbstverständlich gibt es mit Saba al‘AzilAT: auch eine Tochter der Rosen, die schon manch eitlem Magus den Kopf verdreht hat...\newline%
}\trennlinie \kreaturinfo{Quelle}{\href{https://ilarisblog.wordpress.com/downloads/}{Ilaris Regeln}}}}
\newcommand{\kreaturdetailbaromna}{\kreatur{Baromna}{in Bierkrügen lebende Dienerin des Wassers; sehr kleiner Gegner}{gfx/kreaturen/elementar}{\kreaturkampfwerte{3}{8}{}{2}\trennlinie \kreaturvorteile{Wasserwesen}\trennlinie \kreaturwaffe{Wasserschwall}{2}{4}{8}{1W6+0}{Ertränken}\trennlinie \kreaturfertigkeiten{Zwerge (Gebräuche) 10, Menschenkenntnis 10, Schwimmen 14, Wachsamkeit 6, Wahrnehmung 8}\trennlinie \kreaturinfo{AsP}{24}\kreaturinfo{Wasser}{8 (Abvenenum, Macht über den Regen, Manifesto, Sapefacta, Wasserbann, Wellenlauf)}\trennlinie \kreaturinfo{Beschwörung}{Herbeirufung des Wassers}\kreaturinfo{Dienste}{Zecher zur Besinnung bringen+4, zahlreiche Getränke von unangenehmen Inhalten und Giften bis Stufe 24  reinigen+0, ein Fass mit Bier füllen{-}4}\trennlinie \kreaturinfo{Quelle}{\href{https://ilarisblog.wordpress.com/downloads/}{Ilaris Regeln}}}}
\newcommand{\kreaturdetailbaradarasch}{\kreatur{Baradarasch}{plappernder Bader und Dschinn des Wassers}{gfx/kreaturen/elementar}{\kreaturkampfwerte{11}{12}{8}{6}\trennlinie \kreaturvorteile{Amphibisch, Paraphysikalität}\trennlinie \kreaturwaffe{Sog}{2}{12}{16}{2W6+2}{Ertränken}\kreaturwaffe{Wasserstrahl}{8}{}{}{2W6+4}{Zurückstoßen, Ertränken}\trennlinie \kreaturfertigkeiten{Elementarismus 16, Tulamidenlande (Gebräuche) 14, Geschichten und Legenden 16, Menschenkenntnis 14, Schwimmen 20, Sinnenschärfe 12, Wachsamkeit 12, Willenskraft 14}\kreaturfertigkeiten{Bader (meisterlich)}\trennlinie \kreaturinfo{AsP}{64}\kreaturinfo{Wasser}{16 (Aquafaxius, Aquasphaero, Hartes Schmelze, Macht über den Regen, Mahlstrom, Manifesto, Nebelwand, Nebelleib, Sapefacta, Wand aus Wasser, Wellenlauf)}\trennlinie \kreaturinfo{Beschwörung}{Herbeirufung des Wassers}\kreaturinfo{Dienste}{Konversation oder Beratung (1 Stunde, +4)+0, Fluten eines Bades mit duftendem Wasser+0, Badende ausspionieren ({-}4)+0}\trennlinie \kreaturinfo{Quelle}{\href{https://ilarisblog.wordpress.com/downloads/}{Ilaris Regeln}}}}
\newcommand{\kreaturdetailbashaidor}{\kreatur{Brashaidor}{gemächlicher und gutmütiger Meister der Luft; sehr großer Gegner}{gfx/kreaturen/elementar}{\kreaturkampfwerte{15}{24}{24}{22}\trennlinie \kreaturvorteile{Flugfähig, Paraphysikalität, Unsichtbar}\trennlinie \kreaturwaffe{Orkanbö}{8}{18}{22}{1W20+0}{Flächenangriff (180° vor dem Elementar), Zurückstoßen}\trennlinie \kreaturfertigkeiten{Fliegen 20, Geographie 20, Meer (Überleben) 18, Pirschen 20, Sinnenschärfe 20, Tulamidenlande (Gebräuche) 14, Untertauchen 20, Wachsamkeit 20}\kreaturfertigkeiten{Seefahrer 3, Navigator 3}\trennlinie \kreaturinfo{AsP}{128}\kreaturinfo{Luft}{24 (alle Zauber des Elements Luft und ähnliche Effekte)}\trennlinie \kreaturinfo{Beschwörung}{Herbeirufung der Luft}\kreaturinfo{Dienste}{Schiff antreiben+4 (1 Tag), Sturm abschwächen+0 (1 Tag), Schiff über Riff heben{-}4 (1 Tag)}\trennlinie \kreaturinfo{Quelle}{\href{https://ilarisblog.wordpress.com/downloads/}{Ilaris Regeln}}}}
\newcommand{\kreaturdetailchakharim}{\kreatur{Chakharîm}{Meister des Erzes und Wächter von Schätzen; sehr großer Gegner}{gfx/kreaturen/elementar}{\kreaturkampfwerte{18}{20}{2}{3}\trennlinie \kreaturinfo{Magieabweisend}{Zauber wirken auf dich deutlich schwächer. Du ignorierst bei allen Zaubern eine Stufe der spontanen Modifikation Mächtige Magie. Zauber ohne Mächtige Magie haben auf dich keine Wirkung. Voraussetzung: 40 EP Nachkauf: extrem selten\newline%
}\kreaturinfo{Tarnung}{unterirdisch}\kreaturvorteile{Schreckgestalt II}\trennlinie \kreaturwaffe{Steinschlag}{8}{16}{20}{2W20+0}{Flächenangriff (1 Schritt Radius um das Hauptziel), Niederwerfen ({-}4)}\trennlinie \kreaturfertigkeiten{Menschenkenntnis 30, Sinnenschärfe 30, Tulamidenlande (Gebräuche) 8, Wachsamkeit 30, Zauberpraxis 18}\trennlinie \kreaturinfo{AsP}{128}\kreaturinfo{Erz}{24 (alle Zauber des Elements Erz und ähnliche Effekte)}\trennlinie \kreaturinfo{Beschwörung}{Herbeirufung des Erzes}\kreaturinfo{Dienste}{einen gefährlichen Gegenstand unter Tonnen von Stein oder Koschbasalt einschließen+4, Wache halten+0 (für Generationen), elementar gesicherte Schatzkammer anlegen{-}4}\trennlinie \kreaturinfo{Quelle}{\href{https://ilarisblog.wordpress.com/downloads/}{Ilaris Regeln}}}}
\newcommand{\kreaturdetaileichbart}{\kreatur{Eichbart}{Meister des Humus und Hüter der Weisheit; sehr großer Gegner}{gfx/kreaturen/elementar}{\kreaturkampfwerte{16}{20}{3}{6}\trennlinie \kreaturinfo{Tarnung}{im Wald}\kreaturvorteile{Regeneration I}\trennlinie \kreaturwaffe{Rankengriff}{8}{16}{24}{2W6+2}{Fesseln, Umklammern ({-}4, 24)}\kreaturwaffe{Asthieb}{2}{16}{24}{3W6+4}{Niederwerfen ({-}8) , Stumpf}\kreaturkampfvorteile{Zusätzliche Attacke IV}\trennlinie \kreaturfertigkeiten{Holzbearbeitung 22, Mittelaventurien (Überleben) 20, Pflanzenkunde 26, Pirschen 20, Sinnenschärfe 20, Tierkunde 2 4, Tobrien (Gebräuche) 16, Wachsamkeit 20, Zähigkeit 20}\trennlinie \kreaturinfo{AsP}{128}\kreaturinfo{Humus}{24 (alle Zauber des Elements Humus und ähnliche Effekte)}\trennlinie \kreaturinfo{Beschwörung}{Herbeirufung des Humus}\kreaturinfo{Dienste}{Beratung zu Heilkunde+0, Pflanzen oder Humusmagie{-}4 (1 Tag), Augen des Waldes (Eichbart hält durch die Augen aller Waldbewohner Ausschau)+0 (1 Tag), Gedächtnis des Waldes (Eichbart berichtet detailliert von einem Ereignis zu Lebzeiten der Waldbewohner)+0}\trennlinie \kreaturinfo{Quelle}{\href{https://ilarisblog.wordpress.com/downloads/}{Ilaris Regeln}}}}
\newcommand{\kreaturdetailemling}{\kreatur{Emling}{belesener Diener der Luft in Gestalt einer Feder; kleiner Gegner}{gfx/kreaturen/elementar}{\kreaturkampfwerte{1}{8}{1}{4}\trennlinie \kreaturvorteile{Flugfähig}\trennlinie \kreaturwaffe{Luftstoß}{1}{4}{8}{}{Zurückstoßen (KK 16, um zu widerstehen)}\trennlinie \kreaturfertigkeiten{Fliegen 12, Geographie 6, Pirschen 8, Sinnenschärfe 10, Thorwal (Gebräuche) 8, Untertauchen 12, Wachsamkeit 10, Willenskraft 4}\kreaturfertigkeiten{Hjaldingsche Schriftzeichen (erfahren)}\trennlinie \kreaturinfo{AsP}{24}\kreaturinfo{Luft}{8 (Aeolitus, Aeropulvis, Aufgeblasen, Manifesto, Sapefacta, Windgeflüster, Windstille)}\trennlinie \kreaturinfo{Beschwörung}{Herbeirufung der Luft}\kreaturinfo{Dienste}{Kitzeln+4, kurze Nachricht in Runen verfassen+0, Runeninschrift entziffern{-}4}\trennlinie \kreaturinfo{Quelle}{\href{https://ilarisblog.wordpress.com/downloads/}{Ilaris Regeln}}}}
\newcommand{\kreaturdetailfarlin}{\kreatur{Farlin}{geschickter und ordnungsliebender Dschinn des Erzes}{gfx/kreaturen/elementar}{\kreaturkampfwerte{16}{12}{3}{5}\trennlinie \kreaturwaffe{Hammer}{1}{12}{16}{2W6+4}{Stumpf, Niederschmettern}\kreaturkampfvorteile{Hammerschlag}\trennlinie \kreaturfertigkeiten{Mechanik 16, Untertauchen 12, Schmieden 18, Sinnenschärfe 16, Tulamidenlande (Gebräuche) 8, Wachsamkeit 16, Zähigkeit 16}\kreaturinfo{Profane Vorteile}{Meisterwerk}\trennlinie \kreaturinfo{AsP}{64}\kreaturinfo{Erz}{16 (Adamantium, Archofaxius, Archosphaero, Armatrutz, Kraft des Erzes, Manifesto, Metamorpho, Pfeil des Erzes, Wand aus Erz)}\trennlinie \kreaturinfo{Beschwörung}{Herbeirufung des Erzes}\kreaturinfo{Dienste}{Werkstatt aufräumen und warten+4, Gebrauchsgegenstand schmieden+0, Waffe oder Rüstung schmieden (2x Hohe Qualität, kein Meisterwerk, {-}4)+0}\trennlinie \kreaturinfo{Quelle}{\href{https://ilarisblog.wordpress.com/downloads/}{Ilaris Regeln}}}}
\newcommand{\kreaturdetailflammina}{\kreatur{Flammina}{aufgeweckte Dienerin des Feuers; sehr kleiner Gegner}{gfx/kreaturen/elementar}{\kreaturkampfwerte{1}{8}{4}{6}\trennlinie \kreaturvorteile{Regeneration I}\trennlinie \kreaturwaffe{Funke}{2}{2}{8}{2W6{-}2}{Nachbrennen}\trennlinie \kreaturfertigkeiten{Magische Elixiere 12, Mittelreich (Gebräuche) 4, Sinnenschärfe 12, Wachsamkeit 12, Zähigkeit 6}\trennlinie \kreaturinfo{AsP}{24}\kreaturinfo{Feuer}{8 (Brenne, Caldofrigo, Custodosigil, Ignifugo, Ignimorpho, Manifesto)}\trennlinie \kreaturinfo{Beschwörung}{Herbeirufung des Feuers}\kreaturinfo{Dienste}{Entzünden von nassem Brennholz+4, Lagerfeuer ohne Brennmaterial über Nacht brennen lassen+0, Lagerfeuer erhitzen sodass es Erz schmilzt oder alchemistische Synthesen erlaubt{-}4}\trennlinie \kreaturinfo{Quelle}{\href{https://ilarisblog.wordpress.com/downloads/}{Ilaris Regeln}}}}
\newcommand{\kreaturdetailfirndorn}{\kreatur{Firndorn}{menschenhassender und mörderischer Dschinn des Eises}{gfx/kreaturen/elementar}{\kreaturkampfwerte{9}{12}{8}{14}\trennlinie \kreaturinfo{Aura}{Kälte, Zähigkeit (20) alle 4 Initiativephasen, 1 Wunde}\kreaturinfo{Tarnung}{in Schnee und Eis}\kreaturvorteile{}\trennlinie \kreaturwaffe{Eisdorn}{0}{12}{16}{2W6+0}{Erfrieren, Todesstoß}\kreaturwaffe{Hagelsturm}{8}{}{}{2W6+4}{Stumpf}\trennlinie \kreaturfertigkeiten{Geographie 8, Hoher Norden (Überleben) 12, Nordaventurien (Gebräuche) 8, Pirschen 20, Sinnenschärfe 16, Tierkunde 16, Wachsamkeit 16, Zähigkeit 12}\trennlinie \kreaturinfo{AsP}{64}\kreaturinfo{Eis}{16 (Corpofrigo, Frigifaxius, Frigisphaero, Glacioflumen, Gletscherwand, Manifesto, Metamorpho, Pfeil des Eises, Stillstand)}\trennlinie \kreaturinfo{Beschwörung}{Herbeirufung des Eises}\kreaturinfo{Dienste}{Beratung bei einem Attentat+4, Bau von Fallen im Eis (z.B. verdeckte Gletscherspalten)+0, Lebewesen in der Nähe suchen und töten{-}4}\trennlinie \kreaturinfo{Quelle}{\href{https://ilarisblog.wordpress.com/downloads/}{Ilaris Regeln}}}}
\newcommand{\kreaturdetailglacius}{\kreatur{Glacius}{Meister des Eises und vollendeter Baumeister; sehr großer Gegner}{gfx/kreaturen/elementar}{\kreaturkampfwerte{15}{20}{8}{18}\trennlinie \kreaturinfo{Aura}{Kälte, Zähigkeit (28) alle 4 Initiativephasen, 1 Wunde}\kreaturvorteile{Amphibisch, Schreckgestalt I}\trennlinie \kreaturwaffe{Lawine}{8}{14}{16}{2W20+0}{Erfrieren, Flächenangriff (90° vor dem Elementar), Zurückstoßen}\trennlinie \kreaturfertigkeiten{Gebirge (Überleben) 32, Nordaventurien (Gebräuche) 12, Sinnenschärfe 30, Tierkunde 24}\kreaturfertigkeiten{Rechenkünstler 3, Architekt 3}\trennlinie \kreaturinfo{AsP}{128}\kreaturinfo{Eis}{24 (alle Zauber des Elements Eis und ähnliche Effekte)}\trennlinie \kreaturinfo{Beschwörung}{Herbeirufung des Eises}\kreaturinfo{Dienste}{Bau eines Iglus für die Helden und etwa 20 Begleiter+4, Bau einer 50 Schritt langen Brücke aus Eis+0, Verschließen eines Tales mit einer Mauer aus Eis{-}4}\trennlinie \kreaturinfo{Quelle}{\href{https://ilarisblog.wordpress.com/downloads/}{Ilaris Regeln}}}}
\newcommand{\kreaturdetailgrommur}{\kreatur{Grommur}{zerstörerischer Diener des Erzes; kleiner Gegner}{gfx/kreaturen/elementar}{\kreaturkampfwerte{4}{8}{1}{3}\trennlinie \kreaturvorteile{Zerstörerisch I}\trennlinie \kreaturwaffe{Steinfaust}{0}{2}{0}{2W6+2}{Niederschmettern}\kreaturwaffe{Steinwurf}{4}{}{}{2W6+2}{Niederschmettern}\trennlinie \kreaturfertigkeiten{Gebirge (Überleben) 12, Sinnenschärfe 10, Untertauchen 10, Wachsamkeit 10, Zwerge (Gebräuche) 4}\trennlinie \kreaturinfo{AsP}{24}\kreaturinfo{Erz}{8 (Eisenrost, Kraft des Erzes, Manifesto, Weiches Erstarre, Zagibu Ubigaz)}\trennlinie \kreaturinfo{Beschwörung}{Herbeirufung des Erzes}\kreaturinfo{Dienste}{Schätze vernichten+4, Mechanismus sabotieren+0, Gefängnistür aufbrechen{-}4}\trennlinie \kreaturinfo{Quelle}{\href{https://ilarisblog.wordpress.com/downloads/}{Ilaris Regeln}}}}
\newcommand{\kreaturdetailhefashar}{\kreatur{Hefashar}{Meister des Wassers und gischtgekrönte Woge; sehr großer Gegner}{gfx/kreaturen/elementar}{\kreaturkampfwerte{13}{24}{}{12}\trennlinie \kreaturinfo{Resistenz I}{Stichwaffen}\kreaturvorteile{Paraphysikalität, Schreckgestalt I, Wasserwesen}\trennlinie \kreaturwaffe{Flutwelle}{8}{15}{24}{2W20+0}{Ertränken, Flächenangriff (90° vor dem Elementar), Zurückstoßen}\trennlinie \kreaturfertigkeiten{Tulamidenlande (Gebräuche) 4, Geographie 16, Meer (Überleben) 24, Pirschen 20, Schwimmen 32, Sinnenschärfe 28, Wachsamkeit 24}\trennlinie \kreaturinfo{AsP}{128}\kreaturinfo{Wasser}{24 (alle Wasserzauber und ähnliche Effekte)}\trennlinie \kreaturinfo{Beschwörung}{Herbeirufung des Wassers}\kreaturinfo{Dienste}{Schiff festhalten+4, Schiff versenken+0, Küste verwüsten{-}4}\trennlinie \kreaturinfo{Quelle}{\href{https://ilarisblog.wordpress.com/downloads/}{Ilaris Regeln}}}}
\newcommand{\kreaturdetailkaeuterkopf}{\kreatur{Kräuterkopf}{pflanzenaffiner Diener des Humus; kleiner Gegner}{gfx/kreaturen/elementar}{\kreaturkampfwerte{4}{8}{3}{1}\trennlinie \kreaturvorteile{Regeneration I}\trennlinie \kreaturwaffe{Rankengriff}{1}{5}{9}{2W6+2}{Fesseln}\trennlinie \kreaturfertigkeiten{Gifte und Krankheiten (Heilkunde) 10, Mittelaventurien (Überleben) 14, Mittelreich (Gebräuche) 4, Pflanzenkunde 12, Wachsamkeit 4}\trennlinie \kreaturinfo{AsP}{24}\kreaturinfo{Humus}{14 (Balsam, Haselbusch, Manifesto, Sumus Elixiere, Klarum Purum)}\trennlinie \kreaturinfo{Beschwörung}{Herbeirufung des Humus}\kreaturinfo{Dienste}{Wachstum einer Heilpflanze außerhalb der Saison+4, Stärkung eines Heilkrauts (+2 Stufen Hohe Qualität für daraus   angefertigte Heilmittel)+0, Gegenmittel für ein starkes Gift{-}4}\trennlinie \kreaturinfo{Quelle}{\href{https://ilarisblog.wordpress.com/downloads/}{Ilaris Regeln}}}}
\newcommand{\kreaturdetaillagramagmangr}{\kreatur{Lagramagmangr}{hitzköpfige Lavawalze und Meister des Feuers; sehr großer Gegner}{gfx/kreaturen/elementar}{\kreaturkampfwerte{11}{20}{12}{20}\trennlinie \kreaturinfo{Aura}{Hitze, Zähigkeit (28) jede Initiativephase, 1 Wunde}\kreaturvorteile{Explosion (8W6), Schreckgestalt II}\trennlinie \kreaturwaffe{Inferno}{8}{16}{24}{3W20+0}{Flächenangriff (180° vor dem Elementar), Nachbrennen, Niederwerfen}\trennlinie \kreaturfertigkeiten{Einschüchtern 30, Sinnenschärfe 12, Schmieden 18, Wachsamkeit 12, Willenskraft 40, Zwerge (Gebräuche) 12}\trennlinie \kreaturinfo{AsP}{128}\kreaturinfo{Feuer}{24 (alle Zauber des Elements Feuer und ähnliche Effekte)}\trennlinie \kreaturinfo{Beschwörung}{Herbeirufung des Feuers}\kreaturinfo{Dienste}{mehrere Quader Gestein schmelzen+4, Lavaströme umlenken+0, Vulkanausbruch verursachen+0, Vulkanausbruch verhindern{-}4}\trennlinie \kreaturinfo{Quelle}{\href{https://ilarisblog.wordpress.com/downloads/}{Ilaris Regeln}}}}
\newcommand{\kreaturdetailsausewind}{\kreatur{Sausewind}{gesprächiger Luftdschinn und verlässlicher Bote}{gfx/kreaturen/elementar}{\kreaturkampfwerte{7}{12}{10}{11}\trennlinie \kreaturvorteile{Flugfähig}\trennlinie \kreaturwaffe{Luftstoß}{1}{15}{15}{2W6+0}{Zurückstoßen}\kreaturwaffe{Luftwirbel}{8}{}{}{2W6+4}{}\kreaturkampfvorteile{Scharfschuss, Sturmangriff}\trennlinie \kreaturfertigkeiten{Gebräuche 12, Geographie 14, Fliegen 18, Pirschen 10, Untertauchen 10, Rhetorik 14, Überreden 14, Wachsamkeit 16}\kreaturfertigkeiten{Philosophie (erfahren)}\trennlinie \kreaturinfo{AsP}{64}\kreaturinfo{Luft}{16 (Aeolitus, Aeropulvis, Leib des Windes, Manifesto, Nebelleib, Orcanofaxius, Sapefacta, Wettermeisterschaft, Windgeflüster, Windhose)}\trennlinie \kreaturinfo{Beschwörung}{Herbeirufung der Luft}\kreaturinfo{Dienste}{Nachrichtenübermittlung Puni{-}Gareth+4, Nachrichtenübermittlung Khunchom{-}Gareth+0, mehrere Städte nach einem Adressaten absuchen{-}4 (1 Tag)}\trennlinie \kreaturinfo{Quelle}{\href{https://ilarisblog.wordpress.com/downloads/}{Ilaris Regeln}}}}
\newcommand{\kreaturdetailsnefnug}{\kreatur{Snefnug}{hilfsbereite Schneeflocke und Diener des Eises; kleiner Gegner}{gfx/kreaturen/elementar}{\kreaturkampfwerte{2}{8}{3}{5}\trennlinie \kreaturwaffe{Eisgriff}{0}{4}{4}{1W6+0}{Erfrieren}\kreaturwaffe{Schneeball}{4}{}{}{2W6+0}{Erfrieren}\trennlinie \kreaturfertigkeiten{Geographie 8, Nordaventurien (Überleben) 8, Pirschen 8, Thorwal (Gebräuche) 6, Tierkunde 6, Wachsamkeit 6, Wahrnehmung 8}\trennlinie \kreaturinfo{AsP}{24}\kreaturinfo{Eis}{8 (Caldofrigo, Glacioflumen, Manifesto, Metamorpho, Warmes gefriere!)}\trennlinie \kreaturinfo{Beschwörung}{Herbeirufung des Eises}\kreaturinfo{Dienste}{Kühlen von einigen Maß Getränk +4+0, Einfrieren von Nahrungsmittelvorräten+0, Erschaffen von einigen Stein Eis {-}4+0}\trennlinie \kreaturinfo{Quelle}{\href{https://ilarisblog.wordpress.com/downloads/}{Ilaris Regeln}}}}
\newcommand{\kreaturdetailamrychoth}{\kreatur{Amrychoth}{fünfgehörnter Rochen und Verderben der Seefahrer; großer Gegner}{gfx/kreaturen/daemon}{\kreaturkampfwerte{16}{20}{}{8}\trennlinie \kreaturvorteile{Regeneration I, Schreckgestalt II, Wasserwesen}\trennlinie \kreaturwaffe{Flutwelle}{8}{14}{18}{1W20+5}{Ertränken, Flächenangriff (90° vor dem Dämon), Zurückstoßen}\kreaturwaffe{Hornstoß}{1}{2}{4}{3W20+0}{}\trennlinie \kreaturfertigkeiten{Schwimmen 24, Wachsamkeit 14}\trennlinie \kreaturinfo{Beschwörung}{Invocatio}\kreaturinfo{Dienste}{Schiffbrüchige ertränken+4, Schiff manövrierunfähig schlagen+0, Schiff versenken{-}4}\trennlinie \kreaturinfo{Quelle}{\href{https://ilarisblog.wordpress.com/downloads/}{Ilaris Regeln}}}}
\newcommand{\kreaturdetailakrhobal}{\kreatur{Arkhobal}{siebengehörnter Verderber der Wälder; sehr großer Gegner}{gfx/kreaturen/daemon}{\kreaturkampfwerte{14}{20}{0}{4}\trennlinie \kreaturinfo{Tarnung}{im Wald}\kreaturinfo{Dämonenharz}{Wer dem Arkhobal\newline%
mit einem Nahkampfangriff mindestens 2 Kratzer zufügt, muss eine\newline%
GE{-}Probe (20) ablegen. Misslingt die Probe, erleidest du einen Malus von {-}2 auf alle körperlichen Proben (kumulativ) und musst eine KO{-}Probe\newline%
(20) ablegen. Misslingt sie, verwandelst du dich innerhalb von 2W6\newline%
Tagen in einen Menschenbaum; die Alveranie gilt als Gegenmittel.)\newline%
}\kreaturvorteile{Regeneration I}\trennlinie \kreaturwaffe{Rankenangriff}{8}{16}{20}{2W6+2}{Fesseln, Umklammern ({-}4, 20)}\kreaturwaffe{Asthieb}{2}{16}{20}{3W6+2}{Niederwerfen ({-}8)}\kreaturkampfvorteile{Zusätzliche Attacke IV}\trennlinie \kreaturfertigkeiten{Pirschen 16, Sinnenschärfe 20, Wachsamkeit 20}\trennlinie \kreaturinfo{AsP}{64}\kreaturinfo{Dämonisch}{16 (Fesselranken, Pandämonium, Wand aus Dornen)}\trennlinie \kreaturinfo{Beschwörung}{Invocatio}\kreaturinfo{Dienste}{Dämonisches Holz erschaffen+8, mittelgroßen Wald verderben+0 (permanent)}\trennlinie \kreaturinfo{Info}{Varianten: Alter Arkhobal (für je 50 Jahre: Wundschwelle, AT,  TP: +1)\newline%
}\trennlinie \kreaturinfo{Quelle}{\href{https://ilarisblog.wordpress.com/downloads/}{Ilaris Regeln}}}}
\newcommand{\kreaturdetailazzitai}{\kreatur{Azzitai}{dreigehörnter Flammendämon, der sogar Gestein in Brand setzt}{gfx/kreaturen/daemon}{\kreaturkampfwerte{10}{10}{5}{4}\trennlinie \kreaturinfo{Aura}{Hitze, Zähigkeit (20) alle 4 Initiativephasen, 1 Wunde}\kreaturinfo{Immunität}{Feuer}\kreaturinfo{Resistenz I}{Stichwaffen}\kreaturinfo{Verwundbarkeit I}{Wasser}\kreaturinfo{Flammenkörper}{Alle Waffen, die Holz beinhalten, gelten gegen den Azzitai als Zerbrechlich (S. 47).}\kreaturvorteile{}\trennlinie \kreaturwaffe{Hörner}{0}{2}{16}{2W6+7}{Nachbrennen}\kreaturwaffe{Feuerpranken}{1}{14}{16}{2W6+2}{Nachbrennen}\kreaturkampfvorteile{Zusätzliche Attacke I}\trennlinie \kreaturinfo{AsP}{64}\kreaturinfo{Dämonisch}{20 (Brenne!, Hartes schmelze!, Wand aus Flammen)}\trennlinie \kreaturinfo{Beschwörung}{Invocatio}\kreaturinfo{Dienste}{große Mengen Gestein schmelzen+4, Dorf oder Stadtviertel in Brand stecken+0, Ziel suchen und einäschern{-}4}\trennlinie \kreaturinfo{Quelle}{\href{https://ilarisblog.wordpress.com/downloads/}{Ilaris Regeln}}}}
\newcommand{\kreaturdetaildharai}{\kreatur{Dharai}{zweigehörnter Baudämon mit immensen Kräften; großer Gegner}{gfx/kreaturen/daemon}{\kreaturkampfwerte{13}{18}{3}{2}\trennlinie \kreaturinfo{Formlosigkeit}{kann seinen Körper so verändern, dass er durch handbreite Spalte passt}\kreaturvorteile{Regeneration I}\trennlinie \kreaturwaffe{Gallertarm}{2}{2}{8}{}{Festes Umklammern (Das Ziel wird umschlungen. Gelingt ihm oder einem Helfer bis zur nächsten Initiativephase keine KK{-}Probe (32), erleidet es 4W20 SP.)}\trennlinie \kreaturinfo{Beschwörung}{Invocatio (24), 32 AsP}\kreaturinfo{Dienste}{Transport riesiger Massen+4 (1 Stunde), Transport riesiger Massen+0 (1 Tag), Hilfe bei Bauarbeiten{-}4 (1 Tag)}\trennlinie \kreaturinfo{Varianten}{Je{-}Chrizlayk{-}Ura (Verfügt über noch höhere Kräfte, ist dafür aber noch langsamer: GS: und INI: {-}2.)}\kreaturinfo{Anmerkung}{Dharai und Je{-}Chrizlayk{-}Ura kämpfen nur, wenn sie angegriffen werden.}\trennlinie \kreaturinfo{Quelle}{\href{https://ilarisblog.wordpress.com/downloads/}{Ilaris Regeln}}}}
\newcommand{\kreaturdetaildifar}{\kreatur{Difar}{extrem flinker Botendämon; sehr kleiner Gegner}{gfx/kreaturen/daemon}{\kreaturkampfwerte{2}{12}{15}{12}\trennlinie \kreaturwaffe{Biss}{0}{16}{6}{1W6+0}{}\trennlinie \kreaturinfo{Beschwörung}{Invocatio (16), 16 AsP}\kreaturinfo{Dienste}{Botschaft über wenige Meilen überbringen+4, Botschaft über große Entfernungen überbringen+0, Gegenstand suchen{-}4 (1 Tag)}\trennlinie \kreaturinfo{Quelle}{\href{https://ilarisblog.wordpress.com/downloads/}{Ilaris Regeln}}}}
\newcommand{\kreaturdetailduglum}{\kreatur{Duglum}{siebengehörnter Überbringer von Krankheiten; großer Gegner}{gfx/kreaturen/daemon}{\kreaturkampfwerte{16}{20}{5}{12}\trennlinie \kreaturvorteile{Regeneration I, Schreckgestalt III}\trennlinie \kreaturwaffe{Zangen}{1}{12}{16}{4W6+5}{Rüstungsbrechend}\kreaturwaffe{Schleimsekret}{8}{}{}{1W6+0}{Infektion (Das Ziel wird mit einer Krankheit bis Stufe 24 infiziert.)}\kreaturkampfvorteile{Zusätzliche Attacke I}\trennlinie \kreaturinfo{AsP}{64}\kreaturinfo{Dämonisch}{18 (Fluch der Pestilenz, Schwarz und Rot, Tlalucs Odem))}\trennlinie \kreaturinfo{Beschwörung}{Invocatio}\kreaturinfo{Dienste}{nahes Ziel mit einer Krankheit bis Stufe 32 infizieren+4, Seuche bis Stufe 24 in einer Stadt verbreiten+0, Besitzer eines vorhandenen Körperteils suchen und mit einer Krankheit bis Stufe 32 infizieren{-}4}\trennlinie \kreaturinfo{Quelle}{\href{https://ilarisblog.wordpress.com/downloads/}{Ilaris Regeln}}}}
\newcommand{\kreaturdetailgotongi}{\kreatur{Gotongi}{Spionagedämon aus der Domäne Blakharaz‘; sehr kleiner Gegner}{gfx/kreaturen/daemon}{\kreaturkampfwerte{2}{12}{}{12}\trennlinie \kreaturwaffe{Ausweichen}{0}{18}{}{}{}\trennlinie \kreaturinfo{AsP}{16}\kreaturinfo{Hellsicht}{14 (Exposami, Penetrizzel)}\trennlinie \kreaturinfo{Beschwörung}{Invocatio}\kreaturinfo{Dienste}{Haus durchsuchen+4, Meldung von Eindringlingen+0 (1 Tag), Ausspionieren von Personen{-}4 (1 Tag)}\trennlinie \kreaturinfo{Quelle}{\href{https://ilarisblog.wordpress.com/downloads/}{Ilaris Regeln}}}}
\newcommand{\kreaturdetailheshtot}{\kreatur{Heshthot}{niederer Rachedämon aus der Domäne Blakharaz‘}{gfx/kreaturen/daemon}{\kreaturkampfwerte{4}{5}{5}{4}\trennlinie \kreaturwaffe{Peitsche}{2}{2}{11}{2W6{-}1}{Lähmend (Übersteigt der Schaden deinen RS, sind bis zum nächsten\newline%
Morgen alle Proben auf KK und Fertigkeiten mit KK kumulativ um {-}1 erschwert)\newline%
}\kreaturwaffe{Schwert}{1}{6}{10}{2W6+3}{Zerstörerisch (Deine Waffe erleidet bei jeder gelungenen Verteidigung\newline%
außer Ausweichen den vollen Waffenschaden des Schwertes.)\newline%
}\kreaturkampfvorteile{Zusätzliche Attacke I}\trennlinie \kreaturinfo{AsP}{16}\kreaturinfo{Umwelt}{16 (Dunkelheit)}\trennlinie \kreaturinfo{Beschwörung}{Invocatio}\kreaturinfo{Dienste}{Waffe aushändigen+4 (1 Woche), Kampf+0 (1 Minute), Ziel in der Nähe suchen und töten{-}4}\trennlinie \kreaturinfo{Quelle}{\href{https://ilarisblog.wordpress.com/downloads/}{Ilaris Regeln}}}}
\newcommand{\kreaturdetailirrhalk}{\kreatur{Irrhalk}{viergehörnte Verhöhnung eines Greifen; großer Gegner}{gfx/kreaturen/daemon}{\kreaturkampfwerte{12}{18}{6}{5}\trennlinie \kreaturinfo{Resistenz II}{Feuer}\kreaturinfo{Verwundbarkeit I}{Wasser}\kreaturinfo{Explosion}{5W6}\kreaturvorteile{Flugfähig, Regeneration I, Schreckgestalt II}\trennlinie \kreaturwaffe{Schnabelhieb}{0}{2}{14}{4W6+4}{}\kreaturwaffe{Prankenhieb}{1}{12}{12}{3W6+2}{Nachbrennen}\kreaturkampfvorteile{Zusätzliche Attacke I, Sturmangriff, Niederwerfen}\trennlinie \kreaturinfo{Beschwörung}{Invocatio}\kreaturinfo{Dienste}{Kampf+4 (1 Minute), Wache+0 (1 Tag), Suchen und Töten einer Person{-}4 (1 Tag)}\trennlinie \kreaturinfo{Quelle}{\href{https://ilarisblog.wordpress.com/downloads/}{Ilaris Regeln}}}}
\newcommand{\kreaturdetailkarakil}{\kreatur{Karakil}{Flugdämon in Gestalt einer geflügelten Schlange; großer Gegner}{gfx/kreaturen/daemon}{\kreaturkampfwerte{12}{10}{1}{4}\trennlinie \kreaturvorteile{Flugfähig}\trennlinie \kreaturwaffe{Klauenhieb}{1}{8}{8}{4W6+2}{}\kreaturkampfvorteile{Sturmangriff}\trennlinie \kreaturfertigkeiten{Fliegen 14}\trennlinie \kreaturinfo{Beschwörung}{Invocatio (24), 32 AsP}\kreaturinfo{Dienste}{Beschwörer transportieren+4 (1 Stunde), Beschwörer in den Kampf tragen+0 (1 Stunde), Transport großer Lasten{-}4}\trennlinie \kreaturinfo{Quelle}{\href{https://ilarisblog.wordpress.com/downloads/}{Ilaris Regeln}}}}
\newcommand{\kreaturdetailkarmanath}{\kreatur{Karmanath}{eiskalter Höllenhund und unbarmherziger Hetzer}{gfx/kreaturen/daemon}{\kreaturkampfwerte{5}{8}{10}{7}\trennlinie \kreaturvorteile{Rudel}\trennlinie \kreaturwaffe{Biss}{0}{6}{8}{2W6+2}{Erfrieren}\trennlinie \kreaturfertigkeiten{Laufen 16, Sinnenschärfe 14, Wachsamkeit 14}\trennlinie \kreaturinfo{Beschwörung}{Invocatio}\kreaturinfo{Dienste}{Spur verfolgen+4 (1 Stunde), Hetzen und Töten+0 (1 Stunde), Kampf{-}4 (1 Minute)}\trennlinie \kreaturinfo{Quelle}{\href{https://ilarisblog.wordpress.com/downloads/}{Ilaris Regeln}}}}
\newcommand{\kreaturdetailkhidmakhabul}{\kreatur{Khidma‘kha‘bul}{kleine Rostratte und niederer Diener Tasfarelels}{gfx/kreaturen/daemon}{\kreaturkampfwerte{3}{6}{7}{8}\trennlinie \kreaturvorteile{Rudel}\trennlinie \kreaturwaffe{Biss}{0}{10}{10}{2W6+2}{}\trennlinie \kreaturfertigkeiten{Untertauchen 16, Wachsamkeit 14}\trennlinie \kreaturinfo{Beschwörung}{Invocatio}\kreaturinfo{Dienste}{Wache über Schätze+0 (1 Jahr), Wache über andere Gegenstände{-}4 (1 Woche)}\trennlinie \kreaturinfo{Quelle}{\href{https://ilarisblog.wordpress.com/downloads/}{Ilaris Regeln}}}}
\newcommand{\kreaturdetaillaraan}{\kreatur{Laraan}{fünfgehörnter Verführer, der seine Opfer in den Wahnsinn treibt}{gfx/kreaturen/daemon}{\kreaturkampfwerte{11}{20}{5}{10}\trennlinie \kreaturinfo{Unsichtbarkeit}{Ausnahme: Ziel der Jagd}\kreaturinfo{Verführerische Gestalt}{Der Dämon nimmt die Gestalt eines\newline%
wunderschönen Wesens mit übergroßen Geschlechtsteilen an. In dieser\newline%
Gestalt verliert der Dämon Schreckensgestalt und Unsichtbarkeit und\newline%
erhält die Eigenschaften/Attacken der angenommenen Gestalt.\newline%
}\kreaturvorteile{Regeneration I, Schreckgestalt II, Tarnung}\trennlinie \kreaturwaffe{Krallen*}{1}{12}{16}{2W6+6}{Mutation}\kreaturkampfvorteile{Niederwerfen, Zusätzliche Attacke I}\trennlinie \kreaturfertigkeiten{Betören 24, Gebräuche 20, Menschenkenntnis 18, Sinnenschärfe 16, Wachsamkeit 20}\trennlinie \kreaturinfo{Beschwörung}{Invocatio}\kreaturinfo{Dienste}{Ziel verführen und ablenken+4 (1 Tag), Ziel verführen und dem Beschwörer hörig machen+0 (1 Woche), Ziel verführen und in den Wahnsinn treiben{-}4 (bis von meisterlichem Seelenheiler geheilt)}\trennlinie \kreaturinfo{Dienstopfer}{Jeder Dienst des Laraans beginnt damit, dass der Beschwörer ihm ein Körperteil (z.B. Haar) des Zieles opfert.}\trennlinie \kreaturinfo{Quelle}{\href{https://ilarisblog.wordpress.com/downloads/}{Ilaris Regeln}}}}
\newcommand{\kreaturdetailnephazz}{\kreatur{Nephazz}{niederer Diener Thargunitoths, der Leichen erheben kann}{gfx/kreaturen/daemon}{\kreaturkampfwerte{1}{10}{4}{8}\trennlinie \kreaturinfo{Unsichtbarkeit}{Ausnahme: Ziel der Jagd}\kreaturinfo{Untoten beseelen}{Der Dämon kann eine Leiche als Untoten erheben oder fährt in einen bestehenden Untoten ein. Er verliert einige Eigenschaften/Attacken und erhält die Eigenschaften/Attacken des Untoten. Der Dämon kontrolliert den Untoten vollständig.)\newline%
}\kreaturvorteile{Rudel}\trennlinie \kreaturwaffe{Ausweichen}{0}{2}{}{}{}\trennlinie \kreaturinfo{Beschwörung}{Invocatio (16), 16 AsP}\kreaturinfo{Dienste}{bestehenden Untoten verstärken, indem der Untote zusätzliche Fähigkeiten (S. 81) im Wert von {-}6 erhält+4 (1 Woche oder bis der Untote zerfällt), eine Leiche als schwachen oder nützlichen Untoten (S. 81) erheben und anschließend im Körper des Untoten einen weiteren Dienst erfüllen+0 (1 Stunde), einen schwachen oder nützlichen Untoten (S. 81) übernehmen und anschließend im Körper des Untoten einen weiteren Dienst erfüllen{-}4 (1 Stunde)}\trennlinie \kreaturinfo{Quelle}{\href{https://ilarisblog.wordpress.com/downloads/}{Ilaris Regeln}}}}
\newcommand{\kreaturdetailnurumbaal}{\kreatur{Nurumbaal}{verfluchter zweigehörnter Goldesel}{gfx/kreaturen/daemon}{\kreaturkampfwerte{8}{6}{8}{6}\trennlinie \kreaturwaffe{Tritt}{1}{8}{10}{2W6+4}{}\trennlinie \kreaturinfo{Beschwörung}{Invocatio}\kreaturinfo{Dienste}{Gold vernichten+4, Herz eines Humanoiden verspeisen, um 1 Stein Gold zu erschaffen+0, Herz eines magiebegabten Humanoiden verspeisen, um 1 Stein Mindorium zu erschaffen{-}4}\trennlinie \kreaturinfo{Quelle}{\href{https://ilarisblog.wordpress.com/downloads/}{Ilaris Regeln}}}}
\newcommand{\kreaturdetailquitslinga}{\kreatur{Quitslinga}{viergehörnter und hochintelligenter Gestaltwandler}{gfx/kreaturen/daemon}{\kreaturkampfwerte{11}{12}{5}{10}\trennlinie \kreaturinfo{Gestaltwandeln}{Der Dämon nimmt die Gestalt einer Person an, von der er einen Körperteil besitzt. Jeder Gestaltwechsel senkt seine WS um 1. In einer fremden Gestalt verliert der Dämon Eigenschaften/Attacken und erhält die Eigenschaften/Attacken der angenommenen Gestalt.}\kreaturvorteile{Regeneration I, Schreckgestalt II, Tarnung}\trennlinie \kreaturwaffe{Tentakel}{2}{8}{12}{2W6+5}{Stumpf}\kreaturkampfvorteile{Niederwerfen, Zusätzliche Attacke IV}\trennlinie \kreaturfertigkeiten{Gebräuche 16, Menschenkenntnis 14, Sinnenschärfe 18, Überreden 16, Wachsamkeit 14}\trennlinie \kreaturinfo{AsP}{64}\kreaturinfo{Illusion}{20 (alle Illusionszauber)}\kreaturinfo{Dämonisch}{12 (einige passende Zauber)}\kreaturinfo{Einfluss}{12 (einige passende Zauber)}\trennlinie \kreaturinfo{Beschwörung}{Invocatio}\kreaturinfo{Dienste}{Diebstahl+4, Ausspionieren von Spielercharakteren+0 (1 Woche), Person verkörpern{-}4 (1 Monat)}\trennlinie \kreaturinfo{Dienstbeginn}{Jeder Dienst des Quitslingas beginnt damit, dass er in einen Wirtskörper einfährt und ihn umformt.}\trennlinie \kreaturinfo{Quelle}{\href{https://ilarisblog.wordpress.com/downloads/}{Ilaris Regeln}}}}
\newcommand{\kreaturdetailshruuf}{\kreatur{Shruuf}{viergehörnter Kampfdämon; großer Gegner}{gfx/kreaturen/daemon}{\kreaturkampfwerte{13}{10}{4}{3}\trennlinie \kreaturvorteile{Schreckgestalt II, Regeneration I}\trennlinie \kreaturwaffe{Schnabel}{0}{2}{16}{4W6+4}{}\kreaturwaffe{Tentakel}{2}{8}{14}{3W6+2}{}\kreaturkampfvorteile{Zusätzliche Attacke IV, Niederwerfen}\trennlinie \kreaturattribute{GE 18, KK 30}\kreaturfertigkeiten{Wachsamkeit 12, Pirschen 0}\trennlinie \kreaturinfo{Beschwörung}{Invocatio}\kreaturinfo{Dienste}{Kampf+4 (1 Minute), Wache+0 (1 Woche), Kampf{-}4 (1 Stunde)}\trennlinie \kreaturinfo{Quelle}{\href{https://ilarisblog.wordpress.com/downloads/}{Ilaris Regeln}}}}
\newcommand{\kreaturdetailtlaluc}{\kreatur{Tlaluc}{zweighörnter, miasmatischer Schneckendämon}{gfx/kreaturen/daemon}{\kreaturkampfwerte{8}{20}{1}{3}\trennlinie \kreaturinfo{Aura}{Gestank, Zähigkeit (28) jede Intitiavephase, kampfunfähigkeit für 1W6 Minuten}\kreaturvorteile{}\trennlinie \kreaturwaffe{Verdauen}{0}{2}{4}{3W6+0}{}\trennlinie \kreaturinfo{Beschwörung}{Invocatio (24), 32 AsP}\kreaturinfo{Dienste}{Fest des Nachbars ruinieren+4, Wache+0 (1 Woche), bewaffnete Gegner angreifen{-}4 (1 Stunde)}\trennlinie \kreaturinfo{Quelle}{\href{https://ilarisblog.wordpress.com/downloads/}{Ilaris Regeln}}}}
\newcommand{\kreaturdetailulchuchu}{\kreatur{Ulchuchu}{eingehörnter, ständig wachsender Algendämon}{gfx/kreaturen/daemon}{\kreaturkampfwerte{9}{14}{1}{3}\trennlinie \kreaturinfo{Verwundbarkeit I}{Feuer}\kreaturvorteile{Regeneration II, Wasserwesen}\trennlinie \kreaturwaffe{Tangschlingen}{2}{4}{12}{2W6+0}{Umklammern ({-}4, 12), Ersäufen (Ein umklammertes Ziel wird unter Wasser gezogen und erleidet in jeder Initiativephase 1 Punkt Erschöpfung. Der Dämon kann bis zu 2 Ziele unter Wasser halten.)}\kreaturkampfvorteile{Zusätzliche Attacke I}\trennlinie \kreaturinfo{Beschwörung}{Invocatio}\kreaturinfo{Dienste}{Ort bewachen+0 (1 Jahr), Kampf{-}4 (1 Minute)}\trennlinie \kreaturinfo{Varianten}{Ein Ulchuchu kann über die Jahre immer weiter wachsen: Großer Ulchuchu (Koloss I, Zusätzliche Attacke II,  AT: +2, Umklammern ({-}6, 16), kann 4 Ziele unter Wasser halten)   Riesiger Ulchuchu (Koloss II, Zusätzliche Attacke IV,  AT: +2, Umklammern ({-}8, 20), kann 8 Ziele unter Wasser halten)\newline%
}\trennlinie \kreaturinfo{Quelle}{\href{https://ilarisblog.wordpress.com/downloads/}{Ilaris Regeln}}}}
\newcommand{\kreaturdetailzant}{\kreatur{Zant}{wendiger Kampfdämon; großer Gegner}{gfx/kreaturen/daemon}{\kreaturkampfwerte{7}{12}{8}{12}\trennlinie \kreaturvorteile{Schreckgestalt II, Regeneration I, Rudel}\trennlinie \kreaturwaffe{Prankenhieb}{1}{8}{15}{2W6+3}{Wendig}\kreaturwaffe{Biss}{0}{2}{13}{3W6+2}{}\kreaturwaffe{Schwanz}{1}{2}{13}{2W6+0}{Niederwerfen}\kreaturkampfvorteile{Niederwerfen, Sturmangriff, im Reiterkampf}\trennlinie \kreaturattribute{GE 28, KK 24}\kreaturfertigkeiten{Laufen 16, Pirschen 10, Wachsamkeit 10}\trennlinie \kreaturinfo{Beschwörung}{Invocatio (20), 24 AsP}\kreaturinfo{Dienste}{Barrikade zerstören+4, Kampf+0 (1 Minute), Reittier{-}4 (1 Stunde)}\trennlinie \kreaturinfo{Quelle}{\href{https://ilarisblog.wordpress.com/downloads/}{Ilaris Regeln}}}}
\newcommand{\kreaturdetaileule}{\kreatur{Eule}{Häufiges Vertrautentier}{gfx/kreaturen/tier}{\kreaturkampfwerte{2}{4}{1}{1}\trennlinie \kreaturinfo{Angepasst II (Dunkelheit)}{Angepasst I/II (Umgebung) Durch deine Spezies oder langjährige Erfahrung hast du dich an eine bestimmte Umgebung oder Umweltbedingung gewöhnt. Abzüge durch diese Umgebung (Beispiele auf S. 38), insbesondere im Kampf, sinken für dich um eine/zwei Stufen. Die Kosten für Angepasst legt der Spielleiter fest, wobei er sich an der Häufigkeit der Umgebung orientieren sollte. Zu allgemein gefasste Umgebungen wie „unsicherer Untergrund“ sollte er nicht zulassen. Beispiele für Angepasst sind: • Dunkelheit: verringert Abzüge durch schlechte Lichtverhältnisse (40 EP pro Stufe) • Schnee: verringert Abzüge durch schneebedeckten oder eisigen Untergrund (20 EP pro Stufe) • Wasser: verringert Abzüge durch knie{-} oder hüfttiefes Wasser und unter Wasser (20 EP pro Stufe) • Wald: verringert Abzüge durch Wurzeln, Gestrüpp und dichtes Unterholz (40 EP pro Stufe) Voraussetzungen: keine/Angepasst I Nachkauf: häufig/selten\newline%
}\kreaturvorteile{Flieger}\trennlinie \kreaturwaffe{Krallen}{0}{6}{13}{1W6+0}{}\trennlinie \kreaturattribute{KK {-}8, KL {-}4, MU 4}\kreaturfertigkeiten{Pirschen 16, Wachsamkeit 16, Zähigkeit {-}4}\trennlinie \kreaturinfo{Quelle}{\href{https://ilarisblog.wordpress.com/downloads/}{Ilaris Regeln}}}}
\newcommand{\kreaturdetailkatze}{\kreatur{Katze}{Häufiges Vertrautentier}{gfx/kreaturen/tier}{\kreaturkampfwerte{2}{4}{6}{6}\trennlinie \kreaturinfo{Angepasst II (Dunkelheit)}{Angepasst I/II (Umgebung) Durch deine Spezies oder langjährige Erfahrung hast du dich an eine bestimmte Umgebung oder Umweltbedingung gewöhnt. Abzüge durch diese Umgebung (Beispiele auf S. 38), insbesondere im Kampf, sinken für dich um eine/zwei Stufen. Die Kosten für Angepasst legt der Spielleiter fest, wobei er sich an der Häufigkeit der Umgebung orientieren sollte. Zu allgemein gefasste Umgebungen wie „unsicherer Untergrund“ sollte er nicht zulassen. Beispiele für Angepasst sind: • Dunkelheit: verringert Abzüge durch schlechte Lichtverhältnisse (40 EP pro Stufe) • Schnee: verringert Abzüge durch schneebedeckten oder eisigen Untergrund (20 EP pro Stufe) • Wasser: verringert Abzüge durch knie{-} oder hüfttiefes Wasser und unter Wasser (20 EP pro Stufe) • Wald: verringert Abzüge durch Wurzeln, Gestrüpp und dichtes Unterholz (40 EP pro Stufe) Voraussetzungen: keine/Angepasst I Nachkauf: häufig/selten\newline%
}\kreaturvorteile{}\trennlinie \kreaturwaffe{Biss}{0}{12}{12}{1W6+0}{}\trennlinie \kreaturattribute{KK {-}8, KL {-}6, MU 0}\kreaturfertigkeiten{Pirschen 20, Wachsamkeit 12, Zähigkeit {-}4}\trennlinie \kreaturinfo{Quelle}{\href{https://ilarisblog.wordpress.com/downloads/}{Ilaris Regeln}}}}
\newcommand{\kreaturdetailkleinaffe}{\kreatur{Kleinaffe}{Häufiges Vertrautentier}{gfx/kreaturen/tier}{\kreaturkampfwerte{2}{1}{5}{3}\trennlinie \kreaturwaffe{Biss}{0}{12}{9}{1W6{-}1}{}\trennlinie \kreaturattribute{FF 6, KK {-}6, KL {-}6, MU {-}2}\kreaturfertigkeiten{Pirschen 16, Wachsamkeit 12, Zähigkeit {-}4}\trennlinie \kreaturinfo{Quelle}{\href{https://ilarisblog.wordpress.com/downloads/}{Ilaris Regeln}}}}
\newcommand{\kreaturdetailkroete}{\kreatur{Kröte}{Häufiges Vertrautentier}{gfx/kreaturen/tier}{\kreaturkampfwerte{1}{8}{1}{{-}1}\trennlinie \kreaturinfo{Magiegespür}{In der Nähe astraler Kräfte überfällt dich ein Frösteln, du hörst sphärische Klänge oder ein Farbschleier legt sich für dich über die Umgebung. Mit dem Talent Sinnenschärfe kannst du Intensitätsanalysen (S. 80) von magischen Gegenständen durchführen. Nach dem aktiven Einsatz der Gabe erleidest du einen Punkt Erschöpfung. Voraussetzungen: 60 EP Nachkauf: extrem selten\newline%
}\kreaturvorteile{}\trennlinie \kreaturwaffe{Ausweichen}{0}{6}{}{}{}\trennlinie \kreaturattribute{KK {-}16, KL {-}4, MU 0}\kreaturfertigkeiten{Pirschen 28, Sinnenschärfe 12, Wachsamkeit 12, Zähigkeit {-}8}\trennlinie \kreaturinfo{Quelle}{\href{https://ilarisblog.wordpress.com/downloads/}{Ilaris Regeln}}}}
\newcommand{\kreaturdetailrabe}{\kreatur{Rabe}{Häufiges Vertrautentier}{gfx/kreaturen/tier}{\kreaturkampfwerte{2}{2}{1}{0}\trennlinie \kreaturvorteile{Flieger}\trennlinie \kreaturwaffe{Krallen}{0}{8}{12}{1W6+1}{}\trennlinie \kreaturattribute{KK {-}8, KL {-}2, MU {-}2}\kreaturfertigkeiten{Pirschen 12, Wachsamkeit 18, Zähigkeit {-}4}\trennlinie \kreaturinfo{Quelle}{\href{https://ilarisblog.wordpress.com/downloads/}{Ilaris Regeln}}}}
\newcommand{\kreaturdetailschlange}{\kreatur{Schlange}{Häufiges Vertrautentier}{gfx/kreaturen/tier}{\kreaturkampfwerte{2}{6}{2}{0}\trennlinie \kreaturinfo{Magiegespür}{In der Nähe astraler Kräfte überfällt dich ein Frösteln, du hörst sphärische Klänge oder ein Farbschleier legt sich für dich über die Umgebung. Mit dem Talent Sinnenschärfe kannst du Intensitätsanalysen (S. 80) von magischen Gegenständen durchführen. Nach dem aktiven Einsatz der Gabe erleidest du einen Punkt Erschöpfung. Voraussetzungen: 60 EP Nachkauf: extrem selten\newline%
}\kreaturvorteile{}\trennlinie \kreaturwaffe{Biss}{0}{8}{13}{1W6+1}{Gifte je nach Schlange}\trennlinie \kreaturattribute{KK {-}8, KL {-}6, MU {-}4}\kreaturfertigkeiten{Pirschen 24, Sinnenschärfe 10, Wachsamkeit 16, Zähigkeit {-}4}\trennlinie \kreaturinfo{Quelle}{\href{https://ilarisblog.wordpress.com/downloads/}{Ilaris Regeln}}}}
\newcommand{\kreaturdetailspinne}{\kreatur{Spinne}{Häufiges Vertrautentier}{gfx/kreaturen/tier}{\kreaturkampfwerte{1}{8}{1}{0}\trennlinie \kreaturwaffe{Biss}{0}{8}{11}{1W6{-}2}{Gifte je nach Spinne}\trennlinie \kreaturattribute{KK {-}16, KL {-}6, MU {-}6}\kreaturfertigkeiten{Pirschen 28, Wachsamkeit 12, Zähigkeit {-}8}\trennlinie \kreaturinfo{Quelle}{\href{https://ilarisblog.wordpress.com/downloads/}{Ilaris Regeln}}}}
\newcommand{\kreaturdetailzombie}{\kreatur{Zombie}{leicht zu erhebender schwacher Untoter}{gfx/kreaturen/untot}{\kreaturkampfwerte{5}{5}{2}{1}\trennlinie \kreaturinfo{Astralsinn}{Astralsinn erlaubt es der Kreatur, ihre Umgebung magisch wahrzunehmen. Sie erleidet keine Abzüge durch schlechte Sicht. Der Astralsinn kann durch Antimagie, zum Beispiel Hellsicht trüben in der Modifikation Magie unterdrücken, gestört werden. Die Schwierigkeit dafür liegt mindestens bei 20, bei mächtigen Wesen deutlich höher.\newline%
}\kreaturinfo{Resistenz II}{Stichwaffen}\kreaturvorteile{Rudel}\trennlinie \kreaturwaffe{Hände}{1}{4}{10}{1W6+2}{Umklammern({-}4) nur ab 2 Zombies pro Opfer}\kreaturwaffe{Biss}{0}{4}{8}{1W6{-}1}{Hochansteckend}\trennlinie \kreaturattribute{GE 0, KO 24, KK 8, MU 24}\trennlinie \kreaturinfo{Beschwörung}{Skelettarius, Totes Handle!}\kreaturinfo{Dienste}{Kampf (1 Minute)+0, Wache (1 Tag, mit Totes handle! permanent, 4)+0}\trennlinie \kreaturinfo{Varianten}{Mit Nephazz’ besetzt zus. Eig. S. 81, Mutation (nekrotische Veränderung)}\kreaturinfo{Hochansteckend}{Jede erlittene Wunde erhöht die Chance einer Ansteckung mit Wundbrand oder einer anderen Krankheit um 50\%. Eine Probe auf Gifte und Krankheiten (24) eliminiert das Risiko}\trennlinie \kreaturinfo{Quelle}{\href{https://dsaforum.de/viewtopic.php?f=180&t=57100\#p2048310}{Heiners Zombies}}}}
\newcommand{\kreaturdetailfetterzombie}{\kreatur{Fetter Zombie}{nützlicher Untoter}{gfx/kreaturen/untot}{\kreaturkampfwerte{5}{5}{1}{1}\trennlinie \kreaturinfo{Astralsinn}{Astralsinn erlaubt es der Kreatur, ihre Umgebung magisch wahrzunehmen. Sie erleidet keine Abzüge durch schlechte Sicht. Der Astralsinn kann durch Antimagie, zum Beispiel Hellsicht trüben in der Modifikation Magie unterdrücken, gestört werden. Die Schwierigkeit dafür liegt mindestens bei 20, bei mächtigen Wesen deutlich höher.\newline%
}\kreaturinfo{Explosion}{3W6, Säure}\kreaturvorteile{}\trennlinie \kreaturwaffe{Hände}{1}{4}{12}{1W6+4}{Niederwerfen(AT{-}4)}\kreaturwaffe{Biss}{0}{4}{10}{1W6{-}1}{Hochansteckend}\trennlinie \kreaturattribute{GE {-}4, KO 24, KK 12, MU 24}\trennlinie \kreaturinfo{Beschwörung}{Skelettarius, Totes Handle!}\kreaturinfo{Dienste}{Kampf (1 Minute)+0, Wache (1 Tag, mit Totes handle! permanent, 4)+0}\trennlinie \kreaturinfo{Varianten}{Mächtige Magie Explosion(4W6, Säure)}\kreaturinfo{Beschwörung}{Aus fettleibiger oder aufgedunsener Leiche}\kreaturinfo{Hochansteckend}{Jede erlittene Wunde erhöht die Chance einer Ansteckung mit Wundbrand oder einer anderen Krankheit um 50\%. Eine Probe auf Gifte und Krankheiten (24) eliminiert das Risiko}\trennlinie \kreaturinfo{Quelle}{\href{https://dsaforum.de/viewtopic.php?f=180&t=57100\#p2048310}{Heiners Zombies}}}}
\newcommand{\kreaturdetailtlaluczombie}{\kreatur{Tlaluczombie}{nützlicher Untoter}{gfx/kreaturen/untot}{\kreaturkampfwerte{4}{5}{2}{1}\trennlinie \kreaturinfo{Astralsinn}{Astralsinn erlaubt es der Kreatur, ihre Umgebung magisch wahrzunehmen. Sie erleidet keine Abzüge durch schlechte Sicht. Der Astralsinn kann durch Antimagie, zum Beispiel Hellsicht trüben in der Modifikation Magie unterdrücken, gestört werden. Die Schwierigkeit dafür liegt mindestens bei 20, bei mächtigen Wesen deutlich höher.\newline%
}\kreaturinfo{Aura}{Gestank, Zähigkeit (12) alle 4 Intitiavephasen, Kampfunfähigkeit für 2 Initiativphase}\kreaturvorteile{}\trennlinie \kreaturwaffe{Tlalucs Odem}{8}{}{}{}{gelingt autom., nur erste AT}\kreaturwaffe{Hände}{1}{4}{12}{1W6+2}{}\kreaturwaffe{Biss}{0}{4}{10}{1W6{-}1}{Hochansteckend}\trennlinie \kreaturattribute{GE 0, KO 20, KK 8, MU 24}\trennlinie \kreaturinfo{Beschwörung}{Skelettarius, Totes Handle!}\kreaturinfo{Dienste}{Kampf (1 Minute)+0, Wache (1 Tag, mit Totes handle! permanent, 4)+0}\trennlinie \kreaturinfo{Varianten}{Mächtige Magie (Gestank, Zähigkeit (16) alle 4 Intitiavephasen, Kampfunfähigkeit für 4 Initiativphase)}\kreaturinfo{Hochansteckend}{Jede erlittene Wunde erhöht die Chance einer Ansteckung mit Wundbrand oder einer anderen Krankheit um 50\%. Eine Probe auf Gifte und Krankheiten (24) eliminiert das Risiko}\trennlinie \kreaturinfo{Quelle}{\href{https://dsaforum.de/viewtopic.php?f=180&t=57100\#p2048310}{Heiners Zombies}}}}
\newcommand{\kreaturdetailzombiehauptmann}{\kreatur{Zombiehauptmann}{starker Untoter}{gfx/kreaturen/untot}{\kreaturkampfwerte{4}{5}{2}{1}\trennlinie \kreaturinfo{Astralsinn}{Astralsinn erlaubt es der Kreatur, ihre Umgebung magisch wahrzunehmen. Sie erleidet keine Abzüge durch schlechte Sicht. Der Astralsinn kann durch Antimagie, zum Beispiel Hellsicht trüben in der Modifikation Magie unterdrücken, gestört werden. Die Schwierigkeit dafür liegt mindestens bei 20, bei mächtigen Wesen deutlich höher.\newline%
}\kreaturinfo{Untotenanführer}{Ein Untotenanführer wird automatisch mit W6 Gefolgsleuten ohne weitere Kosten für den Beschwörer gerufen, wenn entsprechendes „Material” verfügbar ist. Die Gefolgsleute erheben sich gemäß ihrem Verwesungsgrad als gewöhnliche Skelette, Zombies, Mumien oder Leichname und können nicht weiter mit Eigenschaften versehen werden. Sie unterstützen alle Dienste, die dem Anführer aufgetragen werden. Zudem verfügen sie über das Manöver Umklammern(AT{-}4). Der Anführer verfügt über einen eventuellen Kriegskunst{-}TaW, der halb so groß ist wie der TaW zu Lebzeiten des erhobenen Wesens (1W6, wenn nicht bekannt).\newline%
}\kreaturvorteile{}\trennlinie \kreaturwaffe{Hände}{1}{6}{14}{1W6+2}{Umklammern({-}4) falls Gefolgsleute vorhanden}\kreaturwaffe{Biss}{0}{6}{12}{1W6{-}1}{Hochansteckend}\kreaturkampfvorteile{Entfernung verändern, Entwaffnen, Umreißen, Wuchtschlag}\trennlinie \kreaturattribute{GE 2, KO 24, KK 12, MU 24}\trennlinie \kreaturinfo{Beschwörung}{Skelettarius, Totes Handle!}\kreaturinfo{Dienste}{Kampf (1 Minute)+0, Wache (1 Tag, mit Totes handle! permanent, 4)+0}\trennlinie \kreaturinfo{Varianten}{Mächtige Magie 1 zus. Fähigkeit von S. 81}\kreaturinfo{Hochansteckend}{Jede erlittene Wunde erhöht die Chance einer Ansteckung mit Wundbrand oder einer anderen Krankheit um 50\%. Eine Probe auf Gifte und Krankheiten (24) eliminiert das Risiko}\kreaturinfo{Untotenanführer}{}\trennlinie \kreaturinfo{Quelle}{\href{https://dsaforum.de/viewtopic.php?f=180&t=57100\#p2048310}{Heiners Zombies}}}}